\documentclass[10pt, a4paper]{article}
    \author{ianaesthetic}
    \title{3d Game Programming with DirectX 11}
\usepackage{indentfirst, amsmath, fontspec, listings, xcolor, amssymb}
%\usepackage{verbatim}

\usepackage{xeCJK}
\setCJKmainfont{Microsoft YaHei}

\usepackage{url}

\newfontfamily\consolas{Consolas}
\lstset{numberstyle = \small\consolas, basicstyle=\small\consolas}

\XeTeXlinebreaklocale "zh"
\XeTeXlinebreakskip = 0pt plus 1pt
\setmainfont{Consolas}

\begin{document}
    \section{中国腾飞的原因}
        \subsubsection{完整的工业化体系}
            中国2017年现有的工业体系(点进去404网址后面加.html)

            \url{http://www.stats.gov.cn/tjsj/tjbz/hyflbz/201710/t20171012_1541679.html}
    
            来源:苏联扶住中国的156个工业形成基础的工业体系; 西方的援助
            
            成就:现有工业产值 中美对比图 (excel 文件)
        \subsubsection{廉价劳动人口}
            数据:labor force participate rate \& labor force total \& education rate \& 人均gdp ? 

        \subsubsection{制造业出口导向}
            汇率变化 1995年大幅贬值以刺激出口
        \subsubsection{外国投资}
            Foreign direct investment, net inflows
\section{中国现有问题}    

    主要讨论的时间段: 2008年全球金融危机至今。

    \subsection{制造业自身问题:人口红利消失}

        技术升级加上初期的高额出口回报导致了过量生产。

        计划生育带来的持续人口增速降低,社会老年化,内部需求降低,经济缺乏活力
        数据:新生儿数目

        2008年以来的经济危机使得外需极大减小,在内部需求减小的情况下造成制造业产能过剩 
        数据:PPI

        劳动人口减少,劳动力价格升高,导致制造业竞争力下降,原有的中国制造的业务被东南亚地区抢占,并且国内企业成本升高。 (新闻)

        制造业的利润率下降。

    \subsection{实体经济无法得到有效投资}
        在2008年金融危机的大背景下,央行开始持续向市场注资 Money Policy (M1, M2(10\%),CPI增速(1.5\%)) $\Rightarrow$  制造业没有得到资金

        原因: 

        1.中国金融结构简单,以银行贷款为主,中小企业融资困难。

        2.投资进入了大企业的手中,由于重复性投资,效率下降 (MPK sloping down) 
        
        3.由于实业投资回报下降,实体经济无法得到上层投资的收益,因为资本会流向回报率更高的虚拟经济部分。(供需曲线) 
        
        4.2008年后的4万亿投资,市场转向房地产和基础建设,投机倾向高。(08年后 GDP 增速贡献)
        
        结果:货币政策失效 

    \subsection{负债率高}
        体现在企业负债率高筑,同时居民负债率快速增长。 负债率导致信用度下降,最后造成融资成本上升 。

        企业负债率,居民负债率
        
    \subsection{贫富差距}
        贫富差距过 导致 自由市场竞争收到威胁 
\end{document}
