\documentclass[10pt, a4paper]{article}
    \author{ianaesthetic}
    \title{3d Game Programming with DirectX 11}
\usepackage{indentfirst, amsmath, fontspec, listings, xcolor, amssymb}
%\usepackage{verbatim}

\usepackage{xeCJK}
\setCJKmainfont{Microsoft YaHei}

\usepackage{url}

\newfontfamily\consolas{Consolas}
\lstset{numberstyle = \small\consolas, basicstyle=\small\consolas}

\XeTeXlinebreaklocale "zh"
\XeTeXlinebreakskip = 0pt plus 1pt
\setmainfont{Consolas}

\begin{document}
    \section{中国腾飞的原因}
        \subsection{第二产业}
        \subsubsection{完整的工业化体系}
            中国2017年现有的工业体系(点进去404网址后面加.html)

            \url{http://www.stats.gov.cn/tjsj/tjbz/hyflbz/201710/t20171012_1541679.html}
    
            来源:苏联扶住中国的156个工业形成基础的工业体系; 西方的援助
            
            成就:现有工业产值 中美对比图 (excel 文件)
        \subsubsection{廉价劳动人口}
            数据:labor force participate rate \& labor force total \& education rate \& 人均gdp ? 

        \subsubsection{制造业出口导向}
            汇率变化 1995年大幅贬值以刺激出口
        \subsubsection{外国投资}
            Foreign direct investment, net inflows
        \subsection{第三产业}
\section{中国现有问题}
    \subsection{人口红利消失}
    \subsection{制造业产能过剩}
    \subsection{负债率高}
    \subsection{缺乏新的经济增长点}
\end{document}
