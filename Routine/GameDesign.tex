\documentclass[12pt, a4paper]{article}   
\usepackage{indentfirst, amsmath, fontspec, listings, xcolor, amssymb}

\setmainfont{Microsoft YaHei}
    
\newfontfamily\consolas{Consolas}
\lstset{numberstyle = \small\consolas, basicstyle=\small\consolas}
    
\XeTeXlinebreaklocale "zh"
\XeTeXlinebreakskip = 0pt plus 1pt

\begin{document}
    \section*{基础信息}
    游戏名称:破碎之忆

    游戏类型:ARPG

    设计理念:
    
        \quad 没总结好,可以说是乱设计了

    \noindent /*
   
    这一页由于还没有填充好,稍微写一下观看指南:
        
        本版本还没有太多涉及到游戏的具体实现和关卡设计,大多数是剧情设定

        语句不通顺等请无视,理解至上

        大部分的设定仅为初稿,大部分是在框架范围, 甚至可能是我突然想到然后先写上去作为参考的文字

        在讨论结束后请删去。。我不想设定被流传出去,虽然应该不太可能。。太羞耻了

        看的话可以直接从后面的故事梗概开始,前面的可以先行略过,大部分的设定也还没有争对剧情发展进行调整。

    \noindent */
    \newpage 

    \section*{游戏背景设定和特殊设定解释}
        
        \textbf{意识世界}:每一个人都有期望的世界,而这样的幻景将被抽象影响着整个意识。在当前的认知和记忆的作用下,一个基础的所谓意识世界将会被搭建。而在人自我愿景的影响下,这个世界也戴上了更多的主观色彩。意识世界具有相似的结构,大部分都是由记忆填充的:记忆单元作为每一个记忆片段的主体,而这个主体具体是什么则各不相同:那段记忆中的自己,和自己交流的父母,常伴身边的宠物;所有被意识主体感知为有意识的东西都能作为记忆单元而表现出来。而记忆所涉及的其它的内容,则笼统的融合到了意识世界的环境当中。而真正被称为意识的逻辑过程则支配了每一个记忆单元在这个世界的行为和抉择,最终这些抉择的结果将改变记忆的形态,改变这个世界。严格的说,意识主体和意识世界是不断交互和影响的,其反应为认知和意识的不断交流以产生实际生活中的实际行为。 
        
       而在很长的一段时间内,这样的世界都是不可知的。对于意识主体以外的人而言,意识主体的所思所想自然不得而知;而对于意识主体本身,这样的一个世界只是一大段大脑活动的附属品。而现在,研究人员协力终于得知了那个世界的存在,而且能够从意识主体之外
        观察和影响到意识世界。 


        \medskip
        \textbf{记忆单元}:作为意识的重要组成部分,记忆是以单元的形式存在的。所有的记忆共享着意识主体的意识。然而,与其被称为共享,支配一词更加合适。意识主体对待记忆片段的看法决定了记忆在这个意识世界的体现,同时也决定了不同记忆对于意识主体的影响。举例而言,对于同一段时间的两种不同的记忆将会最终留存下一个作为真正的记忆,而这样的决定过程在意识世界中被表现为记忆单元的死亡和周围环境的崩塌,消失。记忆单元之间互动的结果实际上就是意识主体最后立场的表现,这将直接影响意识主体在现实当中的行动。 

        //更加细致的说明!

        记忆的真假跟记忆单元本身的内容无关,而是意识主体的意识活动的结果。这种真假是完全独立于事实上的真假,而是主观上的,会对于意识主体的行为产生直接影响。即使是相同的记忆单元,不同的意识主体对于此段记忆的看法会截然不同,这表示记忆在不同的意识上的表现也将不尽相同

        \medskip
        \textbf{意识的扩散和影响}:意识本身包好了很多的东西:性格,理想,愿望等。如上所述,意识最终决定了记忆单元的在一个人意识世界中的体现,而这种体现通过共享意识进行与其他记忆进行反应最终决定意识主体的行为。而对某些强烈的意志,在合适环境的诱导之下,他们甚至能够对于他人的记忆单元进行操纵和修改,以达到控制他人行为的目的。从外界通过科技对人的记忆进行篡改基于相似的原理,而在0125号实验体将是第一个体现出如此能力的个体。

        \medskip
        \textbf{记忆篡改}:记忆篡改的目的是为了改变一个人在现实生活中的抉择。有两种方式实现记忆的篡改:在意识世界中制造假的记忆碎片并被意识主体所接受;在已有的记忆碎片中进行有意的额外的修改。无论使用那种方法,对于想要篡改记忆以达到意识控制的势力来说,其实都是诱导意识主体去相信虚假的事实来达到控制的目的,而从原理上和洗脑非常相似:首先通过对于实验题的记忆进行影响而试图孤立实验体在社会上的地位认知,已让意识感受到无助和绝望;在如下的条件下,篡改的记忆更容易被接受,而甚至最后完全丧失意识的主动性。由于它能够从记忆层次影响人的判断,同时由于向实验体输出绝望和孤独感拥有更加通用的方式:通过篡改关系相近的人的相关记忆,使得他们对于周围的社会关系产生绝望,达到这一步基本已经成功了,在大量的可观察个体中还未曾发现例外,直到0125号实验体,产生了完全不同的效应。
        
        \textbf{0125号实验体}:0125号的特殊性在于:1.任何形式的记忆篡改都没有办法对其真实的记忆和意识认知产生任何的影响,但是会留下所有的篡改痕迹,将会以额外的记忆碎片存在于0125的意识世界中。2.能够在小范围内进行意识散播;在本体意识急剧波动时产生,能向外急剧的输出主观意识修正的记忆修改。但是从本质上,这个和记忆篡改是不同的:记忆篡改在最初的阶段通过找到被修改者与世界的感情联系然后恶意扭曲所有相关的记忆碎片以产生负面情绪和意识,而此处的记忆修改是直接强制的无视所有的被输出者的所有社会经历,以0125号个人意志所希望的行为而直接强行使得意志在瞬间对此行为进行执行。具体来说,为了使得意志执行,将会突然大量的在被输出者的头脑者创建拥有极强倾向性的记忆去迷惑意识,而最终取代原有的记忆。两者的主要区别就在于这种修改是否源生于特定的倾向。暂定为唯一的特殊个体,后续可继续扩展。

        \textbf{男主和女主}:

        //本段文档旨在确定在设计之处对于男女关系的设想和对于两人的定位,可能会稍有不适(趴
        //由于设计者没有女朋友而且曾经也没有过,所以设计者不知道怎么写这份关系

        所有的角色行为都建立在无穷的保护欲之上:父亲由于社会失去了母亲所以拼命的保护独生的儿子,男主由于年幼丧母以及孤立的社会环境而失败。

    \newpage        
    \section*{角色设计}

        \textbf{父母}: 束章平 / 林沁宜
            父亲早年为精英特种小队的队长,母亲则是政府脑科学研究所的研究员。由于在记忆领域的不断进展,整个小队被要求秘密保护整个脑科学研究所。然而,由于神秘组织为了能够得到更多的科学家资源,逼迫所有的研究员转移到组织名下的研究机构而架空掉整个研究所。母亲同时因为公开反对和身孕而远离了研究所,但仍然在医院惨遭毒手。组织并没有对新生的男主出手,是因为他们自信在医院的医疗事故能够摆平一切,而区区一个单亲家庭是不会对组织产生任何的威胁。从母亲死去的那天开始,父亲就选择了退役,并开办了一家安保公司作为独立的力量以保护自己的仅存的家人,同时对于自己的儿子守口如瓶,决心让儿子脱这阴谋的怪圈。

        \textbf{男主}: 束康

            “父亲告诉我:‘你生下来了,就是唯一的火种。’” 
            
            在出生之后几天,母亲离奇死去,死因为:伤口感染恶化至死亡。自幼接受父亲的知道,一直保持锻炼,学习使用各式武器。由于天生缺乏母爱,不善言谈,但天性认真,勤奋。因为这样的性格,在学校里面虽然因为自身的实力而让人畏惧,但是也不乏嘲笑和恶意。
            


            现25岁。

            强壮,稚气,复杂融合,迷茫。 正直。

            由于丧母一直处于精神的不稳定和高度的抑郁当中,导致了强烈的对外恐惧和自闭。

        \medskip
        \textbf{女主}: 符依


            男主的妻子,自幼认识男主,强大的意志力:勇敢,乐观。守护。 
            
    \newpage 

    \section*{故事梗概}

    \noindent /*

    大部分的信息都将不会在游戏内显式获得,这里写的较为详细是为了保证逻辑经得起足够的推敲。当然也会有部分牵强的部分,请一并提出

    \noindent */

    \subsection*{大的世界观设定}    

    //我显著的认识到了我是一个政治白痴,所以这段可能槽点过多
    
    总体的科技水平随着基础物理的跳跃发展,结合信息时代的爆炸迎来了地球文明的黄金时期。但是同时, 贫富差距却不断拉大,经济寡头化严重。受限于在蓬勃的科技发展中被忽略的落后经济体系(不要问我为什么落后),大量的财富被寡头所收集,大量的普通人甚至无法享受科技发展带来的福利。逐渐走高的犯罪率和日益升高的生活成本,使得平民的生活没有办法进一步提高,长期的原地踏步。\textbf{21世纪70年代中期},世界各个国家开始相继爆发经济危机,使得正常的社会秩序被进一步打乱,社会矛盾更加突出。从此刻,全世界开始动荡:多数国家开始发生政权的不正常更迭,大的利益集团开始收割压榨普通民众的利益,战争相继爆发,地球文明由盛转衰。

    %cSpell: ignore Hausos

    \subsection*{背后的组织 -- Hausos}

    \textbf{20世纪中叶},由于同时代的科技进步,以脑科学为代表的生物科学获得了难得的发展契机。随着研究的深入,科学家逐渐发现意识世界的存在以及意识与记忆之间的关系。虽然经过数年的研究,他们依然没有解开意识形成的底层原因,但是自从记忆单元这个概念被提出后,控制一个人的记忆成为了可能。由于社会持续动荡,神秘组织(我真的想不出来名字)开始策划利用根植的影响力妄图掌握新兴技术而达到独立的霸权地位。在脑科学上,他们瞄准了当时位于世界顶尖的某研究所,通过威逼利诱的方式想尽办法将科学家收入麾下,而对于少部分不听话的学者则抹杀致死,没有任何例外。\textbf{2066年},研究所正式被架空,而组织内部也紧锣密鼓的进行着相关的研究。

    随着研究的不断深入,研究所花费了二十年的时间,最终开始获知到了意识世界和记忆单元的存在。虽然相关的研究并没有成熟,组织迅速的将研究方向调整为记忆修改与意识引导,以期望最终实现对于大众的意识控制。通过普通科学实验的名号和丰厚的报酬吸引了生活艰难的普通群众充当实验材料。而实际上,大部分初期的实验体最终在实验结束都伴随着精神恍惚,反应速度下降,而换来的就是组织对其意识的主导权。为了追求更好的实际效果,组织需要越来越大量的数据用以研究,绑架低层次阶级和无背景平民成为了更加简单粗暴的选择,同时利用自己在暗处的影响力也能够轻松的消除绑架所带来的负面影响。\textbf{2090年},他们将目标选在了在记录中没有父母的0125号实验体。

    \subsection*{支离破碎的家庭}
        
    男主的一家曾经是一个小康的中产阶级。父亲成年参军,效力于特种部队数年,作为小队的队长。由于社会的动荡,政府为了保证科研的继续发展,指派父亲带领这支特殊力量保护某脑科学的研究所。母亲则是脑科学研究的专家,在收到父亲保护之时相识相爱,结为夫妻。\textbf{2065年},母亲怀孕,同时所属的研究所正处于危机当中。她严厉的拒绝了组织向她开出的条件,并以身孕为由处于休假状态。\textbf{2066年},男主出生。然而就在男主出生后的第二天,母亲在医院里面被下了神秘组织下了最后通牒:加入或是死亡。母亲自然不愿意屈服,并于父亲商量这样的计划:凭借父亲特殊的身份以及社会关系,同时在院方熟人的帮助下,母亲将被秘密转移到其他的子医院躲避组织的追杀,同时由院方和父亲带领的小队近身保护母亲的安全,在此期间通过释放不明确的信号来拖时间,在母亲痊愈后立马离开这个国家。然而,组织已经没有耐心等下去了,他们自信获得了足够的科学家资源,通过安插在医院的护士进入到母子二人的病房,暗杀了母亲,并将母亲的死亡认定为外伤感染,不了了之。

    备受刺激的父亲愤然离开了军队,与少部分知晓真相的战友开创了私人安保公司。他们对于政府和现状感到无比的失望,无论是因为作为丈夫看到妻子死在了自己面前,还是因为保护许久的科研人才就此殒命。这一沉痛的教训也让父亲变得十分的谨慎:他在表面上服务于各大利益集团,作为佣兵收取高额的费用同时获得了广泛的社会门路;在暗地里收集有关组织的资料,同时通过各种方法,掌握前沿的军事科技以备有朝一日为妻子报仇。从那时起,父亲只有一个念头:尽可能的保护好唯一的下一代,同时为妻子报仇。

    \subsection*{隐藏的资质和正义}  
        从男主出生开始,父亲并没有执意将他培养为一个战士,但是也十分注重从小的体能训练,以人生的经历教导他。自从上小学之后,他便具有超过同龄人的体能和力量。然而,表面的力量之下确是十分脆弱的内心:丧母之痛对于男主的性格造成了极大的影响,有着强烈的自闭倾向,同时感情长期压于心中。父亲本来期望他能够在小学的时候有所改观,但男主却因为丧母以及不善与人交谈长期收到言语上的欺凌,最终遭到孤立。这加剧了男主的自闭症。

        父亲为了让男主能够更多的融入社会,希望能够带着他多去和外面的世界接触。所以即使天下并不太平,他也常常带男主出来走走。

        \textbf{2078年8月12日},父亲如往常一样带男主来到城市中最大的公园散步。同一时间,公园周围的银行遭到了袭击。然而这并不是往常意义上的抢劫,而是蓄意已久的阴谋。此时,灾难性的经济危机已经开始,银行为了保住资产都保有着大量的现金,所以最近抢劫银行的事情已经屡见不鲜。然而这次袭击实际上是为了逼迫现有政府的一次恐怖袭击:有大量的群众想要在经济危机中取出现金而留在了银行,而银行前面的公园也提供了绝好的舞台。/*我还没有设定好这次行动的势力,可以考虑是组织,也可以为更大的世界观留坑*/ 他们计划控制住尽可能多的普通民众,再等待媒体到达的时候能够逐一的枪杀所有的被绑架者。

        听到枪声后,拥有广泛作战经验的父亲立刻意识到了事情的严重性:袭击者拥有非常好的武器装备,同时纪律严明,绝非等闲之辈。他犹豫着是要维护这个城市的安全,还是要优先保证儿子的安全。通过了私人关系通报了位于警局的朋友之后,他最终决定出动自己的保卫力量来协助缺少经费,疲于奔命的警方:警方将在正面包围所有的匪徒,并尝试进行沟通。而父亲的特殊力量则负责潜入击杀所有的敌人,父亲则作为指挥的参谋留守在正面,同时将儿子安置在后方的装甲车当中。此时,入侵者假借着提出人质与现金的交换,将几个普通民众带到了银行前面的平台上面,在时机成熟之时,展开了一层护盾(防止狙击手)/*大世界设定,也是为什么这游戏是 arpg 而不是tps的设定*/, 开始逐步执行枪决,并开始一大段的对于软弱政府的攻击。情急之下,父亲只能等待着暗中的队友传来就位的消息以一击击溃对手。在得到所有的就为的信息之后,他下达了进攻的指令,准备瞬间除掉所有的在场的敌人,以救出所有的人质。对他而言,他需要干净利落的解决现在正在持枪对向一个小女孩的枪手以保证不再有更多的伤亡,然后破坏掉发生器使得外场的狙击手和其他的警员发挥作用。正在他发动进攻时,一个幼小的身姿瞬间冲上前去,从站在前面正在大放厥词的军官一样的角色的侧腰抽出匕首并刺向了枪手。父亲注意到那是自己的儿子之后,吓出一身冷汗,然后立马开始攻击其他敌人以防止儿子受伤。最后,儿子晕了过去,最后一脸苍白的小女孩也救了下来。

    \subsection*{独具一格的二人}
        女主与2067年出生在一个普通家庭。虽然生活已经愈发艰难,但是家庭内部和睦的氛围依然让她度过着不错的童年。但是很快,一家人就被经济危机所波及,一家人只能变卖掉所有的资产以继续活下去。父母并不忍心让女儿的欢乐童年生活背负上压力和阴影,对她隐瞒了所有的事情。在\textbf{2078年8月12日}, 父母借着带女儿取公园玩耍的机会去银行进行财产清算 /* 吐槽:这大概是金融知识唯一起作用的地方(笑)*/。就在这时,银行被不明武装分子袭击,一家人都被压为人质。本身他们估计这是一次对于银行的抢劫,所以还抱有稍微的乐观。然而逐渐想起的枪声逐渐摧毁着每一个人质的意志,甚至身边有妄图反抗的人被当场枪杀。最终,一家人被带到了已经流满鲜血的银行大门,空气中已经沾上了硝烟的味道。随后,她眼看着父亲和母亲接连倒下,还有旁边陌生的士兵正在叫嚣着。在此刻,女主内心中被恐惧和求救所占领。而在护盾发生器强烈能力场的刺激下,她开始无意识的从意识层面向外进行记忆修改以获取帮助。但此时这种能力刚刚在护盾发生器的刺激下产生,非常的弱小,所以几乎不能对任何正常人产生任何的影响,但却影响到了年幼的男主。

        男主正坐在装甲车上,但是对于枪声的好奇和对于父亲的依赖使得他跑下了车,然后向父亲的方向走去。在此时,他第二次见证到了死亡:第一次正式在很小的时候,见到了母亲的死。虽然记忆已经非常模糊,但是这仍然让他感受到了极强的不适感。长期由于自闭积压的情绪和恐惧完全动摇了他的内心,在收到女主的影响之后形成了一段虚假的记忆印象:如果你能救下眼前的人,你就能改变事实。男主在一瞬之间接受了这样的指令,然后其过应的身体天赋在强烈的感情下被触发,在冲向枪手的过程中,夺得一把匕首并完成进攻。随后由于过度的精神和体力消耗而晕了过去。

        在这次事件之后,父亲决定    收留了这个剩下的小女孩。这就是男主和女主的相遇了。父亲收养她的原因有两个:1是惊讶的发现儿子竟然能够对别人出手而感到有机会能够帮助儿子走出自闭;2是少许的意识层次波动被母亲曾经研究留下的监测装置起了反应,所以处于保护目的而将她收入自己的保护之中。这次事件也让男主父亲和他的儿子被入侵者所标记。

        \noindent /*

            一些想法:

            父亲保留着母亲和研究所的当时的研究资料,并将一部分实用化。

        \noindent */

    \subsection*{还没有写好 ...}
        这之后男主和女主还会有发生一些故事,最后过度到组织捕获到女主作为0125号实验体。


    \subsection*{0125号实验体的实验}

    在与同批次的实验体进行实验的时候,在设法单独改变0125号实验体的记忆之时,记忆单元虽然被实际的扭曲了,但是这种扭曲没有直接改变记忆单元,而是作为一种特殊的扭曲存在在实验体的意识世界中,同时其周围所有的实验体却意外的收到了额外的不可知的影响。自以后,0125号是实验体作为重点观察对象而被持续的实验。研究组织认为她可能作为一个相当好的广域控制的媒介进行使用,并随后对她进行了细致的调查,注意到了男主一家的特殊之处。

    //下一段的内容将作为游戏整体的开头cg

    在男主发现妻子已经失踪了后,与父亲进行了商量。经过了几天的消息,终于确定了女主的位置。由于对于神秘组织的了解,他们决定直接进行暴力的营救活动:父亲以及其下的私人队伍将会从正面进攻,然后派遣男主潜入组织的研究基地。得益于多年的训练和注意吸引,男主顺利的进入了囚禁女主的隔间。然而,令他吃惊的是,这并不是一个单纯的隔间,一个类似动力中枢一样的地方。这个地方还没有完全的运转起来,四周的设备和电线依然杂乱的摆放着,而他的妻子垂着头坐在中央,头戴着怪异的装置。广播中他突然传出了一个合成的人声:“欢迎来到天堂。” 随后,周围的世界中开始闪耀着电光,使的女主突然抬起了头来。而这一瞬间映入男主眼睛的,是一张憔悴的,盖满了泪痕的脸。她的嘴似乎在抽动着,但是男主很难辨别出她想要传到什么。他决定冲过去,割掉那个设备上的电线,然而冲到半途之中,突然感觉到了脑袋的剧痛,眼前一片漆黑。而在分秒之间,当他回过神来之后,血液的温热灼烧着他的脸庞:妻子死在了他的面前,脸上带着最后的一丝微笑;手上的蓝色的荧光刀沾染上了血色,也逐步染红着整个炫白的地面。在这一瞬间,他想要大吼出来,却只能失声的无能为力的跪倒在了血泊中。后面的大门响起了急促的脚步声和呼喊声。他茫然的回头,看到一队卫兵在一名表情凝重的白衣研究者的带领下整齐的跑向这里。他绝望的将武器扔在了一旁,直直的看着满地深红色的鲜血。随后,眼前的一切变得模糊,周围素白的大厅变成了高坡。旁边的父亲一脸复杂和叹息的看着男主,然后进行了撤退。
    
    \medskip

    关于最后这里到底发生了什么的设定:在得知了男主一家的背景之后,研究头子想出了一个疯狂的计划:男主一家一定会进攻设施设法营救女主,所以在他们进入女主的实验舱时,通过女主特殊的能力改变他们的记忆来达到控制他们的目的。此项能够作为一个绝好的实验机会,可以实验在非实验条件下记忆改变的情况和数据,同时位于实验中心也便于控制。如果控制效果良好,就相当于他们能够控制一个高水平的战士(由多年前的事件推知),那这项技术实际上就可以很快投入实用了。所以,这实际上是一个已经设置好的圈套。

    唯一的不可预计变量:女主。在多天的实验中,多个异常的记忆单元改变已经非常严重的影响到了意识世界的稳定。然而,强大的意识依旧识别出了正确的记忆而始终保持着正常的意识,只是精神的摧残让她非常的虚弱。但另一个方面,这种特殊的能力让她能够扫描研究人员的记忆同时知晓了周围的状况。在知道为男主设立的陷阱后,她决定以最残酷的方式防止男主的记忆受到改变。在最后一次研究人员试图想要以她作为载体向周围广播篡改的记忆之时,她通过意识修改了这样的记忆,向男主传递了这样的信息:眼前的这个女人就是当年杀死母亲的凶手。而被拨到心弦的男主在瞬间的愤怒中则会砍向她自己,这样就能够从记忆影响中摆脱出来,拥有了能够离开的机会。在男主冲过来之前,她在口中不断的念着:"是我,杀死了你的妈妈",以加强意识层次的力量,与原来一样影响了男主对于母亲被刺杀的判断。同时,她将她所能收集到的尽可能多的记忆碎片,也拼劲全力向外传输。最终,男主的脑海中接收到了这样的记忆碎片:被篡改的记忆,女主再次篡改的记忆,女主大部分真实的记忆

    \newpage 
    \section*{游戏流程的具体设计}

    //主要就看第一个,后面的会随着剧情不断的调整和添加,还处在设计阶段,第一个应该是作为设定不会有太多的变化

    无神的男主被父亲掺回了家,男主由于极端的疲乏而进入了熟睡。而此时意识世界的结果将决定男主的未来。玩家将扮演营救女主这件事情所形成的记忆单元。由于受到了记忆修改的影响,玩家扮演的记忆单元本身就是破碎而混乱的。意识对于记忆单元的想法,及情绪将会决定记忆单元在意识世界的力量。设定上,记忆单元只拥有产生记忆单元的事件,它甚至是扭曲的,虚假的,但都是主角意识所认知的。

    这里会写上部分统一的游戏性设定,以供下面的关卡设定。

    \subsection*{守护三界的堕天使}
            扮演的记忆单元最开始的记忆: 男主进入研究所营救妻子。由于男主脑中的记忆已经被篡改,在此段记忆单元中男主认为坐在类似大厅一样的囚禁隔间的人正是杀害母亲的凶手。此时,整个研究院变成了类似城堡的样子。在现实中,由于这是布下的圈套,男主的潜入过程没有任何的干扰,所以整个地图就是直直的类似宫殿门口到王座的直线。最终的boss为以妻子为原本的堕天使,触发场景动画:眼前出现母亲的样子,然后被boss击杀。通过对于boss的确认,男主辨认出这正是他的妻子。

            妻子的剧情物品掉落:

                \quad 地牢的钥匙:通往妻子关于研究设施的记忆碎片的关键道具。

            之后,妻子将会从变成带有链锁的天使状态,变为可对话的npc。 在设定中,这就是女主在男主记忆篡改中插入的由自己的意识修正过的记忆碎片。堕天使就是导致男主在现实世界中错杀了女主的错误的记忆,而在击败了堕天使之后也将解放真正的其余的女主记忆碎片。她将作为玩家在游戏世界中最开始的引导者。

            “你所找到的回忆,就是全部的事实。”
            
            *待补充

    \subsection*{夏日悲喜交加的挽歌}

            //我觉得我的恋爱情节写的太尬了。。 


            “挽歌”的意义:对于当今死去女主的回忆挽歌。

            * 新武器引入

            发生于男主高中时的记忆。由于自幼丧母,性格内向,所以并没有什么朋友。但是由于从小在父亲的指导下锻炼,体格强壮而没有被欺负。男主一直不知道父亲的真实工作,只知道是一家安保公司的负责人,对于母亲的死去也停留在死于意外的医疗事故。

            记忆场景开始于父亲教授男主一些特殊武器的用法。在游戏性的设定上大部分的武器在此处开放使用。此时,同年的女主的影像出现并邀请男主去逛街(你不要问我为什么是逛街,找个理由出门就好了)。其余的场景为一张很小的街区地图,但特点是大部分地方被黑雾笼罩无法看清,而仅有男主的小范围和女主的大范围以及家具有正常的光亮。同时,对于男主来说,地图的其他地方出现怪物。在剧情设定上,这里是小女主为了能够持续的让小男主打起精神而邀约他出门。所有的怪物源自于男主的自闭恐惧,同时由于组织记忆篡改的原因而变得更加的扭曲

            剧情上他们顺利的到达了xx商店的门口,此时却遭到了未知武装人员的袭击。在现实当中,这里是由于男主父亲的仇敌对于男主筹划的绑架。在现实生活中,男主艰难的以防御女主目标进行着还击,而最终快力竭的时候,由于艰难的困境激发了女主于记忆的特殊能力,对周围的目标进行了记忆修改,而使得周围所有敌人自杀。记忆的小范围篡改也修改了男主的记忆,使得在游戏场景中的记忆产生了偏差。本关此处的设计就是为了让他们知道虚假记忆单元的存在。在游戏中,玩家将面对一波一波的进攻和最终的队长boss,进入boss战,但是boss战将会在2min内之后将所有的敌人变成了一击致命。从剧情上,将会有提示此处为假记忆的信息(心里活动,暂定)。而在战斗成功后,女主将倒在地上,掉落<破碎的记忆晶体>。
            
            同时这里也会触发额外的逃跑选项:直接从战斗中逃离,将会触发组织记忆篡改的效果,而将在之后的记忆关卡中触发新的后续影响。选择逃跑的玩家将在某一个结局触发之前将不能够获得相应的本章的掉落。但是同时,他们会从被锁链的天使处获得<完整的记忆晶体>。

            完整记忆晶体 相对于 破碎记忆晶体来说。不同的晶体将会解锁不同的记忆单元。

    \subsection*{常驻病房的命运之女}
            
            破碎的记忆晶体,带给大厅中的女主将开启新的此段记忆碎片。在现实生活中,女主在释放了记忆篡改之后便晕倒了,留下了混乱的男主和一地尸体。随后父亲赶到,妥善处理了后事。女主被安排住进了医院,接受相关的关于脑部的治疗。在这里,女主偶然接触到了当年被雇佣做掉男主母亲的助理护士,在强烈的相关性下得知了男主母亲被害的事实。由于这件事,她在修养的期间联系了男主的父亲进行了确认。但是,两人都决定不愿意向男主告知真相,希望他能够远离这一切的阴谋。

    \subsection*{} 


    \subsection*{极乐空间的黑暗领域}
            
            完整的记忆晶体带来的虚假记忆,是完全虚假的。
            ///这段由于比较难以触发所以我暂时不写,目的是为了引导走向自我毁灭和绝望。后续应该还有1-2个关卡,不过由于做不完所以可能会放弃。
            
    \newpage 

        \section*{额外的考虑}
            开局动画和结局动画,男主小时候和长大后两次冲向女主的镜头可以进行比较。
        //条件推理

        //格局太小  

        //细化记忆单元

        //不够arpg,恐怖游戏

        //


        游戏的核心:记忆为基本概念,

        剧情突出的过程:

        boss对应某个形象,跟情绪相连。

        //恐怖原则:表现概念(形象),氛围使用恐惧突出

        //刺激点
        
        //信息给出,要保证信息

        //背景可以更加的艺术

        // 节奏点:混沌系列:脑内世界可以去影响现实世界。 注意和本设定的区别:所有的改变只关于认知而现实已经定好了。tiharu提出这一点需要在战斗和游戏构成中有足够的体现以强调剧情。原话:就是说你的战斗要让玩家有很深的印象,会让玩家觉得 卧槽 这个设定好帅 我要认真过剧情搞明白这是怎么一回事。

        //以为自己刷了一身神装,然后boss面前没揭露了真实,你捡的都是破铜烂铁,你喝的药都是污水

        //或者说你在战斗的时候 boss给你上了一个debuff 你非常清醒地认识到这是他在操纵你的记忆 然后debuff消失你顺利消灭了boss,但是boss其实是为你修正了记忆 你其实是真的很弱 但你拒绝相信自己很弱 继续活在梦里 然后走进be

        //女主的能力设置需要更加的细节以保证自洽性,比如能力发起的条件。
\end{document}
