\documentclass[12pt, a4paper]{article}   
\usepackage{indentfirst, amsmath, fontspec, listings, xcolor, amssymb}

\setmainfont{Microsoft YaHei}
    
\newfontfamily\consolas{Consolas}
\lstset{numberstyle = \small\consolas, basicstyle=\small\consolas}
    
\XeTeXlinebreaklocale "zh"
\XeTeXlinebreakskip = 0pt plus 1pt

\begin{document}
    \section*{基础信息}
    游戏名称:破碎之忆

    游戏类型:ARPG

    设计理念:
    
        \quad 记忆收集与选择。
    \newpage 

    \section*{游戏背景设定和特殊设定解释}
        
        \textbf{意识世界}:每一个人都有期望的世界,而这样的幻景将被抽象影响着整个意识。在当前的认知和记忆的作用下,一个基础的所谓意识世界将会被搭建。而在人自我愿景的影响下,这个世界也戴上了更多的主观色彩。意识世界具有相似的结构,大部分都是由记忆填充的:记忆单元作为每一个记忆片段的主体,而这个主体具体是什么则各不相同:那段记忆中的自己,和自己交流的父母,常伴身边的宠物;所有被意识主体感知为有意识的东西都能作为记忆单元而表现出来。而记忆所涉及的其它的内容,则笼统的融合到了意识世界的环境当中。而真正被称为意识的逻辑过程则支配了每一个记忆单元在这个世界的行为和抉择,最终这些抉择的结果将改变记忆的形态,改变这个世界。严格的说,意识主体和意识世界是不断交互和影响的,其反应为认知和意识的不断交流以产生实际生活中的实际行为。 
        
       而在很长的一段时间内,这样的世界都是不可知的。对于意识主体以外的人而言,意识主体的所思所想自然不得而知;而对于意识主体本身,这样的一个世界只是一大段大脑活动的附属品。而现在,研究人员协力终于得知了那个世界的存在,而且能够从意识主体之外
        观察和影响到意识世界。 

        \medskip
        \textbf{记忆单元}:作为意识的重要组成部分,记忆是以单元的形式存在的。所有的记忆共享着意识主体的意识。然而,与其被称为共享,支配一词更加合适。意识主体对待记忆片段的看法决定了记忆在这个意识世界的体现,同时也决定了不同记忆对于意识主体的影响。举例而言,对于同一段时间的两种不同的记忆将会最终留存下一个作为真正的记忆,而这样的决定过程在意识世界中被表现为记忆单元的死亡和周围环境的崩塌,消失。记忆单元之间互动的结果实际上就是意识主体最后立场的表现,这将直接影响意识主体在现实当中的行动。 

        记忆的真假跟记忆单元本身的内容无关,而是意识主体的意识活动的结果。这种真假是完全独立于事实上的真假,而是主观上的,会对于意识主体的行为产生直接影响。即使是相同的记忆单元,不同的意识主体对于此段记忆的看法会截然不同,这表示记忆在不同的意识上的表现也将不尽相同

        \medskip
        \textbf{意识的扩散和影响}:意识本身包好了很多的东西:性格,理想,愿望等。如上所述,意识最终决定了记忆单元的在一个人意识世界中的体现,而这种体现通过共享意识进行与其他记忆进行反应最终决定意识主体的行为。而对于意志能力强的人来说,在合适环境的诱导之下,他们甚至能够对于他人的记忆单元进行操纵和修改,以达到控制他人行为的目的。从外界通过科技对人的记忆进行篡改也基于相似的原理,而在0125号实验体将是第一个体现出如此能力的个体。

        \medskip
        \textbf{记忆篡改}:记忆篡改的目的是为了改变一个人在现实生活中的抉择。有两种方式实现记忆的篡改:在意识世界中制造假的记忆碎片并被意识主体所接受;在已有的记忆碎片中进行有意的额外的修改。无论使用那种方法,对于想要篡改记忆以达到意识控制的势力来说,其实都是诱导意识主体去相信虚假的事实来达到控制的目的,而从原理上和洗脑非常相似:首先通过对于实验题的记忆进行影响而试图孤立实验体在社会上的地位认知,已让意识感受到无助和绝望;在如下的条件下,篡改的记忆更容易被接受,而甚至最后完全丧失意识的主动性。由于它能够从记忆层次影响人的判断,同时由于向实验体输出绝望和孤独感拥有更加通用的方式:通过篡改关系相近的人的相关记忆,使得他们对于周围的社会关系产生绝望,达到这一步基本已经成功了,在大量的可观察个体中还未曾发现例外,直到0125号实验体,产生了完全不同的效应。
        
        \textbf{0125号实验体}:0125号的特殊性在于:1.任何形式的记忆篡改都没有办法对其真实的记忆和意识认知产生任何的影响,但是会留下所有的篡改痕迹,将会以额外的记忆碎片存在于0125的意识世界中。2.能够在小范围内进行意识散播;在本体意识急剧波动时产生,能向外急剧的输出主观意识修正的记忆修改。但是从本质上,这个和记忆篡改是不同的:记忆篡改在最初的阶段通过找到被修改者与世界的感情联系然后恶意扭曲所有相关的记忆碎片以产生负面情绪和意识,而此处的记忆修改是直接强制的无视所有的被输出者的所有社会经历,以0125号个人意志所希望的行为而直接强行使得意志在瞬间对此行为进行执行。具体来说,为了使得意志执行,将会突然大量的在被输出者的头脑者创建拥有极强倾向性的记忆去迷惑意识,而最终取代原有的记忆。两者的主要区别就在于这种修改是否源生于特定的倾向。暂定为唯一的特殊个体,后续可继续扩展。

        \textbf{男主和女主}:

        //本段文档旨在确定在设计之处对于男女关系的设想和对于两人的定位,可能会稍有不适(趴
        //由于设计者没有女朋友而且曾经也没有过,所以设计者不知道怎么写这份关系

        所有的角色行为都建立在无穷的保护欲之上:父亲由于社会失去了母亲所以拼命的保护独生的儿子,男主由于年幼丧母以及孤立的社会环境而失败。

    \newpage        
    \section*{角色设计}

        \textbf{父母}:
            父亲早年为精英特种小队的队长,母亲则是政府脑科学研究所的研究员。由于在记忆领域的不断进展,整个小队被要求秘密保护整个脑科学研究所。然而,由于神秘组织为了能够得到更多的科学家资源,逼迫所有的研究员转移到组织名下的研究机构而架空掉整个研究所。母亲同时因为公开反对和身孕而远离了研究所,但仍然在医院惨遭毒手。组织并没有对新生的男主出手,是因为他们自信在医院的医疗事故能够摆平一切,而区区一个单亲家庭是不会对组织产生任何的威胁。从母亲死去的那天开始,父亲就选择了退役,并开办了一家安保公司作为独立的力量以保护自己的仅存的家人,同时对于自己的儿子守口如瓶,决心让儿子脱这阴谋的怪圈。

        \textbf{男主}:

            “父亲告诉我:‘你生下来了,就是唯一的火种。’” 
            
            在出生之后几天,母亲离奇死去,死因为:伤口感染恶化至死亡。自幼接受父亲的知道,一直保持锻炼,学习使用各式武器。由于天生缺乏母爱,不善言谈,但天性认真,勤奋。因为这样的性格,在学校里面虽然因为自身的实力而让人畏惧,但是也不乏嘲笑和恶意。
            


            现25岁。

            强壮,稚气,复杂融合,迷茫。 正直。

            由于丧母一直处于精神的不稳定和高度的抑郁当中,导致了强烈的对外恐惧和自闭。

        \medskip
        \textbf{女主}:  


            男主的妻子,自幼认识男主,强大的意志力:勇敢,乐观。守护。 
            
    \newpage 

    \section*{故事梗概}

    \subsection*{大的世界观设定}    

    //我显著的认识到了我是一个政治白痴,所以这段可能槽点过多
    
    总体的科技水平随着基础物理的跳跃发展,结合信息时代的爆炸迎来了地球文明的黄金时期。但是同时, 贫富差距却不断拉大,经济寡头化严重。受限于在蓬勃的科技发展中被忽略的落后经济体系(不要问我为什么落后),大量的财富被寡头所收集,大量的普通人甚至无法享受科技发展带来的福利。逐渐走高的犯罪率和日益升高的生活成本,使得平民的生活没有办法进一步提高,长期的原地踏步。\textbf{21世纪70年代中期},世界各个国家开始相继爆发经济危机,使得正常的社会秩序被进一步打乱,社会矛盾更加突出。从此刻,全世界开始动荡:多数国家开始发生政权的不正常更迭,大的利益集团开始收割压榨普通民众的利益,战争相继爆发,地球文明进入了有一个黑暗时期。

    \subsection*{背后的组织}

    \textbf{20世纪中叶},由于同时代的科技进步,以脑科学为代表的生物科学获得了难得的发展契机。随着研究的深入,科学家逐渐发现意识世界的存在以及意识与记忆之间的关系。虽然经过数年的研究,他们依然没有解开意识形成的底层原因,但是自从记忆单元这个概念被提出后,控制一个人的记忆成为了可能。由于社会持续动荡,神秘组织(我真的想不出来名字)开始策划利用根植的影响力妄图掌握新兴技术而达到独立的霸权地位。在脑科学上,他们瞄准了当时位于世界顶尖的某研究所,通过威逼利诱的方式想尽办法将科学家收入麾下,而对于少部分不听话的学者则抹杀致死,没有任何例外。\textbf{2066年},研究所正式被架空,而组织内部也紧锣密鼓的进行着相关的研究。

    随着研究的不断深入,研究所花费了二十年的时间,最终开始获知到了意识世界和记忆单元的存在。虽然相关的研究并没有成熟,组织迅速的将研究方向调整为记忆修改与意识引导,以期望最终实现对于大众的意识控制。通过普通科学实验的名号和丰厚的报酬吸引了生活艰难的普通群众充当实验材料。而实际上,大部分初期的实验体最终在实验结束都伴随着精神恍惚,反应速度下降,而换来的就是组织对其意识的主导权。为了追求更好的实际效果,组织需要越来越大量的数据用以研究,绑架低层次阶级和无背景平民成为了更加简单粗暴的选择,同时利用自己在暗处的影响力也能够轻松的消除绑架所带来的负面影响。\textbf{2090年},他们将目标选在了在记录中没有父母的0125号实验体。

    \subsection*{暗杀计划}
        
    男主的一家曾经是一个小康的中产阶级。父亲成年参军,效力于特种部队数年,最为小队的队长,而后带领着这支特殊力量保护国家的科研力量。母亲则是脑科学研究的专家,在收到父亲保护之时相识相爱,结为夫妻。\textbf{2065年},母亲怀孕,同时所属的研究所正处于危机当中。她严厉的拒绝了组织向她开出的条件,并以身孕为由处于休假状态。\textbf{2066年},男主出生。然而就在男主出生后的第二天,母亲在医院里面被下了神秘组织下了最后通牒:加入或是死亡。母亲自然不愿意屈服,并于父亲商量这样的计划:凭借父亲特殊的身份以及社会关系,同时在院方熟人的帮助下,母亲将被秘密转移到其他的子医院躲避组织的追杀,同时由院方和父亲带领的小队近身保护母亲的安全,在此期间通过释放不明确的信号来拖时间,在母亲痊愈后立马离开这个国家。然而,组织已经没有耐心等下去了,他们自信获得了足够的科学家资源,通过安插在医院的护士暗杀了尚在恢复的母亲,最终母亲的死亡被认定为外伤感染,不了了之。

    备受刺激的父亲愤然离开了军队,与少部分知晓真相的战友开创了私人保全公司。他们对于政府和现状感到无比的失望,无论是因为作为丈夫看到妻子死在了自己面前,还是因为保护许久的科研人才就此殒命。这一沉痛的教训也让父亲变得十分的谨慎:他在表面上服务于各大利益集团,作为佣兵收取高额的费用同时获得了广泛的社会门路;在暗地里收集有关组织的资料,同时通过各种方法,掌握前沿的军事科技以备有朝一日为妻子报仇。从那时起,父亲只有一个念头:尽可能的保护好唯一的下一代,同时为妻子报仇。

    \subsection*{落魄的女主}

    \subsection*{0125号实验体的实验}

    0125号实验体(女主):在与同批次的实验体进行实验的时候,在设法单独改变0125号实验体的记忆之时,记忆单元虽然被实际的扭曲了,但是这种扭曲没有直接改变记忆单元,而是作为一种特殊的扭曲存在在实验体的意识世界中,同时其周围所有的实验体却意外的收到了额外的不可知的影响。自以后,0125号是实验体作为重点观察对象而被持续的实验。

    社会上已经走漏了部分实验的风声,由于每一个实验结束的个体都被相应删除了对应的试验记忆,产生了一定程度的记忆混乱。在男主发现妻子已经失踪了几天之后,男主随即和其父亲进行了商量。听闻消息后,父亲带着男主前往了男主从没有了解过的秘密武器库并被叮嘱穿戴上的一套特殊的服装。之后,父亲直接标记出了女主被囚禁的位置,然后派遣男主直接潜入组织的研究基地,他自己则在研究基地的高处提供支援。得益于多年的训练,男主顺利的进入了囚禁女主的隔间。然而,令他吃惊的是,这并不是一个单纯的隔间,一个类似动力中枢一样的地方。这个地方还没有完全的运转起来,四周的设备和电线依然杂乱的摆放着,而他的妻子垂着头坐在中央,头戴着怪异的装置。广播中他突然传出了一个合成的人声:“欢迎来到天堂。” 随后,周围的世界中开始闪耀着电光,使的女主突然抬起了头来。而这一瞬间映入男主眼睛的,是一张憔悴的,盖满了泪痕的脸。她的嘴似乎在抽动着,但是男主很难辨别出她想要传到什么。他决定冲过去,割掉那个设备上的电线,然而冲到半途之中,突然感觉到了脑袋的剧痛,眼前一片漆黑。而在分秒之间,当他回过神来之后,血液的温热灼烧着他的脸庞:妻子死在了他的面前,脸上带着最后的一丝微笑;手上的蓝色的荧光刀沾染上了血色,也逐步染红着整个炫白的地面。在这一瞬间,他想要大吼出来,却只能失声的无能为力的跪倒在了血泊中。后面的大门响起了急促的脚步声和呼喊声。他茫然的回头,看到一队卫兵在一名表情凝重的白衣研究者的带领下整齐的跑向这里。他绝望的将武器扔在了一旁,直直的看着满地深红色的鲜血。随后,眼前的一切变得模糊,周围素白的大厅变成了高坡。旁边的父亲一脸复杂和叹息的看着男主,然后伸出手来将男主拉了起来。
    
    关于最后这里到底发生了什么的设定:女主作为特殊的实验对象,不仅有了更多的特殊实验,组织也同时广泛的收集着有关女主的个人关系网以获得这种特殊能力的出处和原因。在得知了男主一家的背景之后,研究头子想出了一个疯狂的计划:男主一家一定会进攻设施设法营救女主,所以在他们进入女主的实验舱时,通过女主特殊的能力改变他们的记忆来达到控制他们的目的。此项能够作为一个绝好的实验机会,可以实验在非实验条件下记忆改变的情况和数据,同时位于实验中心也便于控制。所以,这实际上是一个已经设置好的圈套。

    唯一的不可预计变量:女主。在多天的实验中,多个异常的记忆单元改变已经非常严重的影响到了意识世界的稳定。然而,强大的意识依旧识别出了正确的记忆而始终保持着正常的意识,只是精神的摧残让她非常的虚弱。但另一个方面,这种特殊的能力让她能够扫描研究人员的记忆同时知晓了周围的状况。在知道为男主设立的陷阱后,她决定以最残酷的方式防止男主的记忆受到改变。在最后一次研究人员试图想要以她作为载体向周围广播篡改的记忆之时,她通过意识修改了这样的记忆,向男主传递了这样的信息:眼前的这个女人就是当年杀死母亲的凶手。而被拨到心弦的男主在瞬间的愤怒中则会砍向她自己,这样就能够从记忆影响中摆脱出来,拥有了能够离开的机会。在男主冲过来之前,她在口中不断的念着:"是我,杀死了你的妈妈",以加强意识层次的力量。同时,她将她所能收集到的尽可能多的记忆碎片,也拼劲全力向外传输。最终,男主的脑海中接收到了这样的记忆碎片:被篡改的记忆,女主再次篡改的记忆,女主大部分真实的记忆

    \newpage 
    \section*{游戏流程的具体设计}

    //主要就看第一个,后面的会随着剧情不断的调整,还处在设计阶段,第一个应该是作为设定不会有太多的变化

    无神的男主被父亲掺回了家,男主由于极端的疲乏而进入了熟睡。而此时意识世界的结果将决定男主的未来。玩家将扮演营救女主这件事情所形成的记忆单元。由于受到了记忆修改的影响,玩家扮演的记忆单元本身就是破碎而混乱的。意识对于记忆单元的想法,及情绪将会决定记忆单元在意识世界的力量。设定上,记忆单元只拥有产生记忆单元的事件,它甚至是扭曲的,虚假的,但都是主角意识所认知的。

    这里会写上部分统一的游戏性设定,以供下面的关卡设定。

    \subsection*{守护三界的堕天使}
            扮演的记忆单元最开始的记忆: 男主进入研究所营救妻子。由于男主脑中的记忆已经被篡改,在此段记忆单元中男主认为坐在类似大厅一样的囚禁隔间的人正是杀害母亲的凶手。此时,整个研究院变成了类似城堡的样子。在现实中,由于这是布下的圈套,男主的潜入过程没有任何的干扰,所以整个地图就是直直的类似宫殿门口到王座的直线。最终的boss为以妻子为原本的堕天使,触发场景动画:眼前出现母亲的样子,然后被boss击杀。通过对于boss的确认,男主辨认出这正是他的妻子。

            妻子的剧情物品掉落:

                \quad 地牢的钥匙:通往妻子关于研究设施的记忆碎片的关键道具。

            之后,妻子将会从变成带有链锁的天使状态,变为可对话的npc。 在设定中,这就是女主在男主记忆篡改中插入的由自己的意识修正过的记忆碎片。堕天使就是导致男主在现实世界中错杀了女主的错误的记忆,而在击败了堕天使之后也将解放真正的其余的女主记忆碎片。她将作为玩家在游戏世界中最开始的引导者。

            “你所找到的回忆,就是全部的事实。”
            
            *待补充

    \subsection*{夏日悲喜交加的挽歌}

            //我觉得我的恋爱情节写的太尬了。。 


            “挽歌”的意义:对于当今死去女主的回忆挽歌。

            * 新武器引入

            发生于男主高中时的记忆。由于自幼丧母,性格内向,所以并没有什么朋友。但是由于从小在父亲的指导下锻炼,体格强壮而没有被欺负。男主一直不知道父亲的真实工作,只知道是一家安保公司的负责人,对于母亲的死去也停留在死于意外的医疗事故。

            记忆场景开始于父亲教授男主一些特殊武器的用法。在游戏性的设定上大部分的武器在此处开放使用。此时,同年的女主的影像出现并邀请男主去逛街(你不要问我为什么是逛街,找个理由出门就好了)。其余的场景为一张很小的街区地图,但特点是大部分地方被黑雾笼罩无法看清,而仅有男主的小范围和女主的大范围以及家具有正常的光亮。同时,对于男主来说,地图的其他地方出现怪物。在剧情设定上,这里是小女主为了能够持续的让小男主打起精神而邀约他出门。所有的怪物源自于男主的自闭恐惧,同时由于组织记忆篡改的原因而变得更加的扭曲

            剧情上他们顺利的到达了xx商店的门口,此时却遭到了未知武装人员的袭击。在现实当中,这里是由于男主父亲的仇敌对于男主筹划的绑架。在现实生活中,男主艰难的以防御女主目标进行着还击,而最终快力竭的时候,由于艰难的困境激发了女主于记忆的特殊能力,对周围的目标进行了记忆修改,而使得周围所有敌人自杀。记忆的小范围篡改也修改了男主的记忆,使得在游戏场景中的记忆产生了偏差。本关此处的设计就是为了让他们知道虚假记忆单元的存在。在游戏中,玩家将面对一波一波的进攻和最终的队长boss,进入boss战,但是boss战将会在2min内之后将所有的敌人变成了一击致命。从剧情上,将会有提示此处为假记忆的信息(心里活动,暂定)。而在战斗成功后,女主将倒在地上,掉落<破碎的记忆晶体>。
            
            同时这里也会触发额外的逃跑选项:直接从战斗中逃离,将会触发组织记忆篡改的效果,而将在之后的记忆关卡中触发新的后续影响。选择逃跑的玩家将在某一个结局触发之前将不能够获得相应的本章的掉落。但是同时,他们会从被锁链的天使处获得<完整的记忆晶体>。

            完整记忆晶体 相对于 破碎记忆晶体来说。不同的晶体将会解锁不同的记忆单元。

    \subsection*{常驻病房的命运之女}
            
            破碎的记忆晶体,带给大厅中的女主将开启新的此段记忆碎片。在现实生活中,女主在释放了记忆篡改之后便晕倒了,留下了混乱的男主和一地尸体。随后父亲赶到,妥善处理了后事。女主被安排住进了医院,接受相关的关于脑部的治疗。在这里,女主偶然接触到了当年被雇佣做掉男主母亲的助理护士,在强烈的相关性下得知了男主母亲被害的事实。由于这件事,她在修养的期间联系了男主的父亲进行了确认。但是,两人都决定不愿意向男主告知真相,希望他能够远离这一切的阴谋。

    \subsection*{} 


    \subsection*{极乐空间的黑暗领域}
            
            完整的记忆晶体带来的虚假记忆,是完全虚假的。
            ///这段由于比较难以触发所以我暂时不写,目的是为了引导走向自我毁灭和绝望。后续应该还有1-2个关卡,不过由于做不完所以可能会放弃。
            

\end{document}
