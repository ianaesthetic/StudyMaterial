\documentclass[12pt, a4paper]{article}   
\usepackage{indentfirst, amsmath, fontspec, listings, xcolor, amssymb}

\setmainfont{Microsoft YaHei}
    
\newfontfamily\consolas{Consolas}
\lstset{numberstyle = \small\consolas, basicstyle=\small\consolas}
    
\XeTeXlinebreaklocale "zh"
\XeTeXlinebreakskip = 0pt plus 1pt

\begin{document}
    \section*{基础信息}
    游戏名称:破碎之忆

    游戏类型:ARPG

    设计理念:
    
        \quad 记忆收集与选择。
    \newpage 

    \section*{游戏背景设定}
        
        \textbf{意识世界}:每一个人都有期望的世界,而这样的幻景将被抽象影响着整个意识。在当前的认知和记忆的作用下,一个基础的所谓意识世界将会被搭建。而在人自我愿景的影响下,这个世界也戴上了更多的主观色彩。意识世界具有相似的结构,大部分都是由记忆填充的:记忆单元作为每一个记忆片段的主体,而这个主体具体是什么则各不相同:那段记忆中的自己,和自己交流的父母,常伴身边的宠物;所有被意识主体感知为有意识的东西都能作为记忆单元而表现出来。而记忆所涉及的其它的内容,则笼统的融合到了意识世界的环境当中。而真正被称为意识的逻辑过程则支配了每一个记忆单元在这个世界的行为和抉择,最终这些抉择的结果将改变记忆的形态,改变这个世界。严格的说,意识主体和意识世界是不断交互和影响的,其反应为认知和意识的不断交流以产生实际生活中的实际行为。 
        
       而在很长的一段时间内,这样的世界都是不可知的。对于意识主体以外的人而言,意识主体的所思所想自然不得而知;而对于意识主体本身,这样的一个世界只是一大段大脑活动的附属品。而现在,研究人员协力终于得知了那个世界的存在,而且能够从意识主体之外
        观察和影响到意识世界。 

        \medskip
        \textbf{记忆单元}:作为意识的重要组成部分,记忆是以单元的形式存在的。所有的记忆共享着意识主体的意识。然而,与其被称为共享,支配一词更加合适。意识主体对待记忆片段的看法决定了记忆在这个意识世界的体现,同时也决定了不同记忆对于意识主体的影响。举例而言,对于同一段时间的两种不同的记忆将会最终留存下一个作为真正的记忆,而这样的决定过程在意识世界中被表现为记忆单元的死亡和周围环境的崩塌,消失。记忆单元之间互动的结果实际上就是意识主体最后立场的表现,这将直接影响意识主体在现实当中的行动。 

        记忆的真假跟记忆单元本身的内容无关,而是意识主体的意识活动的结果。这种真假是完全独立于事实上的真假,而是主观上的,会对于意识主体的行为产生直接影响。即使是相同的记忆单元,不同的意识主体对于此段记忆的看法会截然不同,这表示记忆在不同的意识上的表现也将不尽相同

        \medskip
        \textbf{意识的扩散和影响}:意识本身包好了很多的东西:性格,理想,愿望等。如上所述,意识最终决定了记忆单元的在一个人意识世界中的体现,而这种体现通过共享意识进行与其他记忆进行反应最终决定意识主体的行为。而对于意志能力强的人来说,在合适环境的诱导之下,他们甚至能够对于他人的记忆单元进行操纵和修改,以达到控制他人行为的目的。从外界通过科技对人的记忆进行篡改也基于相似的原理,而在0125号实验体将是第一个体现出如此能力的个体。

        \medskip
        \textbf{记忆篡改}:记忆篡改的目的是为了改变一个人在现实生活中的抉择。有两种方式实现记忆的篡改:在意识世界中制造假的记忆碎片并被意识主体所接受;在已有的记忆碎片中进行有意的额外的修改。无论使用那种方法,对于想要篡改记忆以达到意识控制的势力来说,他们倾向于运用人性和记忆中的弱点进行突破,以达到他们的目的。
            

    \newpage        
    \section*{角色设计}

        \textbf{父母}:
            父亲早年为精英特种小队的队长,母亲则是政府脑科学研究所的研究员。由于在记忆领域的不断进展,整个小队被要求秘密保护整个脑科学研究所。然而,由于神秘组织为了能够得到更多的科学家资源,逼迫所有的研究员转移到组织名下的研究机构而架空掉整个研究所。母亲同时因为公开反对和身孕而远离了研究所,但仍然在医院惨遭毒手。组织并没有对新生的男主出手,是因为他们自信在医院的医疗事故能够摆平一切,而区区一个单亲家庭是不会对组织产生任何的威胁。从母亲死去的那天开始,父亲就选择了退役,并开办了以加安保公司作为独立的力量以保护自己的仅存的家人。 

        \textbf{男主}:

            “父亲告诉我:‘你生下来了,就是唯一的火种。’” 
            
            在出生之后几天,母亲离奇死去,死因为:伤口感染恶化至死亡。自幼被迫锻炼,学习使用各式武器。由于天生缺乏母爱,不善言谈。天性认真,勤奋。现25岁。

            强壮,稚气,复杂融合,迷茫。

        \medskip
        \textbf{女主}:  
            男主的妻子,青梅竹马。强大的意志力:勇敢,乐观。守护。 
            
    \newpage 
    \section*{故事梗概}
    \subsection*{游戏开始之前}
        22世纪开始,一部分脑科学家开始深入探索脑的奥秘。由于同时代的科技进步,脑科学也渐渐的取得更多的进展。随着研究的深入,科学家逐渐发现意识世界的存在以及意识与记忆之间的关系。虽然经过数年的研究,他们依然没有解开意识形成的底层原因,但是自从记忆单元这个概念被提出后,控制一个人的记忆成为了可能。某个神秘组织的科学家开始将他们的研究方向更加细化的着重于记忆以及记忆和意识之间的联系,常年的研究使他们逐步发现了如何通过电磁效应和影响人的记忆,通过意识和记忆之间的联系对被实验个体进行影响甚至控制。

        随后,他们进行了广泛的社会实验。组织不断以普通科学实验的名号吸引普通群众。随着实验需求的不断增长,组织开始通过绑架的方式来获得新的实验材料,并通过媒体力量和贿赂来消除自己的痕迹。实验的预计内容:最开始是设法通过外部影响改变记忆单元的细节以扭曲实验体的认知,而后通过创造出与现有记忆单元完全相反的记忆单元来替代现有的记忆单元,最终大范围的替代记忆使实验体完全具有另外一个人或者个体的记忆。得益于长期的前期研究,实验进行的非常熟练,以至于组织决定迅速扩大实验体的范围来。这种状况直到他们获得了0125号实验体。 

        0125号实验体(女主):在与同批次的实验体进行实验的时候,在设法单独改变0125号实验体的记忆之时,记忆单元虽然被实际的扭曲了,但是这种扭曲没有直接改变记忆单元,而是作为一种特殊的扭曲存在在实验体的意识世界中,同时其周围所有的实验体却意外的收到了额外的不可知的影响。自以后,0125号是实验体作为重点观察对象而被持续的实验。

        社会上已经走漏了部分实验的风声,由于每一个实验结束的个体都被相应删除了对应的试验记忆,产生了一定程度的记忆混乱。在男主发现妻子已经失踪了几天之后,男主随即和其父亲进行了商量。听闻消息后,父亲带着男主前往了男主从没有了解过的秘密武器库并被叮嘱穿戴上的一套特殊的服装。之后,父亲直接标记出了女主被囚禁的位置,然后派遣男主直接潜入组织的研究基地,他自己则在研究基地的高处提供支援。得益于多年的训练,男主顺利的进入了囚禁女主的隔间。然而,令他吃惊的是,这并不是一个单纯的隔间,一个类似动力中枢一样的地方。这个地方还没有完全的运转起来,四周的设备和电线依然杂乱的摆放着,而他的妻子垂着头坐在中央,头戴着怪异的装置。广播中他突然传出了一个合成的人声:“欢迎来到天堂。” 随后,周围的世界中开始闪耀着电光,使的女主突然抬起了头来。而这一瞬间映入男主眼睛的,是一张憔悴的,盖满了泪痕的脸。她的嘴似乎在抽动着,但是男主很难辨别出她想要传到什么。他决定冲过去,割掉那个设备上的电线,然而冲到半途之中,突然感觉到了脑袋的剧痛,眼前一片漆黑。而在分秒之间,当他回过神来之后,血液的温热灼烧着他的脸庞:妻子死在了他的面前,脸上带着最后的一丝微笑;手上的蓝色的荧光刀沾染上了血色,也逐步染红着整个炫白的地面。在这一瞬间,他想要大吼出来,却只能失声的无能为力的跪倒在了血泊中。后面的大门响起了急促的脚步声和呼喊声。他茫然的回头,看到一队卫兵在一名表情凝重的白衣研究者的带领下整齐的跑向这里。他绝望的将武器扔在了一旁,直直的看着满地深红色的鲜血。随后,眼前的一切变得模糊,周围素白的大厅变成了高坡。旁边的父亲一脸复杂和叹息的看着男主,然后伸出手来将男主拉了起来。
        
        关于最后这里到底发生了什么的设定:女主作为特殊的实验对象,不仅有了更多的特殊实验,组织也同时广泛的收集着有关女主的个人关系网以获得这种特殊能力的出处和原因。在得知了男主一家的背景之后,研究头子想出了一个疯狂的计划:男主一家一定会进攻设施设法营救女主,所以在他们进入女主的实验舱时,通过女主特殊的能力改变他们的记忆来达到控制他们的目的。此项能够作为一个绝好的实验机会,可以实验在非实验条件下记忆改变的情况和数据,同时位于实验中心也便于控制。所以,这实际上是一个已经设置好的圈套。

        唯一的不可预计变量:女主。在多天的实验中,多个异常的记忆单元改变已经非常严重的影响到了意识世界的稳定。然而,强大的意识依旧识别出了正确的记忆而始终保持着正常的意识,只是精神的摧残让她非常的虚弱。但另一个方面,这种特殊的能力让她能够扫描研究人员的记忆同时知晓了周围的状况。在知道为男主设立的陷阱后,她决定以最残酷的方式防止男主的记忆受到改变。在最后一次研究人员试图想要以她作为载体向周围广播篡改的记忆之时,她通过意识修改了这样的记忆,向男主传递了这样的信息:眼前的这个女人就是当年杀死母亲的凶手。而被拨到心弦的男主在瞬间的愤怒中则会砍向她自己,这样就能够从记忆影响中摆脱出来,拥有了能够离开的机会。在男主冲过来之前,她在口中不断的念着:"是我,杀死了你的妈妈",以加强意识层次的力量。同时,她将她所能收集到的尽可能多的记忆碎片,也拼劲全力向外传输。最终,男主的脑海中接收到了这样的记忆碎片:被篡改的记忆,女主再次篡改的记忆,女主大部分真实的记忆
    
        \newpage
        \subsection*{游戏之中}
            游戏本身发生在意识世界之中,上面的是cg(趴 本段由于融入了游戏性设计考量,所以会少量设计到游戏性的部分,而以剧情为主。这也是玩家将会开始获取到的信息。

            无神的男主被父亲掺回了家,男主由于极端的疲乏而进入了熟睡。而此时意识世界的结果将决定男主的未来。玩家将扮演营救女主这件事情所形成的记忆单元。由于受到了记忆修改的影响,玩家扮演的记忆单元本身就是破碎而混乱的。意识对于记忆单元的想法,及情绪将会决定记忆单元在意识世界的力量。设定上,记忆单元只拥有产生记忆单元的事件,它甚至是扭曲的,虚假的,但都是主角意识所认知的。

            这里会写上部分统一的游戏性设定,以供下面的关卡设定。

            抉择之门:在重要的记忆单元关卡之后,会进入选择环节。每一个问题都将是对于记忆的相关内容的判断。在做出选择之后,后续的关卡将可能会有非常大的变化。从设计的角度来说,这体现的是意识对于记忆的判定,是否选择接受,抑或是放弃思考。

            \subsubsection*{守护三界的堕天使}
            扮演的记忆单元最开始的记忆: 男主进入研究所营救妻子。由于男主脑中的记忆已经被篡改,在此段记忆单元中男主认为坐在类似大厅一样的囚禁隔间的人正是杀害母亲的凶手。此时,整个研究院变成了类似城堡的样子。在现实中,由于这是布下的圈套,男主的潜入过程没有任何的干扰,所以整个地图就是直直的类似宫殿门口到王座的直线。最终的boss为以妻子为原本的堕天使,触发场景动画:眼前出现母亲的样子,然后被boss击杀。通过对于boss的确认,男主辨认出这正是他的妻子。

            妻子的剧情物品掉落:

                \quad 地牢的钥匙:通往妻子关于研究设施的记忆碎片的关键道具。

            抉择之门:她真的是杀死我母亲的凶手吗?
                
                \quad -- 不可能!(正视记忆的判断, 需要前置支持,无前置时锁定)
                
                \quad -- 怎么会是她? (由于严重与事实的错位而导致的正常怀疑) 

                \quad \quad to "夏日的悲痛挽歌"

                \quad -- 这是个梦吧? (软弱的自我欺骗,被篡改记忆单元所利用) 

                \quad \quad to "夏日的悲痛挽歌"

            \subsubsection*{夏日的悲痛挽歌}

\end{document}
