\documentclass[12pt, a4paper]{article}   
\usepackage{indentfirst, amsmath, fontspec, listings, xcolor, amssymb}

\setmainfont{Microsoft YaHei}
    
\newfontfamily\consolas{Consolas}
\lstset{numberstyle = \small\consolas, basicstyle=\small\consolas}
    
\XeTeXlinebreaklocale "zh"
\XeTeXlinebreakskip = 0pt plus 1pt

\begin{document}
    \section*{基础信息}
    游戏名称:暂定

    游戏类型:ARPG
    \newpage 

    \section*{游戏背景设定}
        

        \textbf{意识世界}:每一个人都有期望的世界,而这样的幻景将被抽象影响着整个意识。在当前的认知和记忆的作用下,一个基础的所谓意识世界将会被搭建。而在人自我愿景的影响下,这个世界也戴上了更多的主观色彩。意识世界具有相似的结构,大部分都是由记忆填充的:记忆单元作为每一个记忆片段的主体,而这个主体具体是什么则各不相同:那段记忆中的自己,和自己交流的父母,常伴身边的宠物;所有被意识主体感知为有意识的东西都能作为记忆单元而表现出来。而记忆所涉及的其它的内容,则笼统的融合到了意识世界的环境当中。而真正被称为意识的逻辑过程则支配了每一个记忆单元在这个世界的行为和抉择,最终这些抉择的结果将改变记忆的形态,改变这个世界。严格的说,意识主体和意识世界是不断交互和影响的,其反应为认知和意识的不断交流以产生实际生活中的实际行为。 
        
       而在很长的一段时间内,这样的世界都是不可知的。对于意识主体以外的人而言,意识主体的所思所想自然不得而知;而对于意识主体本身,这样的一个世界只是一大段大脑活动的附属品。而现在,研究人员协力终于得知了那个世界的存在,而且能够从意识主体之外
        观察和影响到意识世界。 

        \medskip
        \textbf{记忆单元}:作为意识的重要组成部分,记忆是以单元的形式存在的。所有的记忆共享着意识主体的意识。然而,与其被称为共享,支配一词更加合适。意识主体对待记忆片段的看法决定了记忆在这个意识世界的体现,同时也决定了不同记忆对于意识主体的影响。举例而言,对于同一段时间的两种不同的记忆将会最终留存下一个作为真正的记忆,而这样的决定过程在意识世界中被表现为记忆单元的死亡和周围环境的崩塌,消失。记忆单元之间互动的结果实际上就是意识主体最后立场的表现,这将直接影响意识主体在现实当中的行动。 

        记忆的真假跟记忆单元本身的内容无关,而是意识主体的意识活动的结果。
\end{document}