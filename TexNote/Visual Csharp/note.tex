\documentclass[10pt, a4paper]{article}
    \title{Csharp Note}
    \author{ianaesthetic}
\usepackage{indentfirst, amsmath, fontspec, listings, xcolor}
\setmainfont{微软雅黑}

\newfontfamily\consolas{Consolas}
\lstset{basicstyle = \footnotesize\consolas}

\XeTeXlinebreaklocale "zh"
\XeTeXlinebreakskip = 0pt plus 1pt

\begin{document}
\maketitle
\newpage

    \section{(╯‵□′)╯︵┻━┻}
        \subsection{创建图形程序},
            本小节在使用的是「空白应用(通用windows应用)」。在资源管理器中, .xmal文件决定了GUI,其视图有两部分:代码和框体部分。 在vs2017中,右侧的「工具箱」中含有可以放到窗体上的各种控件。 在创建了一个Textblock之后,右侧的「属性」窗口可以用于具体设置组件的属性。
    
    \section{变量,操作符和表达式}
        没有被赋值的局部变量只有在赋值后才能使用,否则被认为是编译错误。在赋值float时,数值后面一定要加上F来表示这是一个float。默认情况下是一个double值,而在csharp中不能够出现类型不同的赋值。float 和 double 在 csharp中也做 mod操作。

        var类型可以用于隐式变量赋值自动决定变量的类型。
    \section{方法和作用域}
        使用表达式主体方法: 当方法非常简单的时候,可以简化为单个表达式构成的方法。如果有表达式有返回值则返回它,如果没有则方法自动为void方法
\begin{lstlisting}
int AddValue(int leftHandSide, int rightHandSide) 
    => leftHandSide + rightHandSide;
\end{lstlisting}

    \section{(╯‵□′)╯︵┻━┻}
    \section{(╯‵□′)╯︵┻━┻}
    \section{(╯‵□′)╯︵┻━┻}
    \section{类和对象}
        类实例的创建需要用到new + 构造函数

\begin{lstlisting}
c = new Circle(r);
\end{lstlisting}

        静态的理解:成员被声明为静态的时候,静态作为类的一部分而非是实例的一部分,所有的访问都依靠于类名称来访问。

        静态方法:当方法声明为static, 就只能用类名字而不是实例来访问了。 
\begin{lstlisting}
class Math{
    public static double sqrt(double x) {
        ...
    }
}

double y = Math.sqrt(y); 
\end{lstlisting}

        共享字段:当字段被声明为static时,所有类型的实例同时共享同一个字段,这个字段也只能用类名字来访问。 比较常见的应用是对于实例进行计数。

        在调用static关键字的字段或者方法时,可以使用静态using来省略前面的类名字,使用需要谨慎。
\begin{lstlisting}
using static System.Math
\end{lstlisting}
        
        private声明的字段可以通过使用表达式定义的方法来访问
        
        匿名类是直接用new 在后面跟上一个自定义的构造函数,其中包含字段和值。拥有同样数量,名称,类型,顺序的匿名类可以相互赋值,但是可能出事。匿名类必需初始化,不可以是静态。
\begin{lstlisting}
myAccountObject = new { Name = "John", Age = 47};
\end{lstlisting}

    \section{值和引用}
        \subsection{引用}
        基元类型在声明时直接导致编译器在内存中分配内存块,而类类型在声明的时候会分配一个引用的内存块,在对于实例进行赋值时new会在内存中新开一块内存并让之前的引用能够指向这里。 string也是一个类而不是一个基元类型。当下面代码出现的时候:
\begin{lstlisting}
Circle c = new Circle(10); 
Circle o = c; 
\end{lstlisting}

        o和c实际上指向同一块地址,更改其中的任意一个都会导致另一个也被更改。

        空条件操作符
\begin{lstlisting}
c.?Area(); 
\end{lstlisting}
        如果这个实例为null,则忽略这个句子。
        \subsection{可空类型}
\begin{lstlisting}
int? i = null; 
\end{lstlisting}

        这表示i是一个可空变量,可以被赋值,但是不能用于赋值和操作。是否为可空类型可以用HasValue成员来检测。
        \subsection{实参传递的引用}
\begin{lstlisting}
public static void Value(ref int param) {}
public static void Mag(out int param) {
    param = 42;
}
\end{lstlisting}
        ref表示对于参数的引用, 必需使用初始化之后的值。而考虑到要通过某些方法对于值进行初始化,out关键字可以传入未被初始化参数的引用,但是在方法中必须包含对于此参数的初始化。
        \subsection{内存空间于System.Object}
        在调用方法的时候,所有的参数和局部变量都是存储在栈当中。而类型创建所需要的内存则是在堆中进行分配的。 

        所有类都是System.Object的派生类, 同时System.Object也可以引用所有的对象和变量。用类实例对object对象的赋值相当于是创建了一个对于那个对象实例内存块的引用。

        装箱是将在栈中的数据赋值到object对象时自动复制其数据到堆中的一块内存当中。
\begin{lstlisting}
int i = 42;
object o = i;
\end{lstlisting}

        拆箱是指对于变量用object强制转换赋值的时候将堆中相应位置的内存复制到栈中。注意只有当object和变量同类型的时候强制转换才能够生效.
\begin{lstlisting}
int i = 42; 
object o = i; 
i = o;
\end{lstlisting}

        is操作符的作用是验证对象的类型是否是自己希望的,as是is的简化版本具体的使用:
\begin{lstlisting}
Circle c = new Circle(10); 
...
object o1 = c;
if(o1 is Circle) {...}
...
object o2 = c;
c = o2 as Circle.
if(c != null) {...}
\end{lstlisting}
        
        指针在Csharp中依然可以使用,但是必须将代码标记为unsafe时才能使用,同时还要指定程序「允许不安全代码」(项目属性,生成)
\begin{lstlisting}
unsafe {
    swap(&x, &y); 
}
...
public static unsafe void swap(int *x, int *y) {...}
\end{lstlisting}
\end{document}