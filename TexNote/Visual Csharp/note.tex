\documentclass[10pt, a4paper]{article}
    \title{Csharp Note}
    \author{ianaesthetic}
\usepackage{indentfirst, amsmath, fontspec, listings, xcolor}
\setmainfont{微软雅黑}

\newfontfamily\consolas{Consolas}
\lstset{numbers = left, numberstyle = \small\consolas, basicstyle = \small\consolas}

\XeTeXlinebreaklocale "zh"
\XeTeXlinebreakskip = 0pt plus 1pt

\begin{document}
\maketitle
\newpage

    \section{(╯‵□′)╯︵┻━┻}
        \subsection{创建图形程序},
            本小节在使用的是「空白应用(通用windows应用)」。在资源管理器中, .xmal文件决定了GUI,其视图有两部分:代码和框体部分。 在vs2017中,右侧的「工具箱」中含有可以放到窗体上的各种控件。 在创建了一个Textblock之后,右侧的「属性」窗口可以用于具体设置组件的属性。
    
    \section{变量,操作符和表达式}
        没有被赋值的局部变量只有在赋值后才能使用,否则被认为是编译错误。在赋值float时,数值后面一定要加上F来表示这是一个float。默认情况下是一个double值,而在csharp中不能够出现类型不同的赋值。float 和 double 在 csharp中也做 mod操作。

        var类型可以用于隐式变量赋值自动决定变量的类型。
    
\end{document}