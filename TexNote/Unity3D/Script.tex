\documentclass[a4paper, 12pt]{article}
    \title{script} 
    \author{ianaesthetic}
\usepackage{indentfirst, amsmath, listings, xcolor, fontspec}

\begin{document}
    \section{Scripting Overview}
        The whole note is devoted to introduce some special techniques towards engine. 
        \subsection{Creating and Using Script}
            The GameObject is controlled by the \emph{Component}, and script is used to modify and customize. 
            \subsubsection{How to create script}
            \subsubsection{Anatomy of Script File}
                Start(): It's called by Unity before gameplay begins.
                
                Update(): Handling the frame update for the GameObject
                
                The constructor is not needed as it's handled by the editor and may not take place at the start of gameplay. 

            \subsubsection{Controlling a GameObject}
                Script instance must be attached to a GameObject to control. 

                Debug.Log("message") can output to the console in the engine. 
            
        \subsection{Variables and the Inspector}
            Variables are declared public to be access in the Inspector. You can also make changes to the variables during the gameplay temporarily. 
        \subsection{Controlling GameObject Using Component}
            \subsubsection{Accessing Components}
                \emph{GetComponent} function can be used to get reference to the component properties. 
\begin{lstlisting}
RigidBody rb = GetComponent<RigidBody>(); 
\end{lstlisting}
                you can use it to access the parameter and manipulate the instances.

adasdasdasd 

\end{document}