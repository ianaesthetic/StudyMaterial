\documentclass[10pt, a4paper]{article}
    \title{Note} 
    \author{ianaesthetic}
\usepackage{indentfirst, fontspec, amsmath}

\begin{document}
    \section{Measuring GDP}
        \textbf{GDP}:Market value of domestically produced final goods and services. within a year or a quarter.
        \subsection{The Product Approach}
            We will calculate the value of the \emph{final goods and services} excluding the intermediate ones which are used up in the production of other goods and services in the same period that themselves are produced. 
            
            \emph{Capital goods} which are used to produce goods are included. \emph{Inventory investment} which is the amount that inventories of unsold finished goods, goods in process and raw material \emph{have changed} in this period. It's can also be done to add up \emph{value-added}. Tax is another form of adding value on the product from the government. 
            $$
                \text{Value-added = Total\ Revenue - value\ of\ intermediate\ goods}
            $$

        \subsection{The Expenditure Approach}
            It's the standard approach for calculating GPD in most countries. It's consists of several aspects: 

                Consumption: spending by domestic households on final goods and services including Consumer durables, non-durable goods and services. 
                
                Investment: spending for new capital goods (fixed investment) plus inventory investment(the change in the quantity of goods that firms hold in storage, including materials and supplies, work progress and finished goods. \emph{Raw material} is not counted as investment as they will be turned in to product ultimately.
                
                Government purchases of goods and services: Not all government expenditure are purchases of goods and services like interest rate of debt, transfers, and exchange not counted in current goods and services. Some spending is for capital goods to add nation's capital stock.

                Net exports: $Net\ Exports = Exports - Import$

                \begin{align*}
                    GDP &= (C - C_M) + (I - I_M) + (G - G_M) + X \\
                    &= C + I + G + NX
                \end{align*}

        \subsection{The Income Approach}
            Private sector: 
            $$Private\ sector = Y(GDP) + NPF + TR + INT - T$$

            Government sector: 
            $$Government\ sector = T - TR - INT$$
            \begin{align*}
                GDP &= After\text{-}tax\ Wage\ Income \\
                    &+ After\text{-}tax\ Profit \\
                    &+ Interest\ Income\\
                    &+ Taxes\\
            \end{align*}

            Income forms the basis for Expenditure(demand), Disposition of Expenditure determined Production(supply), and Revenue of Supply becomes Income. Three approach will get identical GDP.

            \subsection{Adjustment}
            rent of living is part of rented house part of GDP as it includes rent. So for you lives in your own house, we assumes the house provides service to you as part of house. However other things will have same situations while count not a big par of GDP so we don't consider them.
        
        \subsection{Example}
            Example1: Consumption outside of home country.$\Delta C = +6, \Delta NX = -6$
            
            Investment means that the good will generate new revenue, while consumption will not generate new revenue. 

            Inventory investment;
        \subsubsection{GDP and GDP per capita}
            Gini coefficient -- income inequality, the welfare of the economy
            Poverty line -- the number of people below the line to evaluate

            Gross national product(GNP): Total income earned by the \emph{nation's} factors of production, regardless of where located. This is used to estimate the completely national production.

            GNP - GPD = net factor payments(NFP). Factor payments such as profits(by capital), wages(by labour), rent, interest, etc. When NFP < 0, it means that the country receives foreign economic factors. 

        \subsection{Which market price to use}
            \textbf{Nominal GDP} uses the values of goods and services at current prices. 

            \textbf{Real GDP} uses the the values of goods and services at constant years. It is used to observe the real development concentrating on quantity rather than price. The gap of real GDP and Nominal GDP symbolizes inflation. 
        \subsection{Inflation} 
            \textbf{Inflation Rate}: $$\frac{P_t - P_{t - 1}}{P_{t - 1}}$$
            
            How to measure overall price level: \emph{GDP deflator} and \emph{CPI(consumer price index}), PPI, PMI, housing price, etc. 

            \subsubsection{GDP deflator and CPI}
            The GDP deflator is defined as:
             $$100 \times \frac{Nominal\ GDP}{Real\ GDP}$$
            
             \textbf{GDP delator}: a weighted sum of prices. Every year GDP is contributed by different goods, and the changing baskets of goods are described by \emph{paasche index}

             Consumer Price Index(CPI): tracking changes in the typical household's cost of living. The cost will be calculated among the typical consumer's baskets of goods. The baskets are described by \emph{Laspeyres index}

             $$CPI = 100 \times \frac{cost\ of\ basket\ in\ that\ month}{cost\ of\  basket\ in\ base\ period}$$

            Head Inflation: describe the total inflation from all areas and may experience sudden inflationary spikes like food and energy. 

            Core inflation: exclude certain items that faces volatile price movements, notably food and energy. This is commonly used.  CPI is also a weighted sum of prices, while the weights remain fixed as the  baskets are fixed. 

            The overstating of inflation rate by CPI:

                $a$. Situation bias
            
                $b$. Introduction of new goods 
            
                $c$. Quality bias
            \newline 

            Differences between CPI and GPD Deflator:

            $a$. Prices of capital goods (exclude / included)
            
            $b$. Prices of imported consumer goods (included / excluded)

            $c$. The baskets of goods (fixed / changing)

            $d$. Different frequencies (monthly / qtrly)

            \subsubsection{PPI \& PMI \& House Price}
            
            \textbf{Producer Price Index(PPI)}: a weighted index of prices measured at the wholesale, or producer level. It's also based on survey. 

            \textbf{Asset Price Inflation}: The rise of asset price. Inflation often refer to the consumer side. 

            \textbf{Purchasing Managers' Index(PMI)}: is based on data compiled from monthly replies to questionnaires sent to purchasing executives. A reading above 50 indicates an expansion of the sector while blow 50 represents a contraction. 

            There are two different types of PMI. \emph{Official PMI} is based on data collected by the NBS on State-owned companies. \emph{Private PMI} is derived rin a survey of private companies. 
            
        \subsection{Categories of the population}  
            \textbf{Employed}: Working at a paid job 
            
            \textbf{Unemployed}: not employed but looking for a job 
            
            \textbf{Labor forces}: the amount of labor available for producing goods 
            and services; all employed plus unemployed persons. 
            
            \textbf{no in the labor force}: not employed, not looking for work, such as distressed worker. 

            $$Unemployment\ rate = \frac{Number\ unemployed}{Labor\ force}$$
            $$Labor\ force\ participation\ rate = \frac{Labor\ force}{Total\ working\ age\ population}$$
    
    \section{The Labor Market: Productivity, Output, and Employment}
        \subsection{The production function: $Y = A\cdot F(K, L)$}
        \subsubsection{Factors of production}
            $K$ = capitals: tools, machines, and structures used in production.
            
            $L$ = labor: the physical and mental efforts of workers. (also commonly denoted as $N$ in macro) 

            $A$ = other: Total Factor Productivity[TFP](Management, Weather, Policy, Technology, etc)
            
            \textbf{Potential GDP}: the GDP with full-employment output
        \subsubsection{Property}
            \textbf{Slopes Upward}: More of any input produces more output. 
            
            \textbf{Diminishing marginal products}: Slope becomes 
            
            \textbf{Marginal Product of Capital}: The extra output the firm can produce from an additional unit of capital. 
            $$MPK = F(K + 1, L) - F(K, L)  = \frac{\partial F}{\partial K}$$
            
            \textbf{Marginal Product of Labor}: The extra output the firm can produce from an additional unit of labor.  
            $$MPL = F(K, L + 1) - F(K, L) = \frac{\partial F}{\partial L}$$

            MPK and MPL, as the property, are always positive and has a diminishing rate. Their units are in \emph{goods} per unit of labor.  

            \textbf{Diminishing Marginal Returns}: As only one input is increased, its marginal product falls.

            \textbf{Returns to scale}: scale all inputs by the same factor, $Y_1 = F(K_1, L_1)$, $Y_2 = F(zK_1, zK_2)$: 

            {Constant returns to scale}: $Y_2 = zY_1$
            
            {Increasing returns to scale}: $Y_2 > zY_1$

            {Decreasing returns to scale}: $Y_2 < zY_1$
            
            \subsubsection{Cobb-Douglas Production Function} 
            $$Y = AK^{\alpha}L^{1 - \alpha}$$

            Each factor's marginal product is proportional to its average product. 
            $$MPK = \frac{\alpha Y}{K}, \quad MPL = \frac{(1 - \alpha) Y}{L}$$

            $\alpha$ is the capital share of total income, with a constant returns to scale.

            $$Capital\ income = MPK \times K = \alpha Y$$
            $$Labor\ income = MPL \times L = (1 - \alpha)Y$$

            \subsubsection{Supply shock}
                Supply shocks are derived from the influence of supply side without control. It has \emph{positive} shock and \emph{adverse(negative)} shock.

        \subsection{The Demand for Labor}
            Assumptions: Short-run analysis, workers are indifferent, markets are competitive, Profit maximization. The cost is real wage $w = \frac{W}{P}$(P is average product, like CPI, GDP deflator); benefit = MPL; The maximization is $w = MPL$.

            Labor demand curve is exactly the MPL curve as $w = MPL$. So the labor demand curve is downward sloping. 

            The difference of Move Along the Curve and Shift the Curve. Move Along the Curve describe the relationship between the elements along the fixed curve. Shift the Curve is move the curve as some circumstances change, so with a given element another element change.

            Factors that shift the labor demand curve(depend on the factors of MPL): Productivity, Capital stock.
            
            Aggregate demand for labor is the sum of the labor demands of all firms in the economy. 
            
        \subsection{The Supply of Labor}
            The Income-leisure tradeoff: $U(C, L)$ is judged. The income and substitution effects are also included. 

            \textbf{Substitution effect}: Higher real wage encourages work, since reward for working is higher, relying on short-turn.

            \textbf{Income effect}: Higher real wage increases income for same amount of work time, so person can afford more leisure, soo will supply less labor, relying on long-run or permanent change. 
            
            The factor that shift the labor supply curve: Wealth and Expected future real wage. 

            A temporary rise of real wage(Substitution effect) will increases labor supply, while a permanent increase in the real wage will make the labor supply falls. 

            Aggregate supply of labor is the total amount of labor supplied by everyone in the economy.
            Other factors for Aggregate supply are Working-age population and Participation rate. 
        \subsection{Labor market equilibrium}
            Aggregate quantity of labor demanded = aggregate quantity of labor supplied. However wage is sticky and may be above the equilibrium which leads to unemployment. 

            Example of changes: government policy to production / retirement age. 
    \section{Goods Markets}
        Assume it is in closed economy with $NX = 0$, $GDP = GNP$, $$NFP = 0$$.
        
        Aggregate demand(spending) represents equilibrium of goods and services: 
        $$Y = C + I + G + NX = C + I + G$$
        The conversion of this equation means the \emph{national savings} and represent equilibrium of loanable funds. 
        $$I = Y - C - G$$ 
        \subsection{Consumption and Saving}
            Household(disposable income $Y - T$): 
            $$S = (Y - T) - C$$

            $C^d$ is the aggregation of consumption amount desired by households. The $C^d$ can be perceived as: 
            
            $$C^d = \sum_{i = 1}^{N}C_i$$

            The household will finally sum together and translate to the aggregate level. 

            Desired national saving(which is the planned saving and investment to find the equilibrium): 
            $$S^d = Y - C^d - G$$
            
            The consumption and savings lead to savior and borrower which is two faces of one thing. The savings introduce tradeoff between current consumption versus future consumption, measuring by real rate.
            
            \subsubsection{Interest Rate}

            \textbf{Interest Rate}: a rate of return promised by a borrower to a lender. 

            \textbf{Real Interest Rate}: rate at which the real value f an asset increases over time, representing real purchasing power. 
            
            \textbf{Nominal Interest Rate}: rate at which the nominal value of an asset increases over time on the normal contrast, affected by the inflation rate. 
            
            Fisher effect: 
            $$(1 + i) = (1 + r)(1 + \pi)$$
            $$i = r +  \pi + r\pi$$
            As $r\pi$ is very small and it can be neglected. 
            $$r = i - \pi$$
            
            $r$ is derived from goods market equilibrium and $\pi$ can be observed.  As $\pi$ needs to be expected to introduce ex-ante rate(prior, before realization): 

            $$i^e = r + E\pi$$

            We also have a after realization we will have a ex-pose $i$ similar to the above. This equation can be used to represent inflation rate's feature. The expected interest rate is usually used. The price of 1 unit of current consumption is 1 + $r$ units of future consumption. 
            
            \subsubsection{Consumption-smoothing Motive}
                It's the desire to have a relatively even pattern of consumption over time. 

                Permanent Income Hypothesis: Focuses in what consumers do with stochastic income receipts, which is not very suitable for China. 

                Life-cycle Hypothesis: Focuses on predictable changes in income over the life cycle. 

                The hypothesis  tries to explain the consumption-smoothing motive. 

            
            \subsection{Consumption function}

                \textbf{Marginal Propensity to Consume}: the change in C when disposable income increase by one dollar. MPC is in the range of $(0, 1)$ due to consumer-smoothing motive. MPC is actually the slope of consumption function in household level, which can be used to evaluate aggregate level $C^d$.
                $$MPC = \frac{\Delta C}{\Delta (Y - T)}$$

                Factors will affect consumption will present as followed. 

                \subsubsection{Current Income $Y_t$}
                    As $Y_t \uparrow$, both consumption and savings $\uparrow$ as $0 < |\Delta C| < |\Delta (Y_t - T_t)|$ (consumption-smoothing motive).  The aggregate level has a similar change. 
                    $$\Delta C_{t + 1} = \Delta Y_t \times MPC \times (1 + r)$$

                \subsubsection{Expected Future Income}
                    Higher expected future income leads to more consumption today with saving falls like deflation.  Deflation will discourage consumption.  
                \subsubsection{Changes in Wealth}
                    $$Wealth = Asset - Liability$$
                    $$Savings = Current\ Income - Current\ Expenditure$$

                    Savings can contribute to wealth. Wealth is not a part of saving. When wealth $\uparrow$, consumption $\uparrow$, savings $\downarrow$, which is similar to mechanism of future income.
                \subsubsection{Changes in Real Interest Rate} 
                    Increased real interest rate has two opposing effects. For saver, interest rate is benefit of saving: 
                    
                    \quad Substitution effect: positive effect on saving, since rate of return is higher. 

                    \quad Income effect: negative effect on saving, since it's easier to save to obtain a given amount in the future.\newline
                    
                    
                    For borrower, interest rate is the cost of borrowing.

                    \quad Substitution effect: positive effect on savings, since the cost is higher. 

                    \quad Income effect: Positive effect on saving, since the higher real interest rate means a loss of wealth.

                    The mixed result will result in a slightly positive slope between $r$ and $S^d$ 
                \subsubsection{Changes in Tax}
                    Interest earning are taxed: $t$ is the rate at which nominal interest income is taxed. Expected after-tax real interest rate: 

                    $$r_{a\text{-}t} = (1 - t)i - E\pi$$

                    The tax will affect interest rate, which influence consumption.
                \subsubsection{Fiscal Policy}

                    The fiscal policy is under balanced Gov Budget:
                    $$Gov\ Budget = T - G - 0$$

                    Affects desired consumption through changes in current and expected future income. it will directly affect desired national saving. 
                    
                    There are two channels: 
                    
                    \quad An increase in current government purchases (G) 

                    \quad Adjusting the timing of taxes to change consumption behavior.

                    Government purchases(temporary increase): There two approaches to compensate $G \uparrow$: $T_t \uparrow$ or $T_{t+1} \uparrow$:
                    
                    $$T_t\uparrow \  \Rightarrow \  0 < |\Delta C| < |\Delta (Y_t - T_t)|$$
                        $$S = (Y_t - T_t) - C_t,\ MPC < 1, \quad  S\downarrow$$
                        $$S^d = Y_t - C_d - G,\ |\Delta C| < |\Delta G| = |\Delta T_t|,\quad S^d\downarrow$$


                    $$T_{t+1}\uparrow \Rightarrow (Y_{t+1}^e - T_{t+1})\downarrow \Rightarrow C_t \downarrow$$
                    $$S = (Y_t - T_t) - C_t,\ \Delta C < 0,\ \Delta T_t = 0, \quad   S\uparrow$$
                    $$S^d = Y_t - C_d - G,\ |\Delta C| < |\Delta G| = |\Delta PV(T_{t+1})|,\quad S^d\downarrow$$

                    Adjusting of the timing of taxes: 

                    \quad Assumption(Ricardian equivalence proposition): If future income loss exactly offsets current income gain, no change in consumption as people are rational. $T_t \uparrow$ and $T_{t+1} \downarrow$ 

                    $$S = (Y_t - T_t) - C_t,\ \Delta C_t = 0,\ \Delta T_t \uparrow,\quad  S \downarrow$$
                    $$S^d = Y_t - C_d - G,\ \Delta C^d = 0,\ \Delta G = 0, \quad \Delta S^d = 0$$

                    \quad Assumption(Myopic): In practice, people may  not see that future taxes will rise if taxes are cut today. $T_t \uparrow$ and $T_{t+1} \downarrow$ 
                    $$S = (Y_t - T_t) - C_t,\ |\Delta C_t| < |\Delta (Y - T_t),\quad S\downarrow$$
                    $$S^d = Y_t - C_d - G,\ \Delta C^d < 0,\ \Delta G = 0, \quad S^d\uparrow$$

        \subsection{Investment}
            Investment refers to the purchase or construction of capital goods, including residential and nonresidential buildings, equipment and software used un production, and additions to inventory stock. Investment intends to fluctuate over the business cycle. Investment decision has a time delay differing from labor markets. 
            \subsubsection{desired capital stock}
                \textbf{Desired Capital Stock} is the amount of capital that allows firms to \emph{earn the largest expected profit}($Max\ E_t(profit_{t+1}$). Desired capital stock depends on costs and benefits of additional capital, similar to labor market with expected profit. For each unit of investment, the balance is seeked similar to labor market: 

                $$MPK^f = E_t(\frac{\Delta Y_{t+1}}{\Delta K_{t+1}})\  = uc\  \Rightarrow E_t(benefit_{t+1}) = E_t(cost_{t+1})$$

                The benefit of this can be using the Expected production function.

                $$p_k = \text{real\ price\ of\ capital\ goods\ per\ unit}$$
                $$d = \text{depreciation\ rate}$$
                $$e = \text{expected\ real\ interest\ rate}$$

                To get expected cost per unit, we define user cost as two parts include depreciation rate and expected real interest rate:
                $$uc = p_k \times (r + d)$$

                $uc$ is a horizontal line and doesn't vary with $K$, like the cost $w$ for per-unit labor. The factors are changes in the real interest rate, depreciation rate , price of capital, or technology factor that will affect $MPK^f$.

                With taxes, the return to capital is only $(1 - \tau)MPK^f)$. It can be also be considered the increase in $uc$. This is the tax-adjusted user cost of capital is $\frac{uc}{1 - \tau}$
            \subsubsection{Investment and capital stock }
                
                The capital stock changes from two opposing channels: New capital increases the capital stock; The capital depreciates. 

                $$K_{t + 1} - K_{t} = I_t - d_{K_t}$$

                Rewrite the equation to get the Investment: 

                $$I_t = K_{t + 1} - K{t} + d_{K_t}$$

                $-K_t + d{K_t}$ is fixed in the decision to the investment. The investment decision relies solely on the desired capital stock in one-on-one relationship. This also means investment covers the net increase in capital stock and depreciation.
                
                The investment is derived from desired capital stock, and desired capital stock is derived from $MPK^f$ and $uc$.  
        \subsection{Goods}
            For goods market equilibrium: Aggregate quantity of goods supplied = aggregate quantity of goods demanded(desired).
            
            $$Y = C^d + I^d + G$$

            The real interest rate adjusts to bring the goods market to equilibrium.


            Goods market equilibrium is changed to loanable funds. 
            $$S^d = I^d$$

            Both of them are a function is dependent on interest rate, forming Saving curve S and Investment curve. 

            Factors for $S^d$: $Y$, $Y^e$, $W$, $G \uparrow$ $T \uparrow$, $T_t \uparrow$ $T_t \downarrow$.

            Factors for $I^d$: $d$, $uc$, $\tau$, $MPK^f$

            the interest rate here is the representation of all rates for the structure of interest rate. 

    \section{Asset Market: the Monetary System and the Quantity Theory of Money}
        \subsection{Asset Market}
            \subsubsection{Money}
                \textbf{Money} is the stock of assets can be readily used to make transactions; 
                
                Functions : 

                a). medium of exchange; 
                
                b). store of value; 
                
                c). unit of account; 

                Types: 

                a). Fiat money: has no intrinsic value 

                b). Commodity money: has intrinsic value, e.g. gold coins; 

                The \emph{money supply} is the quantity of money available in the economy. \emph{Monetary policy} is the control over the money supply. Monetary policy is conducted by a country's \emph{central bank}. 

                measurements: $C$, $M1$, $M2$

                $$M = C + D$$

                Reserves(R): the portion of the deposits that banks have not lent

                A bank's liabilities include deposits and Asset include reserves and outstanding loans. Due to the Reserves, there are 100-percent-reserve banking and fractional-reserve banking. 
                
                When all money deposited in the bank: 
                
                $$\text{Money Supply} = \frac{1}{rr} * C$$

                $rr$ is the proportion of the reserves. This doesn't increase wealth but increase liquidity.

                \textbf{Bank Capital}: the equity a bank's owners have put into bank. 

                Leverage: the use of borrowed money to supplement existing funds purposes of investment; 

                $$\text{Leverage ratio} = \frac{\text{Asset}}{\text{Equity}}$$

                \textbf{Monetary base}, controlled bt the central bank: 

                $$B = C + R$$

                \textbf{Reserve-deposit ratio}, depends on regulations \& bank polices:
            
                $$rr = \frac{R}{D}$$

                \textbf{Currency-deposit ratio}, depends on household's preferences 

                $$cr = \frac{C}{D}$$

                $$M = C + D = m \times B$$
                $$m = \frac{C + D}{B} = \frac{C + D}{C + R} = \frac{cr + 1}{cr + rr}$$

                We can find the relationship between monetary base which is controlled by money policy and money supply. The central bank can just have partial influence on the money supply. The monetary multiplier describes the influence as government cam directly influence of monetary base. 
    \subsubsection{The instruments of monetary policy}
        The Reserves are the point that is partially influenced.They can be ineffected. They can use:
        
        (a). Open Market operations: to increase bases, PBOC will buy government bonds or repo by reserves from commercial banks, resulting the reserves increased in the market. 
        
        (b). Discount Rate: the interest rate PBOC charges on loans to bank. To increase the base, PBOC will lower the discount rate encouraging banks to borrow more reserves. 

        (c). Reserve requirement ratio(RRR). In common day, $rr \approx RRR$. In the crisis, $rr > RRR$ as they feel unsafe, resulting excess reserves $(rr - RRR) \times D$. In the crisis, RRR is not an active instrument.
        
        (d): interest on reserve.

        Decline in money supply mainly depends on monetary multiplier.
    \subsubsection{The Demand for money}
        The demand or money is the quantity of monetary assets people want to hold. The nominal money demand is affeced by the following factors. 

        Price Level: the higher the price level, the more money you need for transaction. Nominal money  demand is thus proportional to the price factor. It's an one-to-one relationship. 

        Real Income: The more transactions you conduct, the more money you need.

        Interest rate: An increase in the interest rate or return on non-monetary asset decrease the demand for money($i$). Another interest form is monetary interest rate(always deposit); increase in the interest rate on money increases money demand ($i^m$). However, the monetary interest is low and constant so is sometimes not considered. 

        For the nominal money demand: 
        
        $$M^d = PL(Y, i, i^m)$$ 
        $$\frac{\partial M^d}{\partial Y} \in (0, 1) \times P, \quad\frac{\partial M^d}{\partial i} < 0, \quad\frac{\partial M^d}{\partial i^m} > 0$$

        More specifically, we ignore monetary interest:
        $$M^d = PL(Y, i)$$

        For real money demand that can be investigated: 
        $$\frac{M^d}{P} = L(Y, r + \pi^e)$$
        
        Other factors: Wealth($\uparrow M^d\downarrow$), Risk(depends on exact risk), liquidity of alternative assets($\uparrow M^d\downarrow)$, Payment technologies($\uparrow M^d\downarrow)$;
    
    \subsubsection{Asset Market  Equilibrium} 
        $$\frac{M}{P} = L(Y, r + \pi^e)$$
        
        $Y$, $r$ is derived from labor market and goods market, Y can both presenting Income or Production as they are one-to-one: 
        $$Y = AF(K, L)$$
        $$r \sim S^d$$

        $r$ will influence $K_{t+1}$ as $MPK^f = UC$, so will influence $Y_{t+1}$ and $I_a$
    \subsubsection{The quantity theory of money}
        velocity: the rate at which money circulates; number of times the average dollar bill changes hands in a given time period. 
        
        assumption: nominal GDP a a proxy value of total transactions: 

        $$V = \frac{P \times Y}{M}$$

        The real terms: 
        $$V = \frac{Y}{(\frac{M}{P})} = \frac{Y}{L}$$
        
        A strong assumption: when we assume $L = kY$ in long run, $V$ is a constant. 
        $$M \times \overline{V} = P \times Y$$
        $$\frac{\Delta M}{M} + \frac{\Delta V}{V} = \frac{\Delta P}{P} + \frac{\Delta Y}{Y}$$
        $$\frac{\Delta V}{V} = 0, \quad \frac{\Delta P}{P} = \pi$$
        $$\pi = \frac{\Delta M}{M} - \frac{\Delta Y}{Y}$$

        Hence, the quantity theory predicts a one-for-one relation between changes in the money growth rate and changes in the inflation rate. 
\newpage
\section{Business Cycle theory: the Economy in Short Run}
    \subsection{Business Cycle}
        Five main points about business cycles: 

        \quad (a). Business cycles are fluctuation of aggregate economic activity, not a specific variable 

        \quad (b). There are expansions and contractions 

        \quad (c). Economic variables show co-movement

        \quad (d). The business cycle is recurrent, not periodic

        \quad (e). The business cycle is persistent. 
    
        %Cspell:ignore procyclical, countercyclical, Acyclical

        Factors involved: Procyclical(coincident, leading, lagging, timing not designated);Countercyclical; Acyclical; 
            
        Time horizons in macroeconomics: 
        
        \quad long run(classical macro theory): prices flexible; output determined by supply side; Changes in demand only affect prices; unemployment equals its natural rate; 

        \quad short run: price fixed(sticky); output depends on demand; unemployment negatively related to output; 

        \medskip

        IS curve: investment \& saving curve 

        LM curve: Asset Market Equilibrium

        \subsection{IS-LM model} 
        \subsubsection{IS curve}
            a graph of all combination of $r$ and $Y$ that result in goods market equilibrium.

            $$S = Y_t - C(Y_t - \overline{T}, r) - \overline{G}$$

            as $Y_t\uparrow$, $S\downarrow$, which will form a downward slope curve; 

            Pay attention to factors to shift the curve; 
        \subsubsection{LM curve}
        
        a graph of all combinations of $r$ and $Y$ that equate the supply and demand for real money balances.

        In short run: 
            $$\frac{M^s}{\overline{P}} = L(Y, i)$$

            supply curve is the vertical line and demand side is a downward sloping curve. As $Y\uparrow$, $L\uparrow$, which will form a upward slop LM curve.

            also there are some factors to shift the curve.  
       
            The intersection of IS-LM curve is the equilibrium in both markets. 
        \subsubsection{The FE line: Labor Market equilibrium}
            In IS-LM, Labor market showed how equilibrium in the labor market leads to employment at its full-employment level and output at its full-employment level. It will be a vertical line indicating the economy's potential. It's a reference point as it's a long-term variable. 
        \subsection{IS-LM and AD}
            The price level con only be derived by money market. When price decreases, $\frac{M}{P} \downarrow$, $r\downarrow$, $Y \downarrow$, so we get $P - Y$ AD curve. The actual data P is gdb deflator and Y is Real GDB (represents quantity)
        \subsection{AS curve}
            In short run, the price level is fixed and firms are willing to sell as much at the price level so AS is a horizontal line; In the long run we assume the output is the full-employment, so $Y$ is fixed and the supply curve is a vertical line. Here we think the unemployment equal its natural rate which is not zero. In reality, the SR output is less than full-employment output level related to AD.

\section{Shock}
    Demand Shock(on IS or LM) and Supply Shock(SR or LR)
    \subsection{Interaction between monetary and fiscal policy}
        \subsubsection{Fiscal Policy}
               An increase in government purchases: $S\uparrow$, $r\uparrow$, $\overline{Y}$, IS shifts right to create new equilibrium with $Y\uparrow$.
            
            A Tax cut: $S\uparrow$, similar;
            
            The new equilibrium will have the cost of increasing of interest rate, which will lead to the decrease of investment. 
            
            \emph{Crowding-out effect}: an expansionary fiscal policy leads to an increase in interest rate, which in turn leads to a reduction in private investment. Government spending crowds out private investment. 
        
        \subsubsection{Monetary policy}
            An increase in M: $r\downarrow$, LM curve move shift as well. The most different point is a decreasing $r$. It is composed of increasing in $C(Y-T, r)$ and $I(r)$.
        \subsubsection{Interaction} 
            the Central Bank's responses to $\Delta G > 0$ 
            
            \quad To Hold M constant: the crowding-out effect stays. This will get lower inflation. 
            
            \quad to Hold r constant: increase M ro shift LM curve right. This can counter out Crowding-out effect. ($\overline{I}$, $G\uparrow$,  $Y\uparrow$). This is usually taken place in recession.

            \quad to Hold Y constant: reduce M to shift LM curve left with even higher interest rate. This is usually taken place in overheating. 

    \subsection{Exogenous shocks} 
        shocks temporarily push the economy away from full employment.

        How long dose it take to get the long run: 

        \quad classical theory: prices adjust rapidly $\rightarrow$ recessions are short-lived and no need for government intervention 

        \quad Keynesian theory: prices adjust slowly 

        A supply shock alters production costs, affects prices that firms charge. The adverse shock will move the supply curve up. 

        IS shock: exogenous changes in the demand for goods and services.

        LM shocks: exogenous changes in the demand for money 

    \subsection{Stabilization}
        policy actions aimed at reducing the severity o short run economic fluctuations. For LM curve, as central bank targets in different targets, which will result in different LM curve. When the aim is to hold $M^s$ constant, the LM curve will become a up sloping curve. when the aim is to hold $r$ constant, then the LM curve will become a vertical line. With different goals will determine the outcome of shocks. 

        The demand shock is to make the demand get back to the initial point; For the supply shock, if you want to make price stable, government should wait. 

    \subsection{From the short run to the long run} 
        When the short-run equilibrium is not equal to potential, it will go back the potential level with price changes. So in the long run the output will stay in $\overline{Y}$. So in the long run monetary will only affect price level with out changes in real output. This is called money  neutrality. 

        Without out any policies, the price will move along AD curve to the new equilibrium. This will reflect in the change of LM curve in the IS-LM curve. 
\newpage
\section{Open Economy}
    $$Y = I + C + G + NX \Rightarrow NX = S -I $$

    Trade surplus: output > spending and exports > imports; 
    
    Size of the trade surplus = $NX$

    Trade deficit: spending > output and imports > exports; 
    
    Size off the trade deficit = -$NX$

    $$\text{Net capital outflow} = S - I$$

    $S > I$, country is a net lender; $S < I$ country is a net borrower;

    \subsection{BOP}
    A summary of a country's transactions with other countries;
            
    Current Account: $CA$, major good trade balance(NX)

    Non-reserve Financial Account: $KA$: Investment and Saving $S - I$
        
        \quad Direct Investment (FDI ODI)

        \quad Portfolio Investment
        
        \quad Other 

    Reserve Assets: $RA$

    Net errors and Omissions Account: $EA$ 

    \subsection{Assumptions}
        a. domestic \& foreign bonds are \emph{perfect substitutes} (same risk and maturity)

        b. \emph{perfect capital mobility}: no restrictions on international trade in assets 

        c. \emph{economy is small}: can't affect the world interest rate, denoted $r^*$

        a, b implies $r = r^*$, c implies $r^*$ is exogenous

    \subsection{Policies}
        \subsubsection{Fiscal policy at home}
            expansionary policies $\rightarrow$ $S\downarrow$ $r^*$ hold $\rightarrow$ $\Delta I = 0$ $\Delta NX = \Delta S < 0$
        \subsubsection{Fiscal policy abroad}
            The world interest rate will be affected resulted from world saving curve 

            expansionary policies $r^*\uparrow$ $\rightarrow$ $\Delta I < 0$, $\Delta S > 0$ $\rightarrow$ $\Delta NX = \Delta(S - I) > 0$
        \subsubsection{Increase Invest Demand} 
            $\Delta I > 0$, $r^*\ S$ hold $\rightarrow$ $\Delta NX < 0$

    \subsection{Exchange Rate}
        Nominal exchange rate ($e$) (dollar / yuan)\& Real exchange rate ($\varepsilon$). The interest is evaluated in direct way. We mainly consider real 

        $$e = \frac{\text{Foreign Currency}}{\text{Home Currency}}$$
        $$\varepsilon = \frac{\text{Foreign goods}}{\text{Home goods}}$$

        \begin{align*}
            \varepsilon &= \frac{e \times P}{P ^ *} 
        \end{align*}

        \begin{center}
        $\varepsilon \uparrow$ $\rightarrow$ $X(\varepsilon) \downarrow$, $M(\varepsilon) \uparrow$ $\rightarrow$ $NX \downarrow$
        \end{center}
   
        $$NX(\varepsilon) = S(r^*) - I(r^*)$$
        $$\frac{M^*}{P^*} = L^*(r^* + \pi^*, Y^*)$$
        $$\frac{M}{P} = L(r ^ * + \pi, Y)$$
        $$\frac{\Delta e}{e} = \frac{\Delta \varepsilon}{\varepsilon} + \frac{\Delta P^*}{P^*} - \frac{\Delta P}{P} = \frac{\Delta \varepsilon}{\varepsilon} + \pi^* - \pi$$

        The equation before will determine real exchange rate. For home currency $S(r^*) - I(r^*)$ is the currency supply of home as home currency flows out and invested abroad, and $NX(\varepsilon)$ is the currency demand of home as exported goods will be calculated in home currency. Consider from the foreign currency will inverse the conclusion.  

    \subsubsection{floating vs. fixed exchange rates}
        Floating exchange rates: $e$ is allowed to fluctuate in response to changing economic conditions, which allows monetary policy to be used to pursue other goals  

        Fixed exchange rates: ..., which avoid uncertainty and volatility, making international transactions easier; Discipline monetary policy to prevent excessive money growth \& hyperinflation. 

        The impossible trinity: A national can't have free capital flows, independent monetary policy, and a fixed exchange rate simultaneously. 

            \quad US: free capital \& independent monetary policy 

            \quad HonKong \& Europe: free capital flows \& fixed exchange rate 

            \quad China(before 2005): Independent monetary policy \& Fixed exchange rate 
    \subsection{Experiment}
        Fiscal policy: expansionary policy, $S(r^*) - I(r^*) \downarrow$, $\varepsilon \uparrow$; Only $S$ change, $I$ and $r^*$ stay same. 

        Fiscal policy abroad: expansionary policy, $r^* \downarrow$, $S(r^*) - I(r^*) \uparrow$ , $\varepsilon \downarrow$ 

        Trade policy: those that directly influence the amount of goods and services exported or imported, usually tax or quota on foreign imports to protect domestic industries, or export subsidies. For $IM \downarrow$, $NX(\varepsilon) \uparrow$ will shift left, $\varepsilon \uparrow$, $\Delta NX  = 0$ as currency supply is the same. 

    %Cspell: ignore Solow
\section{The Growth Theory: Solow Model}
    \subsection{The Sources of Economic Growth}
        $$\frac{\Delta Y}{Y} = \frac{\Delta A}{A} + a_k\frac{\Delta K}{K} + a_L\frac{\Delta L}{L}$$

        In Cobb-Douglas:
        $$a_k = \text{Capital Ratio},\quad a_L = \text{Labor Ratio} = 1 - a_k$$
    \subsection{Solow Model}
        Assumption: Closed economy, no government, focus on $C$ and $I$ and per-worker, ignore productivity growth, assume a constant returns to scale(Cobb-Douglas). 
        $$C_t = Y_t - I_t$$

        $$Y_t = F(K, L)\quad \Rightarrow\quad y_t = f(k),\quad MPK = MPk$$
        $$c_t = y_t - i_t$$
        $$s = \text{the saving rate, the fraction of income that is saved}$$
        $$c_t = (1 - s)y_t,\quad \text{saving} = \text{Investment} = sy_t$$

        \subsubsection{Steady States}
            A steady state is a situation in which the economy's output per worker, consumption per worker and capital stock per worker are constant. That is, in the steady state, $y_t$, $c_t$, $k_t$ don't change over time. 

            $$\text{Break-even investment} = (n + \delta)k$$
            $$\delta k = \text{depreciation},\quad nk = \text{population growth}$$
            $$\Delta k = i - (\delta + n)k = sf(k) - (\delta + n)k$$

            when $\Delta K = 0$ occurring in  $k^*$,this is the steady state. Under Cobb-Douglas, we can find a unique intersection between $sf(k)$ and $(\delta + n)k$. As long as $k < k^*$, the k will continue toward $k^*$. $k^*$ will stay same as all factors are unchanged. 

            Prediction: $s\uparrow$, $k\uparrow$, $y\uparrow$;\quad $n\uparrow$, $k\downarrow$, $y\downarrow$

        \subsection{The Golden Rule}
            Different values of $s$ lead to different steady states. It's aimed to maximized $c$;

            $$c^* = y^* - i^* = f(k^*) - (\delta + n)k^*$$
            $$\frac{\partial c}{\partial k} = 0 \quad \Rightarrow \quad MPK = \delta + n $$

            This comparison is between steady state. The economy in the steady-state may not be the golden steady-state. To achieve golden steady state, government can try to adjust $s$. 
            
\end{document} 