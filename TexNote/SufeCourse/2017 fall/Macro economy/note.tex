\documentclass[12pt, a4paper]{article}
    \title{Note} 
    \author{ianaesthetic}
\usepackage{indentfirst, fontspec}

\begin{document}
    \section{Measuring GDP}
        \textbf{GDP}:Market value of domestically produced final goods and services. within a year or a quarter.
        \subsection{The Product Approach}
            We will calculate the value of the \emph{final goods and services} excluding the intermediate ones which are used up in the production of other goods and services in the same period that themselves are produced. 
            
            \emph{Capital goods} which are used to produce goods are included. \emph{Inventory investment} which is the amount that inventories of unsold finished goods, goods in process and raw material \emph{have changed} in this period. It's can also be done to add up \emph{value-added}. Tax is another form of adding value on the product from the government. 
            $$
                Value-added = Total\ Revenue - value\ of\ intermediate\ goods.
            $$

        \subsection{The Expenditure Approach}
            It's the standard approach for calculating GPD in most countries. It's consists of several aspects: 

                Consumption: spending by domestic households on final goods and services including Consumer durables, nondurable goods and services. 
                
                Investment: spending for new capital goods plus(fixed investment) and inventory investment(the change in the quantity of goods that firms hold in storage, including materials and supplies, work progress and finished goods. \emph{Raw material} is not counted as investment as they will be turned in to product ultimately.
                
                Government purchases of goods and services: Not all government expenditure are purchases of goods and services like interest rate of debt, transfers, and exchange not counted in current goods and services. Some spending is for capital goods to add nation's capital stock.

                Net exports: $Net\ Exports = Exports - Import$
                
                $$GDP\begin{align*}
                     &= Consumption(C) \\
                     &+ Investment(I) \\
                     &+ Government\ Expenditure(E) \\
                     &+ Nex\ Exports(NX)
                \end{align*}$$
        


        \subsection{Example}
            Example1: Consumption outside of home country.$\Delta C = +6, \Delta NX = -6$
            
            Investment means that the good will generate new revenue, while consumption will not generate new revenue. 

            Inventory investment;
        \subsubsection{GDP and GDP per capita}
            Gini coefficient -- income inequality, the welfare of the economy
            Poverty line -- the number of people below the line to evaluate

            Gross national product(GNP): Total income earned by the \emph{nation's} factors of production, regardless of where located. This is used to estimate the completely national production.

            GNP - GPD = net factor payments(NFP). Factor payments such as profits(by capital), wages(by labour), rent, interest, etc. When NFP < 0, it means that the country receives foreign economic factors. 

        \subsection{Which market price to use}
            \textbf{Nominal GDP} uses the values of goods and services at current prices. 

            \textbf{Real GDP} uses the the values of goods and services at constant years. It is used to observe the real development concentrating on quantity rather than price. The gap of real GDP and Nominal GDP symbolizes inflation. 
        \subsection{Inflation} 
            \textbf{Inflation Rate}: $$\frac{P_t - P_{t - 1}}{P_{t - 1}}$$
            
            How to measure overall price level: \emph{GDP deflator} and \emph{CPI(consumer price index}), PPI, PMI, housing price, etc. 

            The GDP deflator is defined as:
             $$100 \times \frac{Nominal\ GDP}{Real\ GDP}$$
            
             \textbf{GDP delator}: a weighted sum of prices. Every year GDP is contributed by different goods, and the changing baskets of goods are described by \emph{paasche index}

             Consumer Price Index(CPI): tracking changes in the typical household's cost of living. The cost will be calculated among the typical consumer's baskets of goods. The baskets are described by \emph{Laspeyres index}

             $$CPI = 100 \times \frac{cost\ of\ basket\ in\ that\ month}{cost\ of\  basket\ in\ base\ period}$$

            Head Inflation: describe the total inflation from all areas and may experience sudden inflationary spikes like food and energy. 

            Core inflation: exclude certain items that faces volatile price movements, notably food and energy. This is commonly used. 







\end{document}