\documentclass[10pt, a4paper]{article}
    \title{Note} 
    \author{ianaesthetic}
\usepackage{indentfirst, fontspec, amsmath}

\begin{document}
    \section{Measuring GDP}
        \textbf{GDP}:Market value of domestically produced final goods and services. within a year or a quarter.
        \subsection{The Product Approach}
            We will calculate the value of the \emph{final goods and services} excluding the intermediate ones which are used up in the production of other goods and services in the same period that themselves are produced. 
            
            \emph{Capital goods} which are used to produce goods are included. \emph{Inventory investment} which is the amount that inventories of unsold finished goods, goods in process and raw material \emph{have changed} in this period. It's can also be done to add up \emph{value-added}. Tax is another form of adding value on the product from the government. 
            $$
                Value-added = Total\ Revenue - value\ of\ intermediate\ goods.
            $$

        \subsection{The Expenditure Approach}
            It's the standard approach for calculating GPD in most countries. It's consists of several aspects: 

                Consumption: spending by domestic households on final goods and services including Consumer durables, nondurable goods and services. 
                
                Investment: spending for new capital goods (fixed investment) plus inventory investment(the change in the quantity of goods that firms hold in storage, including materials and supplies, work progress and finished goods. \emph{Raw material} is not counted as investment as they will be turned in to product ultimately.
                
                Government purchases of goods and services: Not all government expenditure are purchases of goods and services like interest rate of debt, transfers, and exchange not counted in current goods and services. Some spending is for capital goods to add nation's capital stock.

                Net exports: $Net\ Exports = Exports - Import$

                \begin{align*}
                    GDP &= (C - C_M) + (I - I_M) + (G - G_M) + X \\
                    &= C + I + G + NX
                \end{align*}

        \subsection{The Income Approach}
            Private sector: 
            $$Private\ sector = Y(GDP) + NPF + TR + INT - T$$

            Government sector: 
            $$Government\ sector = T - TR - INT$$
            \begin{align*}
                GDP &= After\text{-}tax\ Wage\ Income \\
                    &+ After\text{-}tax\ Profit \\
                    &+ Interest\ Income\\
                    &+ Taxes\\
            \end{align*}

            Income forms the basis for Expenditure(demand), Disposition of Expenditure determined Production(supply), and Revenue of Supply becomes Income. Three approach will get identical GDP.
        \subsection{Example}
            Example1: Consumption outside of home country.$\Delta C = +6, \Delta NX = -6$
            
            Investment means that the good will generate new revenue, while consumption will not generate new revenue. 

            Inventory investment;
        \subsubsection{GDP and GDP per capita}
            Gini coefficient -- income inequality, the welfare of the economy
            Poverty line -- the number of people below the line to evaluate

            Gross national product(GNP): Total income earned by the \emph{nation's} factors of production, regardless of where located. This is used to estimate the completely national production.

            GNP - GPD = net factor payments(NFP). Factor payments such as profits(by capital), wages(by labour), rent, interest, etc. When NFP < 0, it means that the country receives foreign economic factors. 

        \subsection{Which market price to use}
            \textbf{Nominal GDP} uses the values of goods and services at current prices. 

            \textbf{Real GDP} uses the the values of goods and services at constant years. It is used to observe the real development concentrating on quantity rather than price. The gap of real GDP and Nominal GDP symbolizes inflation. 
        \subsection{Inflation} 
            \textbf{Inflation Rate}: $$\frac{P_t - P_{t - 1}}{P_{t - 1}}$$
            
            How to measure overall price level: \emph{GDP deflator} and \emph{CPI(consumer price index}), PPI, PMI, housing price, etc. 

            \subsubsection{GDP deflator and CPI}
            The GDP deflator is defined as:
             $$100 \times \frac{Nominal\ GDP}{Real\ GDP}$$
            
             \textbf{GDP delator}: a weighted sum of prices. Every year GDP is contributed by different goods, and the changing baskets of goods are described by \emph{paasche index}

             Consumer Price Index(CPI): tracking changes in the typical household's cost of living. The cost will be calculated among the typical consumer's baskets of goods. The baskets are described by \emph{Laspeyres index}

             $$CPI = 100 \times \frac{cost\ of\ basket\ in\ that\ month}{cost\ of\  basket\ in\ base\ period}$$

            Head Inflation: describe the total inflation from all areas and may experience sudden inflationary spikes like food and energy. 

            Core inflation: exclude certain items that faces volatile price movements, notably food and energy. This is commonly used.  CPI is also a weighted sum of prices, while the weights remain fixed as the  baskets are fixed. 

            The overstating of inflation rate by CPI:

                $a$. Situation bias
            
                $b$. Introduction of new goods 
            
                $c$. Quality bias
            \newline 

            Differences between CPI and GPD Deflator:

            $a$. Prices of capital goods (exclude / included)
            
            $b$. Prices of imported consumer goods (included / excluded)

            $c$. The baskets of goods (fixed / changing)

            $d$. Different frequencies (monthly / qtrly)

            \subsubsection{PPI \& PMI \& House Price}
            
            \textbf{Producer Price Index(PPI)}: a weighted index of prices measured at the wholesale, or producer level. It's also based on survey. 

            \textbf{Asset Price Inflation}: The rise of asset price. Inflation often refer to the consumer side. 

            \textbf{Purchasing Managers' Index(PMI)}: is based on data compiled from monthly replies to questionnaires sent to purchasing executives. A reading above 50 indicates an expansion of the sector while blow 50 represents a contraction. 

            There are two different types of PMI. \emph{Official PMI} is based on data collected by the NBS on State-owned companies. \emph{Private PMI} is derived rin a survey of private companies. 
            
        \subsection{Categories of the population}  
            \textbf{Employed}: Working at a paid job 
            
            \textbf{Unemployed}: not employed but looking for a job 
            
            \textbf{Labor forces}: the amount of labor available for producing goods 
            and services; all employed plus unemployed persons. 
            
            \textbf{no in the labor force}: not employed, not looking for work, such as distressed worker. 

            $$Unemployment\ rate = \frac{Number\ unemployed}{Labor\ force}$$
            $$Labor\ force\ participation\ rate = \frac{Labor\ force}{Total\ working\ age\ population}$$
    
    \section{The Labor Market: Productivity, Output, and Employment}
        \subsection{The production function: $Y = A\cdot F(K, L)$}
        \subsubsection{Factors of production}
            $K$ = capitals: tools, machines, and structures used in production.
            
            $L$ = labor: the physical and mental efforts of workers. (also commonly denoted as $N$ in macro) 

            $A$ = other: Total Factor Productivity[TFP](Management, Weather, Policy, Technology, etc)
            
            \textbf{Potential GDP}: the GDP with full-employment output
        \subsubsection{Property}
            \textbf{Slopes Upward}: More of any input produces more output. 
            
            \textbf{Diminishing marginal products}: Slope becomes 
            
            \textbf{Marginal Product of Capital}: The extra output the firm can produce from an additional unit of capital. 
            $$MPK = F(K + 1, L) - F(K, L)  = \frac{\partial F}{\partial K}$$
            
            \textbf{Marginal Product of Labor}: The extra output the firm can produce from an additional unit of labor.  
            $$MPL = F(K, L + 1) - F(K, L) = \frac{\partial F}{\partial L}$$

            MPK and MPL, as the property, are always positive and has a diminishing rate. Their units are in \emph{goods} per unit of labor.  

            \textbf{Diminishing Marginal Returns}: As only one input is increased, its marginal product falls.

            \textbf{Returns to scale}: scale all inputs by the same factor, $Y_1 = F(K_1, L_1)$, $Y_2 = F(zK_1, zK_2)$: 

            {Constant returns to scale}: $Y_2 = zY_1$
            
            {Increasing returns to scale}: $Y_2 > zY_1$

            {Decreasing returns to scale}: $Y_2 < zY_1$
            
            \subsubsection{Cobb-Douglas Production Function} 
            $$Y = AK^{\alpha}L^{1 - \alpha}$$

            Each factor's marginal product is proportional to its average product. 
            $$MPK = \frac{\alpha Y}{K}, \quad MPL = \frac{(1 - \alpha) Y}{L}$$

            $\alpha$ is the capital share of total income, with a constant returns to scale.

            $$Capital\ income = MPK \times K = \alpha Y$$
            $$Labor\ income = MPL \times L = (1 - \alpha)Y$$

            \subsubsection{Supply shock}
                Supply shocks are derived from the influence of supply side without control. It has \emph{positive} shock and \emph{adverse(negative)} shock.

        \subsection{The Demand for Labor}
            Assumptions: Short-run analysis, workers are indifferent, markets are competitive, Profit maximization. The cost is real wage $w = \frac{W}{P}$(P is average product, like CPI, GDP deflator); benefit = MPL; The maximization is $w = MPL$.

            Labor demand curve is exactly the MPL curve as $w = MPL$. So the labor demand curve is downward sloping. 

            The difference of Move Along the Curve and Shift the Curve. Move Along the Curve describe the relationship between the elements along the fixed curve. Shift the Curve is move the curve as some circumstances change, so with a given element another element change.

            Factors that shift the labor demand curve(depend on the factors of MPL): Productivity, Capital stock.
            
            Aggregate demand for labor is the sum of the labor demands of all firms in the economy. 
            
        \subsection{The Supply of Labor}
            The Income-leisure tradeoff: $U(C, L)$ is judged. The income and substitution effects are also included. 

            \textbf{Substitution effect}: Higher real wage encourages work, since reward for working is higher, relying on short-turn.

            \textbf{Income effect}: Higher real wage increases income for same amount of work time, so person can afford more leisure, soo will supply less labor, relying on long-run or permanent change. 
            
            The factor that shift the labor supply curve: Wealth and Expected future real wage. 

            A temporary rise of real wage(Substitution effect) will increases labor supply, while a permanent increase in the real wage will make the labor supply falls. 

            Aggregate supply of labor is the total amount of labor supplied by everyone in the economy.
            Other factors for Aggregate supply are Working-age population and Participation rate. 
        \subsection{Labor market equilibrium}
            Aggregate quantity of labor demanded = aggregate quantity of labor supplied. However wage is sticky and may be above the equilibrium which leads to unemployment. 

            Example of changes: government policy to production / retirement age. 
            
\end{document} 