\documentclass[10pt, a4paper]{article}
\usepackage{indentfirst, amsmath}

\begin{document}
    1. \emph{Assume an economy where there are two producers: a wheat producer and a bread
    producer. In a given year: The wheat producer grows 30 million bushels of wheat, of which 25 million are sold to the bread producer at \$$3$ per bushel and 5 million are stored by the wheat producer to use as seed for next year’s crop. The bread producer produces and sells 100 million loaves of bread to consumers for \$ $3.5$ per loaf.}
    \newline

    \emph{Please calculate the GDP for this economy during this year using the three approaches.} 
    \newline
    
    Product approach: 
    $$Y = 100 \times 3.5 + 5 \times 3 = 365$$
    
    Expenditure approach: 
    \begin{align*}
        Y &= C + I + G + NX \\
          &= 100 \times 3.5 + 5 \times 3 + 0 + 0 \\
          &= 365
    \end{align*}

    Income approach: 

    $$Y = Private\ Sector = 30 \times 3 + 3 \times 100 - 5 \times 3 = 365$$ \newpage

    2. \emph{Based on the data in the table below, explain what happened to output and prices in
    the economy between 2009 and 2010.}

        The nominal GDP increases while the Real GDP decreases, which indicates that the reason for growth is price instead of actual development. The economy does actually suffer from declination.\newline

    3. \emph{Explain why the value of GDP in 2012 would or would not change as a result of each transaction described below}:

        a. \emph{In 2012, the Smith family purchases a new house that was built in 2012.}

        It's a part of GDP as it's a part of Consumption.\newline

        b. \emph{In 2012, the Jones family purchases a house that was built in 2001.}

        Weather it can be part of GDP depends on weather the purchasing create added-value. The house is originally belongs to inventory, and now is counted as consumption. Only if the price to buy this house is more than the value of house in 2001 is the GDP changed.\newline

        c. \emph{In 2012, a construction company purchases windows to put in the Smith family home
        that was built in 2012.}
        
        It's a kind of consumption so it's part of GDP. \newline

        d. \emph{In 2012, Mr. Jones paints all of the rooms of the Jones family house purchased in 2009,
        using paint and supplies purchased in 2012.}

        The whole process is not involved in market so there isn't influence toward GDP. \newline

        e. \emph{In 2012, Mr. Smith uses an online brokerage service to purchases shares of stock in a
        construction company.}
        
        It's not counted in GDP as stock is not a product or service. \newline

    4.  \emph{There are a number of statistics computed to measure the price level, such as the
        GDP deflator and the CPI. The choice of which of these measures to use depends in many
        cases on the specific question in which you are interested. For each of the following
        situations, state whether the CPI or GDP deflator is a more appropriate measure to use and
        explain why the statistic is preferred.}

        a.\emph{ You are interested in looking at the impact of higher prices of imported oil in the
        overall cost of living.}

            The CPI is preferred as GDP deflator doesn't count imported consumer goods like imported oil\newline

        
        b. \emph{The government is interested in whether increases in defense spending are affecting 
        the price level.}

            The GDP deflator is preferred as CPI can not reflect the expense of government. 
            

        c. \emph{An economic consulting firm is investigating the impact on the aggregate price level of
        more computers and electronic technology used in production.}

            The GDP deflator is preferred as the CPI doesn't take the inventory investment of production into consideration. \newline
    
    5. \emph{Assume that apples cost \$$0.50$ in 2002 and \$$1$ in 2009, whereas oranges cost \$$1$ in
    2002 and \$$0.50$ in 2009. If 10 apples and 5 oranges were purchased in 2002, and 5 apples and 10
    oranges were purchased in 2009, what is the CPI for 2009, using 2002 as the base year?}

    $$CPI = \frac{10 \times 1 + 5 \times 0.5}{10 \times 0.5 + 5 \times 1} \times 100\% = 125\%$$\newline

    6. \emph{Assume that the market basket of goods and services purchased in 2004 by the
    average family in the United States costs \$$14,000$ in 2004 prices, whereas the same basket
    costs \$$21,000$ in 2009 prices. However, the basket of goods and services actually purchased
    by the average family in 2009 costs \$$20,000$ in 2009 prices, whereas this same basket would
    have cost \$$15,000$ in 2004 prices. Given this data, please calculate the price index of 2009
    prices using 2004 as the base year under the following cases:}
    
    a. \emph{A Laspeyres price index (method of CPI )}

        $$Index = \frac{15000}{14000} \times 100\% = 107.14\% $$\newline

    b. \emph{A Paasche price index (method of GDP deflator)}

        $$Index = \frac{20000}{15000} \times 100\% = 133.33\% $$

\end{document}