\documentclass[10pt, a4paper]{article}
    \usepackage{indentfirst, amsmath}

\begin{document}

1. \emph{Suppose that the production function is $Y = 9K^{0.5}N^{0.5}$. With this production function,
the marginal product of labor is $MPL=4.5K^{0.5}N^{-0.5}$. The capital stock is K=25. The labor supply
curve is} $LS=100[(1-t)w]^2$, \emph{where} $w$ \emph{is the real wage rate}, $t$ \emph{is the tax return on labor income, and
hence}  $(1-t)w$ \emph{is the after-tax real wage rate.}


(a) \emph{Assume that the tax rate on labor income} $t$, \emph{equals zero. Find the equation of the labor demand
curve. Calculate the equilibrium levels of the real wage and employment, the level of full employment
output, and the total after-tax wage income of workers.}

$$w = MPL = \frac{22.5}{L^{0.5}}, \quad L = 100w^2$$
$$L = 225,\quad w = 1.5, \quad Y = 675$$
$$I_{\text{at}} = L \times (1 - t)w = 337.5$$

(b) \emph{Repeat part (a) under the assumption that the tax rate on labor income, $t$, equals $0.6$}.

$$w = MPL = \frac{22.5}{L^{0.5}}, \quad L = 100[(1 - t)w]^2$$
$$L = 90, \quad w = \frac{3\sqrt{10}}{4}, \quad Y = 135\sqrt{10}$$
$$I_{\text{at}} = L \times (1 - t)w = 27\sqrt{10}$$


2.\emph{ Use the saving-investment diagram to analyze the following effect on national
saving, investment, and the real interest rate. Explain your reasoning.}

\emph{“A large number of accessible oil deposits are discovered, which increases the expected future
marginal product of oil rigs and pipelines. It also causes an increase in expected future income.”}
\smallskip

From the investment perspective, the increase in $MPK^f$ will increase desired capital $K_{t+1}$. As $I_t = K_{t+1} - K_{t} - d$, so there is an increase in investment $I_t$, and this will cause the Investment curve to shift right.

From the saving perspective, the increase in expected future income will lower the household current saving ($S\downarrow$) and increase current consumption($C\uparrow$), which will increase aggregate consumption as well ($C^d\uparrow$). As $S^d = Y_t - C^d - G$, so the aggregate saving will be lower($S^d\downarrow$). This will cause the saving curve to shift left.

As the investment curve shifts right and saving curve shifts left, interest rate will increase.
\newpage

3. \emph{The money supply is \$6,000,000. Currency held by the public is \$2,000,000. The
reserve-deposit ratio is 0.25. Banks do not hold excess reserves. Find deposits, bank reserves, the
monetary base, and the money multiplier.}
$$D = M - C = 4 \times 10^6$$
$$R = D \times rr = 1 \times 10^6$$
$$B = R + C = 3 \times 10^6$$
$$m = \frac{M}{B} = \frac{C + D}{C + R} = 2$$

\medskip

4. \emph{An economy has a monetary base of 1000 \$ 1 bills. Calculate the money supply in scenarios (a) - (d) and then answer part (e)}

(a). \emph{All money is held as currency}
$$M = C + D, \quad B = C + R, \quad R = 0 \quad \Rightarrow \quad M = 1000$$

(b).\emph{All money is held as demand deposits. Banks hold 100 percent of deposits as reserves}
$$B = C + R, \quad C = 0,\quad M = C + D,\quad R = D \times rr\quad \Rightarrow\quad  M = 1000$$

(c).\emph{All money is held as demand deposits. Banks hold 20 percent of deposits as reserves}
$$B = C + R,\quad C = 0$$
$$M = C + D = C + \frac{R}{rr} = 5000$$

(d). \emph{people hold equal amounts of currency and demand deposits. Banks hold 20 percent of deposits as reserves}
$$B = C + R,\quad D = C, \quad R = D \times rr$$
$$M = C + D = 1666.67$$

(e). \emph{The central bank decides to increase the money supply by 10 percent. In each of the for scenarios, how much should it increase the monetary base}

$$m = \frac{M}{B},\quad r = \frac{10\%}{m}$$
$$m_a = 1,\quad r = 10\%$$
$$m_b = 1, \quad r = 10\%$$
$$m_c = 5, \quad r = 2\%$$
$$m_d = \frac{5}{3},\quad r = 6\%$$


\end{document}