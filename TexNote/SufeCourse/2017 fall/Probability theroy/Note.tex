\documentclass[10pt, a4paper]{article}
    \title{Note}
    \author{ianaesthetic}
\usepackage{indentfirst, amsmath, fontspec, listings, xcolor}
\setmainfont{微软雅黑}

\newfontfamily\consolas{Consolas}
\lstset{numbers = left, numberstyle = \small\consolas, basicstyle = \small\consolas}

\XeTeXlinebreaklocale "zh"
\XeTeXlinebreakskip = 0pt plus 1pt

\begin{document}
    \section{}
        可能性为0的时间不一定时不可能事件,设想:在圆内找随机找到任意一个点的概率都为0,但是不是不可能事件,

        给定一个随机试验,对于这个试验的样本空间的描述不止一种。    

    \section{条件概率与事件的独立性}
        定义:设A,B为两个事件,且 $P(A) > )$ 则称

        $$P(B|A) = \frac{P(AB)}{P(A)}$$
        
        为事件A发生条件下B发生的条件概率

        注意$P(AB)$和$P(B|A)$的差别。并且注意这两种情况的不同措辞 。 

        计算方法:
            如果A发生之后,样本空间能够重新很清楚的得出,那么就用古典概型来计算。否则使用公式来进行计算(公式中的所有值都是依照于原来的样本空间来计算的)。条件概率同样适用于概率的性质。

            $$P(AB) = P(A) \cdot P(B|A) = P(B) \cdot P(A|B)$$ 

            上面的式子就是概率的乘法定理,用来计算复杂事件的概率,  下面时推广。   
            $$P(A_1A_2\cdots A_n) = P(A_1)P(A_2|A_1)P(_3|A_1A_2)\cdots P(A_n|A_1\cdots A_{n-1})$$

            波利亚罐子模型:两种球,取球时会加入某数量个同色球。这是一个简化的传染病模型。注意在这个模型中复杂事件中使用惩罚概率。 

            在描述事件的时候要尽量做到周全完善,比如「所取两件中至少有一件是不合格品」而不是「所取的两件中有一件是不合格品」。
            \subsection{事件的独立性}
                当A,B是随机事件的时候,$P(A) > 0$, 
                $$P(B) = P(B|A)$$
                
                则称事件B独立于事件A。

                独立性是相互的,及A独立于B则B也独立与A, 以此为基础定义A和B相互独立,当且仅当以下条件满足:
                    $$P(AB) = P(A) \cdot P(B)$$
                注意当P(A) = 0 的时候,则一定A和B相互独立。判断A B事件是否相互独立,一定要按照定义进行计算。相互独立的本质是一个事件的发生对于另外一个事件发生的概率没有任何影响。

                如果A,B相互独立,则$\hat{A}$与B,A与$\hat{B}$,$\hat{A}$和$\hat{B}$相互独立。

                注意「A B相互独立」和「A B互不相容」的区别:互不相容是集合关系,表示交集为空。相互独立表示的是一个数值相等的概率关系。
                    $$if P(A) > 0, P(B) > 0, AB = \phi$$
                    $$P(AB) = 0, P(A) \cdot P(B) != 0$$ 
                \subsubsection{多个事件独立性}
                    n个事件要相互独立,则要求所有的任意两个事件之间都相互独立。
\end{document}