\documentclass[10pt, a4paper]{article}
    \author{ianaesthetic}
    \title{Unity3D note}
\usepackage{indentfirst, amsmath, fontspec, listings, xcolor}
\setmainfont{微软雅黑}  

\newfontfamily\consolas{Consolas}
\lstset{numberstyle = \small\consolas, basicstyle=\small\consolas}

\XeTeXlinebreaklocale "zh"
\XeTeXlinebreakskip = 0pt plus 1pt

\begin{document}
    \section{(╯‵□′)╯︵┻━┻}
        \subsection{创建数据库并读入}
            在excel中打开数据,从其他源里面选取 Microsoft query。在Query里面选择新的数据源可以从创建新的数据源,并在创建的过程中注意驱动程序的选择和数据库文件。这样就可以创建一个新的数据库。同时也可以从query里面选择需要读入的数据。在具体读入的时候,可以选择具体的要读入的表。 
        \subsection{条件查询}
            不同列之间的条件做并,不同行之间的条件作或。针对日期字段的查询要写为 $\#1996/7/1\#$。其他的数字值可以直接用>,<之类的符号进行筛选。

        \subsection{多表查询}
            \subsubsection{内连接}
                内连接:是将多个表中符合条件的记录挑出来组成一个结果。在内连接查询中,几个表之间一定要有关系。如果没有表表和之间没有关系,需要使用中间过渡的表来使得我们需要读取数据的表相互连接。也可以手动将两个字段关联起来,但是要保证字段中的类型和长度要一致,字段名字可以不同, 相当于同种数据直接根据自身的值的到其他表中进行对应以查到进一步的信息。

                连接的条件和查询之间的关系:每一个连接会有一个等式表示要对应的数据关系,这个等式的意思是:其连接字段值同时出现在两个表中的(比如「2」这个数字都在两个表中的相连接字段中出现了,那么在查询1表的某些值时会出现连接字段值为2的数据)每一个对象会作为查询的总范围。
            
                书上的两个例子:不同表中表示同一个东西的不同名字的 「运货商」,「运货商id」;也可以时表示层级关系的上下级关系对应(具体是其上级栏位中对应了上级的雇员id,这样就可以和另一个复制表中的雇员id建立关系)。但是如果没有寻找到对应关系的字段将不会显示出来。
            \subsubsection{外连接}
                外连接适用于将不符合连接条件的记录一并查询出来。在连接的选项中,可以选择将表中没有列入查询范围的对象可以在此表中查询到(所有的此表对象均显示)。
        \subsection{计算字段}
            Query中也可以使用类似excel的方式进行简单的计算。在计算的时候在选择变量的时候需要具体指定在哪个表的字段。例如:数量 * 订单明细.单价。在excel中计算的时候注意公式需要从字段中拖过来保证能够用(╯‵□′)╯︵┻━┻。
    
    \section{数据分类汇总}
        这个是在excel里面进行操作。我们只用D函数和透视表(0. 0 )
        \subsection{一些汇总操作之前的操作}
        存储在excel表中的数据也可能作为一种数据源的方式,然后里面里面的主表可能是有ID构成然后有字表来解释ID的意义。这时候我们也可以用query然后在新的excel文档中打开。注意选取正确的驱动程序来打开新的数据源。在query里面打开的excel作为的数据源时要添加表需要在「选项」里面打开「系统表」。excel之间的表需要手动建立关系。

        \subsection{数据透视表}
            在「插入」选项中可以插入数据透视表(图)。数据区域不包含字段名,会自动忽略然后将他们加入到字段列表里面。由「报表筛选」「列标签」「行标签」「数值」来管理各种的标签和最终汇总表的样式。透视表完成之后在光标放在数据透视表中时可以在「数据透视工具」里面可以做存储数据透视图。也可以直接按Alt + D + P启动数据透视表创建向导然后直接从外部数据源导入数据建立数据透视图。

            将字段放在「报表筛选」之中表示下面表格中出现的项都属于报表筛选中规定的属性。在「行标签」中,最直观显示出来的就是最上面的分类项,其下面的则是对于其父项的更细的分类。

            \subsubsection{按月汇总}
            随便点击一个在数据透视表当中的一个单元格,右键「创建组」就就可以对于不同的条件进行汇总。得到数据透视图的折线图时,点击图右键可以使用「趋势线」进行预测。在「预测线」界面可以选择显示方程和R值。R值处于[0, 1),越靠近1越贴合。由于结束的月份只包含很少的天数,所以我们不包括最后结束的月份以保证预测的正确性。 在标签里面可以选择不显示隐藏的月份。 对于月份的缺失(当月数据为0会不出现在透视表当中 ),可能需要插入这些月份。在字段列表里面点击下面区域中的相应字段右键「字段设置」,在布局和打印中恢复无数据项。对于无数据项进行值修改,右键此字段点击「数据透视表选项.布局 」,对于空单元格显示0,注意设置为数字0而非字符0。在数据透视表透视里面可以点击设计可以更高改显示放法(比如表格形式)。在将需要的数据提取出来的时候,可以在第一个位置指定对应的位置,然后将框拖下去就可以得到下面的对应的数据的复制。 复制的时候注意单元格的格式设置。
            \subsubsection{对于销售次数的频率分布的汇总}
            首先得到一个含有订单时间,销量和产品名称的汇总表。将数量分别拖到行标签和数量中表示「销售数量为x的销售有多少次」。注意到在最开始的默认值字段设置里面指汇总方式的求和表示的是「销售量的求和」,是对于字段的数值的求和。那么对于销售量为2的产品进行统计的时候就会得到要求答案的2倍。将值字段设置为「计数」就是数值出现的次数的求和,满足要求。要得到每一种销量占所有销量的比,在值字段中「值显示方式」修改为列汇总百分比。这里可以指导一个大包装里面装入的最优的数量。
            \subsubsection{计算百分比}
            目标;统计某一个范围内数据的数目占比。在query里面直接对于第一关键字(客户)和第二关键字(订单id)相同的订单直接进行合并(对于合并字段右键求和)。excel对于列进行排序(右键或者上面的「栏位」之中)。对于所选的一块创建组就是直接拉一个组(区别与点一个然后创建组)。

            在得出了柱形图之后,在同时有两种不同单位的数据的时候,可以使用副轴来作为一种单位的轴。可以通过鼠标来电或者用键盘上下去选择比教不容易用鼠标选中的图。右键或者在上面数据透视工具「数据系列格式」「系列选项」中可以设置在副轴上。通过样在系列选项里面可以设计柱形图的间隔。对于坐标轴的显示范围和格式也可以右键「设置坐标轴格式」「坐标轴选项」里面进行设置

            用散点图可以使选中的第一列为x坐标,其余的列都是y轴。
\end{document}