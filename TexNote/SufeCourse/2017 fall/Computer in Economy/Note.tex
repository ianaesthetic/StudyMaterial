\documentclass[10pt, a4paper]{article}
    \author{ianaesthetic}
    \title{Unity3D note}
\usepackage{indentfirst, amsmath, fontspec, listings, xcolor}
\setmainfont{微软雅黑}  

\newfontfamily\consolas{Consolas}
\lstset{numberstyle = \small\consolas, basicstyle=\small\consolas}

\XeTeXlinebreaklocale "zh"
\XeTeXlinebreakskip = 0pt plus 1pt

\begin{document}
    \section{(╯‵□′)╯︵┻━┻}
        \subsection{创建数据库并读入}
            在excel中打开数据,从其他源里面选取 Microsoft query。在Query里面选择新的数据源可以从创建新的数据源,并在创建的过程中注意驱动程序的选择和数据库文件。这样就可以创建一个新的数据库。同时也可以从query里面选择需要读入的数据。在具体读入的时候,可以选择具体的要读入的表。 
        \subsection{条件查询}
            不同列之间的条件做并,不同行之间的条件作或。针对日期字段的查询要写为 $\#1996/7/1\#$。其他的数字值可以直接用>,<之类的符号进行筛选。

        \subsection{多表查询}
            \subsubsection{内连接}
                内连接:是将多个表中符合条件的记录挑出来组成一个结果。在内连接查询中,几个表之间一定要有关系。如果没有表表和之间没有关系,需要使用中间过渡的表来使得我们需要读取数据的表相互连接。也可以手动将两个字段关联起来,但是要保证字段中的类型和长度要一致,字段名字可以不同, 相当于同种数据直接根据自身的值的到其他表中进行对应以查到进一步的信息。

                连接的条件和查询之间的关系:每一个连接会有一个等式表示要对应的数据关系,这个等式的意思是:其连接字段值同时出现在两个表中的(比如「2」这个数字都在两个表中的相连接字段中出现了,那么在查询1表的某些值时会出现连接字段值为2的数据)每一个对象会作为查询的总范围。
            
                书上的两个例子:不同表中表示同一个东西的不同名字的 「运货商」,「运货商id」;也可以时表示层级关系的上下级关系对应(具体是其上级栏位中对应了上级的雇员id,这样就可以和另一个复制表中的雇员id建立关系)。但是如果没有寻找到对应关系的字段将不会显示出来。
            \subsubsection{外连接}
                外连接适用于将不符合连接条件的记录一并查询出来。在连接的选项中,可以选择将表中没有列入查询范围的对象可以在此表中查询到(所有的此表对象均显示)。
        \subsection{计算字段}
            Query中也可以使用类似excel的方式进行简单的计算。在计算的时候在选择变量的时候需要具体指定在哪个表的字段。例如:数量 * 订单明细.单价。在excel中计算的时候注意公式需要从字段中拖过来保证能够用(╯‵□′)╯︵┻━┻。
    
    \section{数据分类汇总}
        这个是在excel里面进行操作。我们只用D函数和透视表(0. 0 )
        \subsection{一些汇总操作之前的操作}
        存储在excel表中的数据也可能作为一种数据源的方式,然后里面里面的主表可能是有ID构成然后有字表来解释ID的意义。这时候我们也可以用query然后在新的excel文档中打开。注意选取正确的驱动程序来打开新的数据源。在query里面打开的excel作为的数据源时要添加表需要在「选项」里面打开「系统表」。excel之间的表需要手动建立关系。

        \subsection{数据透视表}
            在「插入」选项中可以插入数据透视表(图)。数据区域不包含字段名,会自动忽略然后将他们加入到字段列表里面。由「报表筛选」「列标签」「行标签」「数值」来管理各种的标签和最终汇总表的样式。透视表完成之后在光标放在数据透视表中时可以在「数据透视工具」里面可以做存储数据透视图。也可以直接按Alt + D + P启动数据透视表创建向导然后直接从外部数据源导入数据建立数据透视图。

\end{document}