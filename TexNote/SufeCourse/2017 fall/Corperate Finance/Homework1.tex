\documentclass[10pt, a4paper]{article}
\usepackage{indentfirst, amssymb}

\begin{document}
1(a).\emph{The stated annual interest rate is 10\% per year. Find the effective annual interest rate when the stated annual rate is with monthly compounding}.
$$EAR = (1 + \frac{APR}{m})^m - 1 = 10.47\%$$

1(b). \emph{Consider two growing annuities each with 20 annual payments. The first growing annuity starts a year from today at \$10 and grows at a rate o  f 5\% per year. The second growing annuity starts paying \$C five years from today and payments decline at a rate of 5\% per year. If the effective annual interest rate is 10\% per year, what is the value of C such that you are indifferent between these two growing annuities?}

$$PV_1 = \frac{C_1}{r - g_1}(1 - (\frac{1 + g_1}{1 + r})^{20})$$
$$PV_2 = (\frac{1}{1 + r})^5\frac{C}{r - g_2}(1 - (\frac{1 + g_2}{1 + r})^{20})$$
$$C = 30.9$$

1(c).\emph{Suppose the effective annual interest rate is 12\%/year. A security offers the following 36 cash flows: the first 12 cash flows are monthly payments of 
\$50 each, with the first cash flow starting at 3 years from today; the next 24 cash flows are quarterly payments of \$200 each, with the first one starting at 4 years from today. 
Compute the present value of these 36 cash flows.}


$$PV_1 = (\frac{1}{1 + r}) ^ 3 (\frac{C_1}{r}(1 - (\frac{1}{1 + r})^{\frac{m_1}{12}})$$
$$PV_2 = (\frac{1}{1 + r}) ^ 4 (\frac{C_2}{r}(1 - (\frac{1}{1 + r})^{\frac{m_2}{4}})$$
$$PV = PV_1 + PV_2 = 554.351$$

2(a).\emph{What are the payoffs of a portfolio of one share of security A and one share of security B? What is the market price of this portfolio? What expected return will you earn from holding this portfolio}

Payoff: as the combination of this portfolio is risk-free, the payoff is 600
$$P(A + B) = P(A) + P(B) = 577$$
$$r = \frac{\rm{payoff}}{P(A + B)} - 1 = 0.398$$

2(b).\emph{ Suppose security C has a payoff of \$600 when the economy is weak and \$1800 when the economy is strong. The risk-free interest rate is 4\%. What is the no-arbitrage price of security C?}
$$P(C) = 3P(A) + P(B) = 1039$$

2(c).\emph{What is the expected return of security C if both states are equally likely? What is its risk premium?}
$$E = 600 \times \frac{1}{2} + 1800 \times \frac{1}{2} = 1200$$
$$r_s = \frac{E}{P(C)} - 1 = 15.50\%$$
$$r_m = r_s - r_f = 11.50\%$$

2(d).\emph{If security C had a risk premium of 10\%, what arbitrage opportunity would be available}\

As $10\% < 11.50\%$, we can sell security C and buy 3A + B bundle and the left money will become deposit to have additional interest. 

\medskip


3.\emph{You are thinking of making an investment in a new plant. The plant will generate revenues of \$1 million per year for as long as you maintain it. You expect that the maintenance cost will start at \$50,000 per year and will increase 5\% per year thereafter. Assume that all revenue and maintenance costs occur at the end of the year. You intend to run the plant as long as it continues to make a positive cash flow (as long as the cash generated by the plant exceeds the maintenance costs). The plant can be built and become operational immediately. If the plant costs \$10 million to build, and the interest rate is 6\% per year, should you invest in the plant?}
$$50000 \times (1 + g) ^ n \leqslant 1000000,\quad n \leqslant 61$$
$$PV = \frac{C}{r}(1 - (\frac{1}{1 + r})^n) = 1.62 \times 10^7 > 10^7$$
So I will invest.
\medskip 

4(a). \emph{You are saving for retirement. To live comfortably, you decide you will need to save \$2 million by the time you are 65. Today is your 30th birthday, and you decide, starting today and continuing on every birthday up to and including your 65th birthday, that you will put the same amount into a savings account. If the interest rate is 5\%, how much must you set aside each year to make sure that you will have \$2 million in the account on your 65th birthday?}

$$C + \frac{C}{r}(1 - (\frac{1}{1 + r})^{35}) = 2\times 10^6 * (\frac{1}{1 + r})^{35}$$
$$C = 20868.9$$

4(b).\emph{You realize that the plan in part(a) has a flaw. Because your income will increase over your lifetime, it would be more realistic to save less now and more later. Instead of putting the same amount aside each year, you decide to let the amount that you set aside grow by 3\% per year. Under this plan, how much will you put into the account today? (Recall that you are planning to make the first contribution to the account today.)}
$$(1 + g)(\frac{C}{1 + g} + \frac{C}{r - g}(1 - (\frac{1 + g}{1 + r})^{35})) = 2 \times 10^6 * (\frac{1}{1 + r})^{35}$$
$$C = 13823.9$$
\medskip

5.\emph{The mortgage on your house is five years old. It required monthly payments of \$1402, had an original term of 30 years, and had an interest rate of 10\% (APR). In the intervening five years, interest rates have fallen and so you have decided to refinance—that is, you will roll over the outstanding balance into a new mortgage. The new mortgage has a 30-year term, requires monthly payments, and has an interest rate of 6 5⁄8\% (APR)}

5(a).\emph{What monthly repayments will be required with the new loan?}

$$PV = \frac{C_1}{r_1}(1 - (\frac{1}{1 + r_1})^{25 \times 12})$$
$$PV = \frac{C_2}{r_2}(1 - (\frac{1}{1 + r_2})^{30 \times 12})$$
$$C_2 = 987$$

5(b).\emph{If you still want to pay off the mortgage in 25 years, what monthly payment should you make after you refinance?}

$$PV = \frac{C_1}{r_1}(1 - (\frac{1}{1 + r_1})^{25 \times 12})$$
$$PV = \frac{C_2}{r_2}(1 - (\frac{1}{1 + r_2})^{25 \times 12})$$
$$C_2 = 1053.83$$

5(c).\emph{Suppose you are willing to continue making monthly payments of \$1402. How long will it take you to pay off the mortgage after refinancing?}
$$PV = \frac{C_1}{r_1}(1 - (\frac{1}{1 + r_1})^{25 \times 12})$$
$$PV = \frac{C_1}{r_2}(1 - (\frac{1}{1 + r_2})^{n})$$
$$\frac{n}{12} = 14.16$$

5(d).\emph{Suppose you are willing to continue making monthly payments of \$1402, and want to pay off the mortgage in 25 years. How much additional cash can you borrow today as part of the refinancing?}

$$PV_1 = \frac{C_1}{r_1}(1 - (\frac{1}{1 + r_1})^{25 \times 12})$$
$$PV_2 = \frac{C_1}{r_2}(1 - (\frac{1}{1 + r_2})^{25 \times 12})$$
$$\Delta PV = PV_2 - PV_1 = 50973$$




\end{document}
