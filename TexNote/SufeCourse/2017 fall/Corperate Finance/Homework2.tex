\documentclass[10pt, a4paper]{article}
    \author{ianaesthetic}
    \title{Unity3D note}
\usepackage{indentfirst, amsmath, fontspec, listings, xcolor}

%cSpell: ignore makecell

\usepackage{makecell}

\newfontfamily\consolas{Consolas}
\lstset{numbers = left, numberstyle = \small\consolas, basicstyle=\small\consolas}

\XeTeXlinebreaklocale "zh"
\XeTeXlinebreakskip = 0pt plus 1pt

\begin{document}
1.\emph{You are considering a project that involves buying a machine to launch a new line of garden products. The machine costs \$210,000. The project has a time horizon of three years. This machine would allow you to use some spare parts you already own. If the project is not taken, the spare parts can be sold for an after-tax amount of \$10,000. You paid \$5,000 to a marketing firm to estimate the demand for this garden product. With these estimates, you plan to sell 15,000 units during the first year at \$20/unit. The number of units sold will increase by 2\% per year but the price will stay the same. For the first year, each unit requires \$10 of raw material and \$2 of labor. You estimate the cost of the raw material to increase by 2\% per year and the cost of labor by 5\% per year. Your fixed cost for this machine is \$30,000 per year. Finally, you will need an increase in net working capital of \$20,000 at the beginning of the project, but you will recover it at the end of the project. You use straight-line depreciation for the machine and expect to sell it for \$80,000 at the end of the project. The marginal corporate tax rate is 35\%, and the discount rate for the project’s cash flows is 10\%. For simplicity, assume revenues and expenses are received and paid at the end of each year. Compute the net present value of the project}

\bigskip
\begin{center}
\begin{tabular}{|r|r|r|r|r|} 
    \hline
    \makecell{} & \makecell{0} & \makecell{1} & \makecell{2} & \makecell{3} \\
    \hline
    \makecell{Revenues} & \makecell{} & \makecell{300000} & \makecell{306000} & \makecell{312120} \\
    \hline 
    \makecell{Costs} & \makecell{} & \makecell{-210000} & \makecell{-218190} &\makecell{-226776.05} \\
    \hline
    \makecell{Tax} & \makecell{} & \makecell{-7000} & \makecell{-6233.5} & \makecell{-5370.38} \\
    \hline
    \makecell{Capital Expenses} & \makecell{-210000} & \makecell{} & \makecell{} & \makecell{52000} \\
    \hline
    \makecell{$\Delta$ NWC} & \makecell{-20000} & \makecell{} & \makecell{} & \makecell{20000} \\
    \hline
    \makecell{Cash Flow} & \makecell{-240000} & \makecell{83000} & \makecell{81576.5} & \makecell{151973.5649} \\
    \hline
    \makecell{Discounted CF} & \makecell{-240000} & \makecell{75454.55} & \makecell{67418.60} & \makecell{114179.99} \\
    \hline
\end{tabular}
\end{center}
$$NPV = 17053.13$$

\newpage
\emph{
2(a). What is the price today of a two-year, default-free security with a face value of \$1000 and an annual coupon rate of 6\%? Does this bond trade at a discount, at par, or at a premium?
}
$$PV = \frac{1000 * 0.06}{1 + 0.04} + \frac{1000 * 1.06}{1.04^2} = 1037.72$$
\begin{center}
It's sold at premium;
\end{center}


\emph{
2(b). Consider a four-year, default-free security with annual coupon payments and a face value of \$1000 that is issued at par. What is the coupon rate of this bond?
}

$$1000 = \sum_{i = 1}^{3}\frac{C}{(1 + 0.04)^i} + \frac{1000 + C}{1.04^4}$$
$$C = 40, r_C = 0.04$$

3.\emph{You have just started your summer internship, and your boss asks you to review a recent analysis that was done to compare three alternative proposals to enhance the firm’s manufacturing facility. You find that the prior analysis ranked the proposals according to their IRR, and recommended the highest IRR option, Proposal A. You are concerned and decide to redo the analysis using NPV to determine whether this recommendation was appropriate. But while you are confident the IRRs were computed correctly, it seems that some of the underlying data regarding the cash flows that were estimated for each proposal was not included in the report. For Proposal B, you cannot find information regarding the total initial investment that was required in year 0. And for Proposal C, you cannot find the data regarding additional salvage value that will be recovered in year 3. Here is the information you have:}

\medskip

\begin{tabular}{|r|r|r|r|r|r|r|}
    \hline
    \makecell{Proposal} & \makecell{IRRs} & \makecell{Year 0} & \makecell {Year 1} & \makecell{Year 2} & \makecell{Year 3} & \makecell{NPV}\\
    \hline
    \makecell{A} & \makecell{0.6} & \makecell{-100} & \makecell{30} &\makecell{153} & \makecell{88} & \makecell{119.835} \\
    \hline
    \makecell{B} & \makecell{0.55} & \makecell{-111.255} & \makecell{0} &\makecell{206} & \makecell{95} & \makecell{130.368} \\
    \hline
    \makecell{C} & \makecell{0.5} & \makecell{-100} & \makecell{37} &\makecell{0} & \makecell{254.25} & \makecell{124.658} \\
    \hline
\end{tabular}

\medskip
    We shall choose B 

\newpage

4.\emph{LeXing Ltd is a manufacturer of data modems. It has recently hired a marketing consultant at a cost of \$200,000 to carry out a feasibility study on a new 4-year project to manufacture USB-Modems for the export market. LeXing owns an industrial property that was bought 3 years ago at a cost of \$1.2 million for investment purpose, and its original plan was to sell the property at the current market value of \$1.3 million on an after-tax basis. Since the new project will be occupying the premises, the company now plans to sell the property only at the end of 4 years, but at an estimated value of \$1.4 million on an after-tax basis. The new project also requires \$400,000 investment in fixed assets, which have an estimated market value of \$60,000 at the end of the project. The assets will be depreciated fully on a straight-line basis over 4 years. Annual sales generated from this project are estimated as follows:}

\emph{Year 1 sales: \$2,000,000}

\emph{Year 2 sales: \$2,500,000}

\emph{Year 3 sales: \$3,000,000}

\emph{Year 4 sales: \$4,000,000}

\emph{Total annual operating costs (not including depreciation) are estimated to be 60\% of sales. Due to this new project, the company’s current assets will increase by \$100,000 and accounts payable will also increase by \$40,000. The increase in net working capital is expected to be completely recovered at the end of the project. The project will also cause its after-tax cash flows from its desktop modem products to decrease by \$30,000 per year. LeXing’s marginal corporate tax rate is 30\% and the cost of capital for this project is estimated to be 12\%.}

\emph{Compute the incremental annual cash flows and NPV to evaluate this new project.
}


\begin{center}
\begin{tabular} {|r|r|r|r|r|r|}
    \hline
    \makecell{} & \makecell{0} & \makecell{1} & \makecell{2} & \makecell{3} & \makecell{4}\\
    \hline
    \makecell{Revenues} & \makecell{} & \makecell{2000000} & \makecell{2500000} & \makecell{3000000} & \makecell{4000000}\\
    \hline
    \makecell{Costs} & \makecell{} & \makecell{-1200000} & \makecell{-1500000} & \makecell{-1800000} & \makecell{-2400000}\\
    \hline
    \makecell{Taxes} & \makecell{} & \makecell{-210000} & \makecell{-270000} & \makecell{-330000} & \makecell{-450000}\\
    \hline
    \makecell{Externality} & \makecell{} & \makecell{-30000} & \makecell{-30000} & \makecell{-30000} & \makecell{-30000}\\
    \hline
    \makecell{Property} & \makecell{-13000000} & \makecell{} & \makecell{} & \makecell{} & \makecell{14000000}\\
    \hline
    \makecell{CapExpense} & \makecell{-400000} & \makecell{} & \makecell{} & \makecell{} & \makecell{42000}\\
    \hline
    \makecell{$\Delta$NWC} & \makecell{-60000} & \makecell{} & \makecell{} & \makecell{} & \makecell{60000}\\
    \hline
    \makecell{Cash Flow} & \makecell{-1760000} & \makecell{560000} & \makecell{700000} & \makecell{840000} & \makecell{2622000}\\
    \hline
    \makecell{Discounted CF} & \makecell{-1760000} & \makecell{500000} & \makecell{558035.71} & \makecell{597895.41} & \makecell{1666328.4}\\
    \hline
\end{tabular}
\end{center}
$$NPV = 1562259.52$$
\end{document}