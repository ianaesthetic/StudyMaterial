\documentclass[10pt, a4paper]{article}
    \author{ianaesthetic}
    \title{Unity3D note}
\usepackage{indentfirst, amsmath, fontspec, listings, xcolor}

\newfontfamily\consolas{Consolas}
\lstset{numbers = left, numberstyle = \small\consolas, basicstyle=\small\consolas}

\XeTeXlinebreaklocale "zh"
\XeTeXlinebreakskip = 0pt plus 1pt

\begin{document}
    \section{Introduction to Corporate Finance}
        \subsection{Financial Management Decisions}
            Cash first raised from investor, invested in firm, generated by operations, and finally reinvested or returned to inventers.
        \subsection{Corporation}
            \textbf{Definition}: from CF perspective, a firm is a collection of projects. Projects are anything that can generate cash. It's also a business form. There are different types of business forms including: \emph{Sole Proprietorship}, \emph{Partnership (General and Limited-Liability)}, \emph{Limited-Liability Company}, \emph{Corporation}.

            Disadvantage of Sole and Partnership: 

                (a: Unlimited Liability
                (b: difficult to raise money 
                (c: difficult to transfer. 
            
            Differences between of Limited-Liability Company and Corporation: whether it has gone public. 
            
            Corporation is consists of \emph{board of directors} who manage the \emph{asset}, \emph{debt} which is concerned with \emph{debt holders},  \emph{equity} which is concerned with \emph{share holders}.

        \subsection{Goal of Finance Management}
            \textbf{Primary} financial goal is \emph{shareholder wealth maximization}, which can be translated to maximizing \emph{stock price}. 
            
            Stock price maximization is not same as profit maximization: stock price relies upon \emph{current earnings}, \emph{future earnings} and \emph{cash flow}. They may not be change in the same way. In particular, factors that determine stock price are: 

            $a$. Projected cash flow to share holders
            
            $b$. Timing of the cash flow stream

            $c$. Rickness of the cash flow

        \subsection{Agency problem}
            Share holders and Managers: This is the most important agency problem. Managers are inclined to act in their own best interest. 
            
            Share holders Vs Creditors: Deriving from \emph{new investment opportunities, dividend versus retained earnings}.

            Creditors vs Managers: Financing decision; Senior versus Junior bond
        \subsection{Financial Market}
            A market is a venue where goods and services are exchanged. 

            A financial market is a place where individuals and organizations to raise capital with \emph{investor} and \emph{borrower}.

            Primary market: IPO market, usually bankers involved. For individuals, they have limited access limited by original holding and fortune. This the capital the company raised. 

            Secondary market: The trade is between investors involving buying and selling stocks. 

            \noindent Process:

                $a$. Firm issues securities to raise cash.
                
                $b$. Firm invests in assets.
                
                $c$. Firm's operations generate cash flows.

                $d$. Cash is paid to government as \emph{taxes}. \emph{Other stakeholder} may receive cash. 

                $e$. Reinvested cash flows are plowed back into firm
                
                $d$. Cash is paid out to investors in the form of interest and dividends.
    
    \section{Review of Financial Statement Analysis}
        \subsection{Balance Sheet}
            \textbf{Definition}: a snapshot of the firm's asset and liabilities at a given point of time, indicating all operations during the time point. 
            $$Assets = Liability + Stockholder's\ Equity$$

            Assets are listed in order to liquidity(Current asset > Fixed Asset[Tangible and Intangible]). Liquidity stands for ease of cash conversion without significant loss.

            Current asset includes \emph{Cash and equivalents}, \emph{Accounts receivable}, \emph{Inventories}. Fixed asset includes \emph{Property, plant and equipment(PPE)}, \emph{Less accumulated depreciation}, Intangible assets and other. Amotization;
            \begin{align*}
                Total\ Current\ Assets &= Cash\ and\ Equivalents\\
                                       &+ Accounts\ Receivable\\
                                       &+ Inventories  
            \end{align*}
            \begin{align*}
                Total\ Fixed\ Assets &= Net Property,\ Plant,\ equipment\ (PPE) \\
                                     &+ Intangible\ Assets\ and\ Other        
            \end{align*}
            $$Net\ PPE = PPE - Less\ Accumulated\ Depreciation$$

            Current Liabilities includes \emph{Accounts payable}, Notes payable, Accrued expenses. Long term Liabilities includes deferred taxes, \emph{long-term debt}. Stockholder's equity: Preferred stock, Common stock, capital surplus, \emph{Accumulated retained earnings}(which links balance sheet and income statement).
            
            $$Current\ Liability = Accounts\ Payable + Note\ Payable$$
            $$Long\text{-}term Liability = Deferred\ Taxes + Long\text{-}Term Debt$$
            \begin{align*}
                Stockholder's\ equity &= Preferred\ Stock \\
                                      &+ Common\ Stock\\
                                      &+ Capital\ Surplus \\ 
                                      &+ Accumulated Retained Earnings
            \end{align*}
        
        \subsubsection{Market Vs. Book Value}
            $$Market\ Value = P \times N \quad\quad \text{Can't be negative}$$ 
            $$Bool\ Value = A - L \quad\quad \text{Can be negative}$$

            Market value means the value of equity can be sold in the market. Book values are calculated in the balance sheet as historical price. Market values matter more than book value according to the goal of financial management. The market value will be affected be market by price, eg. increasing value of raw material, to be different with book value. 

            $$Net\ Working\ Capital = Current\ Assets - Current\ Liabilities$$

            NWC usually grows with the firm. We can use the change in NWC to estimate the growth situation.

    \subsection{Income Statement}
        \textbf{Definition}: It's more like a video of the firm's operations for a specified period of time. 
        $$Income = Revenue - Expenses$$

        \textbf{Operation section} of the income statement reports the firm's revenue and expenses from \emph{principle operations} including: Total operating revenues; Cost of goods sold; Selling, general and administrative expenses; Depreciation(allocate the value of Inventory) and Operating income.
        \begin{align*}
            Operating\ Income &= Total\ Operating\ revenues \\   
                              &- Costs\ of\ Goods\ Sold \\
                              &- Selling,\ General,\ Administrative\ Expense\\
                              &- Depreciation \\
        \end{align*}

        \textbf{Non-operation section} of the income statement includes all \emph{financing costs}, such as interest expenses including Other income , Earnings before interest and taxes(EBIT); Interest expense and Pretax income. Debt will have interest without tax, which means interest expenses can reduce the tax, the phenomena of \emph{tax shield}. Interest expense should be dealt first as they are not a part of tax. 

        Net Income will be divided into \emph{Retained Earnings} and \emph{Dividends}
        
    \subsection{Cash Flow} 
        \subsubsection{Accounting Perspective}
        \textbf{Cash Flow} is one of the most important pieces of information that a financial manager can derive from financial statements. There is an official accounting statement called the statement of cash flow to identify inflow and outflow. In accounting perspective, what we concern the cash actually earned. 

        There are three types of cash flows 

        $a$. Cash flow from operating activities. 

        $b$. Cash flow from investing activities (Asset). 

        $c$. Cash flow from financing activities (Stock). 

        Accounts payable belongs to operating while Notes payable belongs to financial activity for note are financial instrument. 
        
        \subsubsection{Financial Perspective}
        From the financial perspective, we concern the capability to generate total cash flow. (total / free) Cash Flow from Assets(CFFA) of the firm:   
        \begin{align*}
            CFFA = Cash\ Flow\ to\ Creditors + Cash\ Flow\ to\ Shareholders
        \end{align*}
        \begin{align*}
            CFFA &= Operating\ Cash\ Flow\ (OFC)\\
                 &+ Net\ Capital\ Spending\ (NCS)\\
                 &+ Change\ in\ Net\ Working\ Capital\ (\Delta NWC) 
        \end{align*}
        $$OCF = Earning\ before\ Interest\ and\ Taxes + Depreciation - Taxes$$
        $$NCS = Ending\ Net\ Fixed\ Assets - Beginning\ Net\ Fixed\ Assets +  Depreciation$$
        $$\Delta NWC = Ending\ NWC - Beginning\ NWC$$
        
        $$Cash\ Flow\ to\ Creditors = Interest\ Paid - Net\ New\ Borrowing$$
        $$Cash\ Flow\ to\ Shareholders = Dividends\ Paid - Net\ New\ Equity\ Raised$$

        $\Delta NWC$ can be negative, for example you decrease the 


        Income does not equal to the amount of cash the firm has earned:

            $a$. Non-Cash Expenses
        
            $b$. Uses of Cash not on the Income Statement such as investment in EBIT
    
    \subsection{Categories of Financial Ratios}
        Ratios are main instrument to analyze. The formula is not actually required as they maybe given in exam. It's the analysis that is important. 
        
        \subsubsection{Short-term solvency or liquidity}
        Current Ratio is used as the most important way to measure debt payment in short run.
        $$Current\ Ratio = \frac{Current\ Asset(CA)}{Current\ Liability(CL)}$$

        Quick Ratio is to measure the ability to liquidate at once to repay debt.
        $$Quick\ Ratio = \frac{CA - Inventory}{CL}$$
        
        Cash Ratio is to measure the ability to use money 
        $$Cash\ Ratio = \frac{Cash}{CL}$$

        \subsubsection{Long-term  solvency or financial leverage ratios}
        $$Total\ Debt\ Ratio = \frac{Total\ Asset - Total\ Equity}{Total\ Asset}$$
        $$\frac{Debt}{Equity} = \frac{Total\ Debt}{Total\ Equity}$$
        $$Equity\ Multiplier = \frac{Total\ Asset}{Total\ Equity} = 1 + \frac{Debt}{Equity}$$

        The bigger of $EM$, the ratio of the equity is smaller, meaning more debt and crisis potential. 
        \subsubsection{Coverage Ratio}
        Coverage Ratio: How will you can pay the interest. 
        $$Times\ Interest\ Earned = \frac{EBIT}{Interest}$$
        $$Cash\ Coverage = \frac{EBIT + Depreciation + Amortization}{Interest}$$
        The Depreciation and Amortization is the current one, not the accumulated one. 
        
        \subsubsection{Asset Management or Turnover  ratios}
        
        Inventory Ratios is used to measure the speed the inventory turns to sold goods. Day's scale to some extent is to measure the time to store the inventory.
            $$Inventory\ Turnover = \frac{Cost\ of\ Goods\ Sold}{Inventory}$$
            $$Day's\ Sales\ in\ Inventory = \frac{365}{Inventory\ Turnover}$$
        
        Receivable Ratios is used to measure the speed of retrieving accounts receivable.
            $$Receivable Turnover = \frac{Sales}{Accounts\ Receivable}$$
            $$Day's\ Sales\ in\ Receivable = \frac{365}{Receivable\ Turnover}$$

        Total Asset Turnover, reflects long-run efficiency they generate income. 
            $$Total\ Asset\ Turnover(TAT) = \frac{Sales}{Total\ Asset}$$

            It's not unusual for TAT < 1, especially if a firm has a large amount of fixed assets.  
        \subsubsection{Computing Profitability Measures}    
            The most important ratio in measuring a company.

            $$Profit\ Margin = \frac{Net\ Income}{Sales}$$
            $$EBITDA\ Margin = \frac{EBITDA}{Sales}$$
            
            

            $$Return\ on\ Assets\ (ROA) = \frac{Net\ Income}{Total\ Asset}$$
            $$Return\ on\ Equity\ (ROE) = \frac{Net\ Income}{Total\ Equity}$$
        \subsubsection{Computing Market Value Measures}
            $$Market\ Capitalization = Share\ Price \times Share\ Outstanding$$
            $$EPS = \frac{Net\ Income}{Shares\ Outstanding}$$
            $$PE\ Ratio = \frac{Price\ per\ Share}{Earnings\ per\ Share}$$
            $$Market-to-Book Ratio = \frac{Market\ Value\ per\ Share}{Book\ value\ per\ Share}$$
            \begin{align*}
                Enterprise\ Value\ (EV) &= Market\ Capitalization \\
                                        &+ Market\ Value\ of\ Interest\ Bearing\ Debt \\
                                        &- Cash
            \end{align*}
            $$EV\ Multiple = \frac{EV}{EBITDA}$$
        

\end{document}