\documentclass[10pt, a4paper]{article}
    \author{ianaesthetic}
    \title{Unity3D note}
\usepackage{indentfirst, amsmath, fontspec, listings, xcolor}

\newfontfamily\consolas{Consolas}
\lstset{numbers = left, numberstyle = \small\consolas, basicstyle=\small\consolas}

\XeTeXlinebreaklocale "zh"
\XeTeXlinebreakskip = 0pt plus 1pt

\begin{document}
    \section{Introduction to Corporate Finance}
        \subsection{Financial Management Decisions}
            Cash first raised from investor, invested in firm, generated by operations, and finally reinvested or returned to inventers.
        \subsection{Corporation}
            \textbf{Definition}: from CF perspective, a firm is a collection of projects. Projects are anything that can generate cash. It's also a business form. There are different types of business forms including: \emph{Sole Proprietorship}, \emph{Partnership (General and Limited-Liability)}, \emph{Limited-Liability Company}, \emph{Corporation}.

            Disadvantage of Sole and Partnership: 

                (a: Unlimited Liability
                (b: difficult to raise money 
                (c: difficult to transfer. 
            
            Differences between of Limited-Liability Company and Corporation: whether it has gone public. 
            
            Corporation is consists of \emph{board of directors} who manage the \emph{asset}, \emph{debt} which is concerned with \emph{debt holders},  \emph{equity} which is concerned with \emph{share holders}.

        \subsection{Goal of Finance Management}
            \textbf{Primary} financial goal is \emph{shareholder wealth maximization}, which can be translated to maximizing \emph{stock price}. 
            
            Stock price maximization is not same as profit maximization: stock price relies upon \emph{current earnings}, \emph{future earnings} and \emph{cash flow}. They may not be change in the same way. In particular, factors that determine stock price are: 

            $a$. Projected cash flow to share holders
            
            $b$. Timing of the cash flow stream

            $c$. Rickness of the cash flow

        \subsection{Agency problem}
            Share holders and Managers: This is the most important agency problem. Managers are inclined to act in their own best interest. 
            
            Share holders Vs Creditors: Deriving from \emph{new investment opportunities, dividend versus retained earnings}.

            Creditors vs Managers: Financing decision; Senior versus Junior bond
        \subsection{Financial Market}
            A market is a venue where goods and services are exchanged. 

            A financial market is a place where individuals and organizations to raise capital with \emph{investor} and \emph{borrower}.

            Primary market: IPO market, usually bankers involved. For individuals, they have limited access limited by original holding and fortune. This the capital the company raised. 

            Secondary market: The trade is between investors involving buying and selling stocks. 

            \noindent Process:

                $a$. Firm issues securities to raise cash.
                
                $b$. Firm invests in assets.
                
                $c$. Firm's operations generate cash flows.

                $d$. Cash is paid to government as \emph{taxes}. \emph{Other stakeholder} may receive cash. 

                $e$. Reinvested cash flows are plowed back into firm
                
                $d$. Cash is paid out to investors in the form of interest and dividends.
    
    \section{Review of Financial Statement Analysis}
        \subsection{Balance Sheet}
            \textbf{Definition}: a snapshot of the firm's asset and liabilities at a given point of time, indicating all operations during the time point. 
            $$\text{Assets} = \text{Liability + Stockholder's\ Equity}$$

            Assets are listed in order to liquidity(Current asset > Fixed Asset[Tangible and Intangible]). Liquidity stands for ease of cash conversion without significant loss.

            Current asset includes \emph{Cash and equivalents}, \emph{Accounts receivable} (composed of raw materials to be used in production, work in progress, and finished goods), \emph{Inventories}. Fixed asset includes \emph{Property, plant and equipment(PPE)}, \emph{Less accumulated depreciation}, Intangible assets and other  (Amortization);
            \begin{align*}
                \text{Total\ Current\ Asset}s &= \text{Cash\ and\ Equivalents}\\
                                       &+ \text{Accounts\ Receivable}\\
                                       &+ \text{Inventories}  
            \end{align*}
            \begin{align*}
                \text{Total\ Fixed\ Assets} &= \text{PPE} - \text{Depreciation} \\
                                     &+ \text{Intangible Assets and Others}        
            \end{align*}

            Current Liabilities includes \emph{Accounts payable}, Notes payable, Accrued expenses. Long term Liabilities includes deferred taxes, \emph{long-term debt}. Stockholder's equity: Preferred stock, Common stock, capital surplus, \emph{Accumulated retained earnings}(which links balance sheet and income statement).
            
            \begin{align*}
            \text{Current Liability} &= \text{Accounts\ Payable + Note\ Payable} \\
                &+ \text{Accrued Expenses} \text{(can be neglected)}
            \end{align*}

            $$\text{Long\text{-}term Liability} = \text{Deferred\ Taxes + Long\text{-}Term Debt}$$
            \begin{align*}
                \text{Stockholder's\ equity} &= \text{Preferred\ Stock} \\
                                      &+ \text{Common\ Stock}\\
                                      &+ \text{Capital\ Surplus} \\ 
                                      &+ \text{Accumulated Retained Earnings}
            \end{align*}
        
        \subsubsection{Market Vs. Book Value}
            $$\text{Market\ Value} = P \times N \quad\quad \text{Can't be negative}$$ 
            $$\text{Bool\ Value} = A - L \quad\quad \text{Can be negative}$$

            Market value means the value of equity can be sold in the market. Book values are calculated in the balance sheet as historical price based on book value. Market values matter more than book value according to the goal of financial management. The market value will be affected be market by price, eg. increasing value of raw material, to be different with book value. 
            \begin{center}   
            Net\ Working\ Capital = Current\ Assets - Current\ Liabilities
            \end{center}
            A positive NWC means the cash will become available over the next 12 months will be greater than the cash that must be paid out; NWC usually grows with the firm. We can use the change in NWC to estimate the growth situation.

    %Cspell: ignore EBIT 

    \subsection{Income Statement}
        \textbf{Definition}: It's more like a video of the firm's operations for a specified period of time.

        \begin{center}
        Income = Revenue - Expenses
        \end{center}

        \textbf{Operation section} of the income statement reports the firm's revenue and expenses from principle operation. There also financial cost(interest expense) and tax. EBIT is an important concept that it summarizes earnings before taxes and financing costs. 
        \begin{align*}
            \text{Operating\ Income} &= \text{Total\ Operating\ revenues} \\   
                              &- \text{Costs\ of\ Goods\ Sold} \\
                              &- \text{Selling,\ General,\ Administrative\ Expense}\\
                              &- \text{Depreciation} \\
        \end{align*}  
            $$\text{EBIT} = \text{Operating\ Income} + \text{Other Income}$$
            $$\text{Tax} = (\text{EBIT} - \text{Interest}) \times \tau $$
            \begin{center}
                Net Income = Addition to retained earnings + Dividends
            \end{center}
         Debt will have interest without tax, which means interest expenses can reduce the tax, the phenomena of \emph{tax shield}. Interest expense should be dealt first as they are not a part of tax. 
        
    \subsection{Cash Flow} 
        \subsubsection{Accounting Perspective}
        \textbf{Cash Flow} is one of the most important pieces of information that a financial manager can derive from financial statements. There is an official accounting statement called the statement of cash flow to identify inflow and outflow. In accounting perspective, what we concern the cash actually earned. 

        There are three types of cash flows 

        $a$. Cash flow from operating activities. 

        $b$. Cash flow from investing activities (Asset). 

        $c$. Cash flow from financing activities (Stock). 

        Accounts payable belongs to operating while Notes payable belongs to financial activity for note are financial instrument. 
        
        \subsubsection{Financial Perspective}
        From the financial perspective, we concern the capability to generate total cash flow. Cash flow is not the same as NWC as increasing inventories using cash will not be counted in NWC; (total / free) Cash Flow from Assets(CFFA) of the firm:   
        \begin{align*}
            \text{CFFA} = \text{Cash\ Flow\ to\ Creditors} + \text{Cash\ Flow\ to\ Shareholders}
        \end{align*}
        \begin{align*}
            \text{CFFA} &= \text{Operating\ Cash\ Flow}(OFC) \\
                 &- \text{Net\ Capital\ Spending} (NCS)\\
                 &- \text{Change\ in\ Net\ Working\ Capital} (\Delta NWC) 
        \end{align*}

        Cash flow from operations reflects the cash flow generated by business activities. So we need add back depreciation part in the EBIT.
    
        \begin{align*}
            \text{Operating Cash Flow} &= \text{EBIT} \\
                &+ \text{Depreciation}\\
                &- \text{Taxes}
        \end{align*}

        \begin{align*} \text{Net Capital Spending} &= \text{Ending\ Net\ Fixed\ Assets}\\
            &- \text{Beginning\ Net\ Fixed\ Assets}\\ 
            &+  \text{Depreciation}
        \end{align*}

        $$\Delta NWC = NWC_{t} - NWC_{t - 1}$$
        \begin{center}
        Cash\ Flow\ to\ Creditors = Interest\ Paid - Net\ New\ Borrowing

        Cash\ Flow\ to\ Shareholders = Dividends\ Paid - Net\ New\ Equity\ Raised
        \end{center}



        Income does not equal to the amount of cash the firm has earned:

            $a$. Non-Cash Expenses
        
            $b$. Uses of Cash not on the Income Statement such as investment in EBIT
    
    \subsection{Categories of Financial Ratios}
        Ratios are main instrument to analyze. The formula is not actually required as they maybe given in exam. It's the analysis that is important. 
        
        \subsubsection{Short-term solvency or liquidity ratios}

        Current Ratio is used as the most important way to measure debt payment in short run. Selling inventories will rise it as inventories are usually calculated at cost. 
        $$\text{Current\ Ratio} = \frac{\text{Current\ Asset}(CA)}{\text{Current\ Liability}(CL)}$$

        Quick Ratio is to measure the ability to liquidate at once to repay debt.
        $$\text{Quick\ Ratio} = \frac{CA - \text{Inventory}}{CL}$$
        
        Cash Ratio is to measure the ability to use money 
        $$\text{Cash\ Ratio} = \frac{\text{Cash}}{CL}$$

        \subsubsection{Long-term  solvency or financial leverage ratios}
        $$\text{Total\ Debt\ Ratio} = \frac{\text{Total\ Asset} - \text{Total\ Equity}}{\text{Total\ Asset}}$$
        $$\frac{\text{Debt}}{\text{Equity}} = \frac{\text{Total\ Debt}}{\text{Total\ Equity}}$$
        $$\text{Equity\ Multiplier} (EM) = \frac{\text{Total\ Asset}}{\text{Total\ Equity}} = 1 + \frac{\text{Debt}}{\text{Equity}}$$

        The bigger of $EM$, the ratio of the equity is smaller, meaning more debt and crisis potential. 

        Times Interest Earned: how well a company has its interest obligations covered
        $$\text{Times\ Interest\ Earned} = \frac{\text{EBIT}}{\text{Interest}}$$
        
        As the EBIT is not current cash available to pay back debt, we need to add up current amortization and depreciation.
        $$\text{Cash\ Coverage} = \frac{\text{EBIT + Depreciation + Amortization}}{\text{Interest}}$$
        
        
        \subsubsection{Asset Management or Turnover  ratios}
        
        All the ratios here is to describe the how efficiently or intensively a firm uses its assets to generate sales 

        Inventory Ratios is used to measure the speed the inventory turns to sold goods. Day's scale to some extent is to measure the average time to store the inventory before it is sold.
            $$\text{Inventory\ Turnover} = \frac{\text{Cost\ of\ Goods\ Sold}}{\text{Inventory}}$$
            $$\text{Day's\ Sales\ in\ Inventory} = \frac{365}{\text{Inventory\ Turnover}}$$
        
        Receivable Ratios is used to measure the speed of retrieving accounts receivable. This is an indication how fast we can collect the money.
            $$\text{Receivable Turnover} = \frac{\text{Sales}}{\text{Accounts\ Receivable}}$$
            $$\text{Day's\ Sales\ in\ Receivable} = \frac{365}{\text{Receivable\ Turnover}}$$

        Total Asset Turnover, reflects long-run efficiency they generate income. 
            $$\text{Total\ Asset\ Turnover}(TAT) = \frac{\text{Sales}}{\text{Total\ Asset}}$$

            It's not unusual for TAT < 1, especially if a firm has a large amount of fixed assets.  
        \subsubsection{Computing Profitability Measures}    
            The most important ratio in measuring a company.   

            %Cspell: ignore EBITDA

            $$\text{Profit\ Margin} = \frac{\text{Net\ Income}}{\text{Sales}}$$

            Compared to net income, EBITDA is pre-tax and add up none-cash expenses, which focuses on operating cash flow. 

            $$\text{EBITDA\ Margin} = \frac{\text{EBITDA}}{\text{Sales}}$$
            
            ROA: a measure of profit per dollar of assets. 

            $$\text{Return\ on\ Assets } (ROA) = \frac{\text{Net\ Income}}{\text{Total\ Asset}}$$

            ROE: a measure of how the stockholders fared during the year. R

            $$\text{Return\ on\ Equity } (ROE) = \frac{\text{Net\ Income}}{\text{Total\ Equity}}$$
        \subsubsection{Computing Market Value Measures}
            $$\text{Market\ Capitalization} = \text{Share\ Price} \times \text{Share\ Outstanding}$$
            $$\text{Earnings per Share} (EPS)= \frac{\text{Net\ Income}}{\text{Shares\ Outstanding}}$$
            $$\text{Price Earning Ratio} (PE) = \frac{\text{Price\ per\ Share}}{\text{Earnings\ per\ Share}}$$

            It demonstrates the overall to evaluate company market value. 

            $$\text{Market\text{-}to\text{-}Book\ Ratio} = \frac{\text{Market\ Value\ per\ Share}}
            {\text{Book\ value\ per\ Share}}$$
            $$\text{Book Value per Share} = \frac{\text{Total equity}}{\text{Shares Outstanding}}$$

            As mentioned , book value per share is a historical value. As the investors think the company is potential, the PE \& MB ratio will be bigger than 1. If investors are irrational, the market-value may be overpriced. 

            Enterprise value is a measure of firm value that is very closely related market capitalization.

            \begin{align*}
                \text{Enterprise\ Value } (EV) &= \text{Market\ Capitalization} \\
                                        &+ \text{Market\ Value\ of\ Interest\ Bearing\ Debt} \\
                                        &- \text{Cash}
            \end{align*}

            It represents a value of an enterprises. The purpose is to better estimate how much it would take to buy all of the outstanding stock of a firm and also to pay off the debt. The adjustment for cash is to recognize that if we were a buyer the cash could be used immediately to buy back debt or pay a dividend. 

            $$EV\ \text{Multiple} = \frac{EV}{\text{EBITDA}}$$

            It only tells about the business decision and exclude all non-business factors like accounting factors and leverage. 
            
        \subsubsection{The Du Pont Identity}
            ROE: Rate of Return on Common Stockholder equity. The higher it is, the more earnings for the shareholders to investment 
            \begin{align*}
                ROE &= \frac{NI}{TE} \\
                    &= \frac{NI}{TE} \times \frac{TA}{TA} \\
                    &= \frac{NI}{TA} \times \frac{TA}{TE} = ROA \times EM \\
                    &= \frac{NI}{TE} \times \frac{TA}{TA} \times \frac{Sales}{Sales}\\
                    &= \frac{NI}{\text{Sales}} \times \frac{\text{Sales}}{TA} \times \frac{TA}{TE}\\
                    &= PM \times TAT \times EM
            \end{align*}

            $$\text{Return on Asset} = \text{Profit Margin} \times \text{Total Asset Turnover}$$
            $$\text{Return on Equity} = \text{Profit Margin} \times \text{Total Asset Turnover} \times \text{Equity Multiplier}$$
    \subsection{Potential problems}
        Different fiscal year
        Extraodinary event. 
    
    \subsection{Effect to evaluate financial statement}
        For internal uses: it's a performance evaluation and planning for the future. 
        It's the only way for outsider to know what's real going on within the company.
        
    \section{Arbitrage and Financial Decision Making}
        decision is to compare costs and benefits in \emph{common terms}(cash).
        \subsection{Time Value of Money}
            The difference in value between money today and money in the future is called the \emph{value of time}.

            Interest rate: An exchange rate across time. 
            
            $$Risk\text{-}free\ Interest = r_f$$

            Which is perfectly secured without risk.  
            
            \subsubsection{Net Present Value}
                \textbf{Net Present Value}(NPV) of a project or investment is the difference between the present value of its benefit and the present value of its costs. 

                $$NPV = PV (all\ Project\ cash\ flows)$$
                $$NPV = PV(benefit) - PV(costs)$$
            
                The sign of NPV is the rule to judge weather accepting or rejecting. The higher of NPV, the more priority the project has. 

                The starting cash outlay can affect your choice as you can borrow money. The remaining cash will go  to bank. 
            
            \subsubsection{The separation theorem }
                We assume the interest rate is higher than risk-free rate and consistent with lending and borrowing rate and lending and borrowing is without limitations. 

                The project you take is your \emph{investment decision}. The way you will use the money is your \emph{financial decision}. The investment decision is independent of financial decision. That means you should first maximize NPV first without considering current saving. 

        \subsection{Arbitrage and The Law of One Price}
            \textbf{Arbitrage}: The opportunity to buy and sell equivalent goods in different markets to exploit a price difference without taking any risk or incurring any cost. 

            \textbf{The Law of One Price}
                If equivalent investment opportunities trade simultaneously in different competitive markets, then they must trade for the same price in both markets as arbitrage eliminate the price difference.
                Either overstated or understated price will lead to arbitrage.
                
            \subsection{Valuing a Security}
                No arbitrage price of a security: 
                $$Price(Security) = PV(All\ Cash\ Flows\ Paid\ by\ the\ Security)$$

                When price of a security doesn't equal this price, arbitrage chance will occur. If the price is lower, you can choose to lend the amount of no-arbitrage price and buy it; if the price is higher you can sell it and save the amount of  no-arbitrage price. 

            \subsubsection{Valuing a Portfolio}
                The law of one price guarantee value-additivity: the cash flow C generate is equal to the combination of A and B

                $$Price(C) = Price(A + B) = Price(A) + Price(B)$$

            \subsubsection{The Price of Risk}
                Market index is a value measuring the overall situation. 

                \textbf{Risk Averse}:Investors prefer  to  have a safe income rather than a risky one of the same average amount. 

                \textbf{Risk Premium}: The additional return that investors expect to earn to compensate them for a security's risk. It depends on risk. 

                The risk is relative to the overall market.  A security's risk premium will be higher the more its returns tend to vary with the overall tend to vary with overall economy and the market index. If the security's returns vary in the opposite direction of the market index, it offers insurance and will have a negative risk premium like food industry.  

                In order to lower the risk premium, it's better to have multiple securities. 

                $$r_s = r_f + r_{premium}$$

                The $r_s$ is better to evaluate the interest rate for risky securities. $r_s$ is actually the expected interest rate.  The risk difference is correlate with risk premium. For 3 times of risk range, you will gain 3 times of risk premium.

                To calculate the price of a risky security: 

                $$p = \frac{Expected\ Return}{r_s}$$
    \section{The Time Value of Money}
        The content is devoted for multi-period cash flow. 

        \subsection{Timeline}
            Only the value at the same point in time can be compared or combined; To move a cash flow forward in time, you must compound it: 

            $$FV_n = C \times (1 + r) ^ n$$

            The negative n means move a cash flow backward. 

            $$PV_n = \frac{C}{(1 + r)^n}$$

            The future value is to estimate the amount of savings. 
        \subsection{The Power of Compounding}
           \textbf{Compounding}: the interest on interest as the interest in the early time period will generate interest. 

            The rule of 72 say that the double of wealth will take 72 years with $1\%$ interest rate. 

            $$T_d = 72 \times \frac{1\%}{r}$$

        \subsection{Valuing a Stream of Cash Flow}
        
        $$PV = \sum_{n = 0}^{N}PV(C_n) = \sum_{n = 0}^{N}\frac{C_n}{(1 + r)^n}$$

        Some easier formula for specific situation. 

        \subsection{Annuities and Perpetuities Defined}
        
        \textbf{Annuity}: Finite series of equal payments that occur at regular intervals. If the first payment of the period, it's called an \emph{ordinary annuity}. If the first payment occurs at the beginning of the period, it's called an \emph{annuity due}. 

        \textbf{Perpetuity}: Infinite series of equal payments.
        $$PV = C\sum_{n = 1}^{\infty}\frac{1}{(1 + r)^n} = \frac{C}{r}$$

        For growing Perpetuity with constant growing unit($C$ is the $t_1$ value and $PV$ is evaluated in $t_0$): 

        $$GP = C\sum_{n = 1}^{\infty}\frac{(1 + g)^{n - 1}}{{1 + r)^n}} = \frac{C}{r - g}$$

        For Annuity for $t$ years:

        $$Ann = PV_{0} - PV_{t} = \frac{C}{r}( 1 - \frac{1}{(1 + r)^t})$$

        For growing annuity:

        $$GAnn = GP_0 - GP_t = \frac{C}{r - g}(1 - (\frac{1 + g}{1 + r})^t)$$

        For the interest for more than one year, can be solved by the equation:
        $$1 + R = (1 + r) ^ n$$

\section{Interest rate}
    \subsection{Annual Percentage Rate}
        \textbf{APR} is the annual rate that is quoted by the law; 
            $$APR = Period\ Rate \times Number\ of\ Periods\ per\ Year$$
    \subsection{Effective Annual Rate}
        \textbf{EAR} is the actual rate paid after accounting for compounding that occurs the year.
        
        $$EAR = (1 + \frac{APR}{m})^m - 1$$

        The higher the m is, the more is the EAR; With a given APR and a compounding time span $n$ month, we can compute the monthly interest as :

        $$i = (1 + \frac{APR}{\frac{12}{n}}) ^{\frac{1}{n}} - 1$$
        $$i = (1 + EAR) ^{\frac{1}{12}} - 1,\quad EAR = (1 + \frac{APR}{\frac{12}{n}})^{\frac{12}{n}} - 1$$

        Specifically, to calculate monthly with APR or EAR: 
        $$i = APR / 12$$
        $$i = (1 + EAR) ^ {\frac{1}{12}} - 1$$


    \subsection{Interest Rates and Inflation}
        the real interest rate:
        $$1 + R = \frac{1 + r}{1 + i^e}$$

        \textbf{Term Structure}: The relationship between the investment term and the interest rate.

        \textbf{Yield Curve}: A graph of the term structure.

    \section{Investment Rules}
        \subsection{Capital budgeting}
            Analysis of potential additions to fixed assets. This is a long-term decision involving large expenditures.
            
            Steps:

            \quad(a). Estimate CFs(inflows \& outflows)
            
            \quad(b). Assess riskiness of CFs

            \quad(c). Determine the appropriate cost of capital 

            \quad(d). Evaluate projects 
            
            \quad(e). Accept/Reject decision 

            Estimating cash flow only includes incremental cash flow generated from accepting the decision.
            
            Independent projects: if the cash flows of one are unaffected by the acceptance of the other. 

            Mutually exclusive projects: if the cash flows of one can be adversely impacted by the acceptance of the other. 
        \subsection{Investment Rules} 
            \subsubsection{Payback Period rule}
                Payback period is the number of years required to recover a project's initial cost back with a preset time. Independent projects will be accepted if the payback period is less than some present limit. For mutually exclusive projects, we choose the one with minimum cost.

                However, when using payback period rule, we don't take time value into consideration so that is not so accurate. 

                Drawback:
                
                \quad a). We can't choose a bigger NPV projects 
                
                \quad b). We don't care the income after payback period, which may have big affects. 

                \quad c). Arbitrary standard for payback period
            \subsubsection{Discounted Payback period}
                Take the time value into consideration. 
            \subsubsection{Net Present Value}
                As discussed before:
                $$NPV = \sum_{k = 1}^{n}\frac{\text{CF}_k}{(1 + r)^k}$$

                For independent projects, choose any project that NPV > 0; For mutually exclusive projects, choose the one with highest return. 
            \subsubsection{Internal Rate of Return}
                IRR is the discount rate that forces PV of inflows equal to cost, and the NPV = 0; It means that the highest financial cost you can accept to do the project; However, when using IRR, you assume you the discount rate is IRR rather the real interest rate; 
                $$0 = \sum_{i = 0}^{n}\frac{\text{CF}_t}{(1 + \text{IRR})^t}$$

                For Independent projects: accept a project if IRR > current discount rate(financial cost); For mutually exclusive projects, choose the one with highest IRR; This is even more important and it's independent of other interest rate; 
            \subsubsection{Reinvestment rate assumption}
                NPV assumes we reinvest in $r$  while IRR method assume we reinvest at IRR; There is a little different between NPV and IRR, and NPV is more realistic;

                When NPV and IRR agree: 
                
                \quad(a). There is only one cash outflow at time 0 and all other cash flow are positive; 

                \quad(b). Only one is under consideration (not mutually exclusive projects) 

                \quad(c). The opportunity cost of capital is the same for all period 

                NPV directly measures the increase in value to the firm, which is reliable. 

                Problems with IRR: 

                \quad(a). Negative cash flow happens after positive cash flow(Projects of the loan type)

                %Cspell: ignore IRRs
                \quad(b): Multiple IRRs
                
                \quad(c):  For mutually exclusive projects with different Scale or different pattern

                \quad(d): IRR Ignores Term Structures of Interest Rate: the interest rates may change for periods;
            \subsubsection{Profitability Index(PI)}
                The ratio of the present value of future cash flows and the initial cost. Accept all projects with PI > 1 and for mutually exclusive projects choose the highest; It's consistent with NPV but disregarding initial cost and focusing on efficiency of projects, which can be used in multiple projects decision. 
                $$PI = \frac{PV_f}{C_0}$$
\newpage
    \section{Capital Budgeting}
    $$\begin{aligned}
        \text{EBIT} &= \text{unLevered netIncome} * (1 - \tau) \\
            &= (\text{Revenue - Costs - Depreciation}) * (1 - \tau)
    \end{aligned}$$

    $$\begin{aligned}
                \text{Free Cash Flow} &=
            \text{(Revenues - Costs - Depreciation)} \times (1 - \tau_i) \\
            &+ \text{Depreciation} - \text{CapEx} - \Delta NWC
            \end{aligned}$$
    
    $$c_t = (1 - \tau)(\text{Operating profit})_t + \tau \text{DPR}_r - \text{CAPX}_t - \Delta WC_t$$

    We don't take financing into consideration;

    General rules: 

    \quad(a). Use cash flows rather than accounting earnings: you may not collect the money back of a contract; 
    
    \quad(b). Use incremental cash flow: Only use cash flows attributable to the project; DO NOT USE SUNK COSTS!; Remember Externality and opportunity cost; Sunk costs include all costs before $t_0$. 

    Externality: Indirect effects of the project that may affect the profits of other business activity of the firm. \emph{Cannibalization} is when sales of a new product displaces sales of an existing product, which will influence both revenue and cost.
    
    %Cspell:ignore MACRS,CAPEX

    \subsection{Some adjustment}

    Further  adjustment: Modified accelerated cost recovery system depreciation (MACRS depreciation). This assumes the depreciation starts from year 0, which is not the same with common depreciation. This will affect tax shield.
    \medskip
    
    Liquidation or Salvage Value:

    $$\text{Capital Gain} = \text{Sale Price} - \text{Book Value}$$
    $$\text{Book Value} = \text{Purchase Price} - \text{Accumulated Depreciation}$$
    $$\text{After-Tax Cash Flow Form Asset Sale(CAPEX)} = \text{Sale Price} - (\tau_t \times \text{Capital Gain})$$

    The use of old equipment will have an opportunity cost equaling after-tax cash flow from asset sale as you can sell it. The purchase of new equipment will be exactly the price you buy it; 

    \medskip

    Terminal or Continuation Value: This amount represents the market value of the free cash flow from the project at all future dates. 

    %Cspell: ignore carryforward, carrybacks

    Tax Carryforward: Tax loss carryforward  amd carrybacks allow Corporation to take losses during its current year and offset them against gains in nearby years. 

    \subsection{Evaluating NPV estimates} 
        Forecasting risk -- how sensitive is our NPV to changes in the cash flow estimates, the more sensitive, the greater the forecasting risk. 

        Source of Value: what factors NPV depend on

        \subsubsection{Sensitivity Analysis} 
            Sensitivity analysis examines how sensitive a particular NPV calculation is to changes in the underlying assumption. We set one factors to different situations with other factors stay the constant; after that we can find which parameter is more important. 3r

            Scenario Analysis: As factors can be interrelated, so we created a series of scenarios with different assumptions to estimate the risk. 

\end{document}