\documentclass[10pt, a4paper]{article}
    \author{ianaesthetic}
    \title{Unity3D note}
\usepackage{indentfirst, amsmath, fontspec, listings, xcolor}

\newfontfamily\consolas{Consolas}
\lstset{numbers = left, numberstyle = \small\consolas, basicstyle=\small\consolas}

\XeTeXlinebreaklocale "zh"
\XeTeXlinebreakskip = 0pt plus 1pt

\begin{document}
    \section{Introduction to Corporate Finance}
        \subsection{Financial Management Decisions}
            Cash first raised from investor, invested in firm, generated by operations, and finally reinvested or returned to inventers.
        \subsection{Corporation}
            \textbf{Definition}: from CF perspective, a firm is a collection of projects. Projects are anything that can generate cash. It's also a business form. There are different types of business forms including: \emph{Sole Proprietorship}, \emph{Partnership (General and Limited-Liability)}, \emph{Limited-Liability Company}, \emph{Corporation}.
            
            Corporation is consists of \emph{board of directors} who manage the \emph{asset}, \emph{debt} which is concerned with \emph{debt holders},  \emph{equity} which is concerned with \emph{share holders}.

        \subsection{Goal of Finance Management}
            \textbf{Primary} financial goal is \emph{shareholder wealth maximization}, which can be translated to maximizing \emph{stock price}. 
            
            Stock price maximization is not same as profit maximization: stock price relies upon \emph{current earnings}, \emph{future earnings} and \emph{cash flow}. They may not be change in the same way. In particular, factors that determine stock price are: 

            $a$. Projected cash flow to share holders
            
            $b$. Timing of the cash flow stream

            $c$. Rickness of the cash flow

        
        
\end{document}