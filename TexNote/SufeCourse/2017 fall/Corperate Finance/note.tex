\documentclass[10pt, a4paper]{article}
    \author{ianaesthetic}
    \title{Unity3D note}
\usepackage{indentfirst, amsmath, fontspec, listings, xcolor}

\newfontfamily\consolas{Consolas}
\lstset{numbers = left, numberstyle = \small\consolas, basicstyle=\small\consolas}

\XeTeXlinebreaklocale "zh"
\XeTeXlinebreakskip = 0pt plus 1pt

\begin{document}
    \section{Introduction to Corporate Finance}
        \subsection{Financial Management Decisions}
            Cash first raised from investor, invested in firm, generated by operations, and finally reinvested or returned to inventers.
        \subsection{Corporation}
            \textbf{Definition}: from CF perspective, a firm is a collection of projects. Projects are anything that can generate cash. It's also a business form. There are different types of business forms including: \emph{Sole Proprietorship}, \emph{Partnership (General and Limited-Liability)}, \emph{Limited-Liability Company}, \emph{Corporation}.

            Disadvantage of Sole and Partnership: 

                (a: Unlimited Liability
                (b: difficult to raise money 
                (c: difficult to transfer. 
            
            Differences between of Limited-Liability Company and Corporation: whether it has gone public. 
            
            Corporation is consists of \emph{board of directors} who manage the \emph{asset}, \emph{debt} which is concerned with \emph{debt holders},  \emph{equity} which is concerned with \emph{share holders}.

        \subsection{Goal of Finance Management}
            \textbf{Primary} financial goal is \emph{shareholder wealth maximization}, which can be translated to maximizing \emph{stock price}. 
            
            Stock price maximization is not same as profit maximization: stock price relies upon \emph{current earnings}, \emph{future earnings} and \emph{cash flow}. They may not be change in the same way. In particular, factors that determine stock price are: 

            $a$. Projected cash flow to share holders
            
            $b$. Timing of the cash flow stream

            $c$. Rickness of the cash flow

        \subsection{Agency problem}
            Share holders and Managers: This is the most important agency problem. Managers are inclined to act in their own best interest. 
            
            Share holders Vs Creditors: Deriving from \emph{new investment opportunities, dividend versus retained earnings}.

            Creditors vs Managers: Financing decision; Senior versus Junior bond
        \subsection{Financial Market}
            A market is a venue where goods and services are exchanged. 

            A financial market is a place where individuals and organizations to raise capital with \emph{investor} and \emph{borrower}.

            Primary market: IPO market, usually bankers involved. For individuals, they have limited access limited by original holding and fortune. This the capital the company raised. 

            Secondary market: The trade is between investors involving buying and selling stocks. 

            \noindent Process:

                $a$. Firm issues securities to raise cash.
                
                $b$. Firm invests in assets.
                
                $c$. Firm's operations generate cash flows.

                $d$. Cash is paid to government as \emph{taxes}. \emph{Other stakeholder} may receive cash. 

                $e$. Reinvested cash flows are plowed back into firm
                
                $d$. Cash is paid out to investors in the form of interest and dividends.
    
    \section{Review of Financial Statement Analysis}
        \subsection{Balance Sheet}
            \textbf{Definition}: a snapshot of the firm's asset and liabilities at a given point of time, indicating all operations during the time point. 
            $$Assets = Liability + Stockholder's\ Equity$$

            Assets are listed in order to liquidity(Current asset > Fixed Asset[Tangible and Intangible]). Liquidity stands for ease of cash conversion without significant loss.

            Current asset includes \emph{Cash and equivalents}, \emph{Accounts receivable}, \emph{Inventories}. Fixed asset includes \emph{Property, plant and equipment(PPE)}, \emph{Less accumulated depreciation}, Intangible assets and other. Amotization;

            Current Liabilities includes \emph{Accounts payable}, Notes payable, Accrued expenses. Long term Liabilities includes deferred taxes, \emph{long-term debt}. Stockholder's equity: Preferred stock, Common stock, capital surplus, \emph{Accumulated retained earnings}(which links the asset and liability).
        \subsubsection{Market Vs. Book Value}
            $$Market\ Value = P \times N \quad\quad \text{Can't be negative}$$ 
            $$Bool\ Value = A - L \quad\quad \text{Can be negative}$$

            Market value means the value of equity can be sold in the market. Book values are calculated in the balance sheet to define value. Market values matter more than bool value according to the goal of financial management. The market value will be affected be market, eg. increasing value of raw material, to be different with book value. 

            $$Net\ Working\ Capital = Current\ Assets - Current\ Liabilities$$

            NWC usually grows with the firm. We can use the change in NWC to estimate the growth situation.



        
\end{document}