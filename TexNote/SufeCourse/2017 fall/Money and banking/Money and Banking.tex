\documentclass[10pt, a4paper]{article}
    \author{ianaesthetic}
    \title{Unity3D note}
\usepackage{indentfirst, amsmath, fontspec, listings, xcolor}

\newfontfamily\consolas{Consolas}
\lstset{numberstyle = \small\consolas, basicstyle=\small\consolas}

\XeTeXlinebreaklocale "zh"
\XeTeXlinebreakskip = 0pt plus 1pt

\begin{document}
    \section{Introduction} 
        Study: Financial Market, Financial Institution, Financial Management.
        \subsection{Financial Market} 
        \textbf{Financial Market}: Markets where the funds flow from lenders to the borrowers. It's the \emph{channel} funds from servers to investors and can \emph{promote} economic efficiency.
        \subsubsection{The Bond Market and Interest Rate}
            A \textbf{security (financial instrument)} is a claim on the issuer's future income or assets.  it includes \textbf{Bond} and \textbf{Stock}.

            \textbf{Bond} is a debt security that promises to make payments periodically for a specified period of time. 
            
            \textbf{Stock}: Common stock represents a share of ownership.

            \textbf{Interest Rate}: Cost of borrowing or the price paid for rental of funds. When it increases, It can affect consumption, saving and investment. 
        \subsubsection{The Foreign Exchange Market}
            \textbf{Definition}: where funds are converted from one currency into another. 

            \textbf{Foreign Exchange Rate}: the price of one currency in terms of another currency. It mainly influence the imports and exports. There three methods of quotations : 
            
            \textbf{Direct}: $100\ FOREX = e_x\ Domestic\ Currency$ 

            \textbf{Indirect}: $100\ Domestic\ Currency = e_y\ FOREX$

            \textbf{USD} $1\ USD = e_z\ Domestic\ Currency$
            \newline 

            (de)appreciate: the influence of market; 
            
            (de)valuate: the influence of government

            US Dollar Index(USDX): To evaluate the value of Dollar by exchange rate with other countries. 

            \subsubsection{Factors of Foreign Exchange Rate}
                The Exchange rate is something rated to the Demand \& Supply of foreign currency. For e - Q(foreign currency) graph, it satisfies the normal demand and supply curve. All the exchange rate is discussed in direct quotations. 

                Balance of Payment(BOP):\emph{surplus} (FOREX $\uparrow$, e $\downarrow$),\emph{deficit}(FOREX $\downarrow$, e $\downarrow$).
                
                Economic Performance can be apparent in some time point(end of the year). Good performance will lure foreign cash(e $\downarrow$).
                
                Interest rate will also affect(i $\uparrow$, e $\downarrow$, for it can lure foreign investment). 

                Price will affect as well(p $\uparrow$, e $\uparrow$).
    \subsection{Banking and Financial Institutions}
        \textbf{Functions}: 

            $a$. They make financial markets work 

            $b$. Financial intermediary for funds flowing from savers to investors

            $c$. Important effects on the performance of the economy as a whole
        
        \textbf{Examples}: 
            
            \quad Insurance Company; 
            
            \quad Banks; 
            
            \quad Securities Firm; 
            
            \quad Trust Company; 
            
            \quad Credit Union; 
            
            \quad Financial Company;
            
            \quad Financial Leasing Company;
            
            \quad Credit Rating Agency; 

            \quad Exchanger; 
            
            \quad Funds Management; 

        All examples can be divided to two types: \emph{banks-institution} and \emph{non-banks-institution}. Banks are the largest financial intermediaries in our economy, including Central Bank; Commercial Bank; Policy Bank(non-profitable); Specialized Bank. There is a trend of \emph{disintermediation}.
        
        \subsubsection{Financial Innovation}
            New financial product, financial institution, financial services and more appear, such as e-finance and financial derivatives. 
        \subsection{Money and Money policy}
            \textbf{Definition}: Money is defined as anything that is \emph{generally} accepted in payment for goods or services or in the repayment of debts. It is linked to changes in economic variables that affect all of us and are important to the health of the economy. 

            Money affect \emph{business cycle} (including four stages: \emph{recession},\emph{depression} \emph{recovery} and \emph{boom}. 

            Money growth rate will have a severe decrease and rapid growth. The recession will cause the money decreasing, for example, people are don't intend to consume. After a big recession, government will get in and add the amount of money in the market to boost economy. This phenomena can be used to  predict the performance.
            
            \subsubsection{Relationship between Money and Inflation} 
                \textbf{The aggregate price level}: the average price of goods and services in an economy. 
                
                \textbf{Inflation}: A continue rise in the price level which affects all economic players. 

                The rise of money supply will lead to the rise of inflation. 
            \subsubsection{Money and Interest Rates}
                Interest Rates are the price of money. The increase of money supply will lead to the decrease of interest rate, something like demand and supply.
            \subsubsection{Monetary and Fiscal Policy}
                \textbf{Monetary policy} is the management of the money supply and interest rates. 

                \textbf{Fiscal Policy} is government spending and taxation, which is set by department of Treasure. 

                Both policies make government be able to manage and manipulate markets. Government raise money from taxation, profit of state-owned company, fee, etc. Government spends on procurement, investment, transfer payment,interest payment, etc. 

\end{document}