\documentclass[10pt, a4paper]{article}
    \author{ianaesthetic}
    \title{Unity3D note}
\usepackage{indentfirst, amsmath, fontspec, listings, xcolor}

\newfontfamily\consolas{Consolas}
\lstset{numberstyle = \small\consolas, basicstyle=\small\consolas}

\XeTeXlinebreaklocale "zh"
\XeTeXlinebreakskip = 0pt plus 1pt

%Cspell: ignore repo

\begin{document}
    \section{Introduction} 
        Study: Financial Market, Financial Institution, Financial Management.
        \subsection{Financial Market} 
        \textbf{Financial Market}: Markets where the funds flow from lenders to the borrowers. It's the \emph{channel} funds from servers to investors and can \emph{promote} economic efficiency.
        \subsubsection{The Bond Market and Interest Rate}
            A \textbf{security (financial instrument)} is a claim on the issuer's future income or assets.  it includes \textbf{Bond} and \textbf{Stock}.

            \textbf{Bond} is a debt security that promises to make payments periodically for a specified period of time. 
            
            \textbf{Stock}: Common stock represents a share of ownership.

            \textbf{Interest Rate}: Cost of borrowing or the price paid for rental of funds. When it increases, It can affect consumption, saving and investment. 
        \subsubsection{The Foreign Exchange Market}
            \textbf{Definition}: where funds are converted from one currency into another. 

            \textbf{Foreign Exchange Rate}: the price of one currency in terms of another currency. It mainly influence the imports and exports. There three methods of quotations : 
            
            \textbf{Direct}: $100\ FOREX = e_x\ Domestic\ Currency$ 

            \textbf{Indirect}: $100\ Domestic\ Currency = e_y\ FOREX$

            \textbf{USD} $1\ USD = e_z\ Domestic\ Currency$
            \newline 

            (de)appreciate: the influence of market; 
            
            (de)valuate: the influence of government

            US Dollar Index(USDX): To evaluate the value of Dollar by exchange rate with other countries. 

            \subsubsection{Factors of Foreign Exchange Rate}
                The Exchange rate is something rated to the Demand \& Supply of foreign currency. For e - Q(foreign currency) graph, it satisfies the normal demand and supply curve. All the exchange rate is discussed in direct quotations. 

                Balance of Payment(BOP):\emph{surplus} (FOREX $\uparrow$, e $\downarrow$),\emph{deficit}(FOREX $\downarrow$, e $\downarrow$).
                
                Economic Performance can be apparent in some time point(end of the year). Good performance will lure foreign cash(e $\downarrow$).
                
                Interest rate will also affect(i $\uparrow$, e $\downarrow$, for it can lure foreign investment). 

                Price will affect as well(p $\uparrow$, e $\uparrow$).
    \subsection{Banking and Financial Institutions}
        \textbf{Functions}: 

            $a$. They make financial markets work 

            $b$. Financial intermediary for funds flowing from savers to investors

            $c$. Important effects on the performance of the economy as a whole
        
        \textbf{Examples}: 
            
            \quad Insurance Company; 
            
            \quad Banks; 
            
            \quad Securities Firm; 
            
            \quad Trust Company; 
            
            \quad Credit Union; 
            
            \quad Financial Company;
            
            \quad Financial Leasing Company;
            
            \quad Credit Rating Agency; 

            \quad Exchanger; 
            
            \quad Funds Management; 

        All examples can be divided to two types: \emph{banks-institution} and \emph{non-banks-institution}. Banks are the largest financial intermediaries in our economy, including Central Bank; Commercial Bank; Policy Bank(non-profitable); Specialized Bank. There is a trend of \emph{disintermediation}.
        
        \subsubsection{Financial Innovation}
            Anything new in Finance. 

            New financial product, financial institution, financial services and more appear, such as e-finance and financial derivatives. 
        \subsection{Money and Money policy}
            \textbf{Definition}: Money is defined as anything that is \emph{generally} accepted in payment for goods or services or in the repayment of debts. It is linked to changes in economic variables that affect all of us and are important to the health of the economy. 

            Money affect \emph{business cycle} (including four stages: \emph{recession},\emph{depression} \emph{recovery} and \emph{boom}. 

            Money growth rate will have a severe decrease and rapid growth. The recession will cause the money decreasing, for example, people are don't intend to consume. After a big recession, government will get in and add the amount of money in the market to boost economy. This phenomena can be used to  predict the performance.
            
            \subsubsection{Relationship between Money and Inflation} 
                \textbf{The aggregate price level}: the average price of goods and services in an economy. 
                
                \textbf{Inflation}: A continue rise in the price level which affects all economic players. 

                The rise of money supply will lead to the rise of inflation. 
            \subsubsection{Money and Interest Rates}
                Interest Rates are the price of money. The increase of money supply will lead to the decrease of interest rate, something like demand and supply.
            \subsubsection{Monetary and Fiscal Policy}
                \textbf{Monetary policy} is the management of the money supply and interest rates. (Central bank) 

                \textbf{Fiscal Policy} is government spending and taxation, which is set by department of Treasure. (fiscal revenue and expense, the department of treasury) 

                Both policies make government be able to manage and manipulate markets. Government raise money from taxation, profit of state-owned company, fee, etc. Government spends on procurement, investment, transfer payment,interest payment, etc. 

                The effect on aggregate demanding($C + I + G + NX$): Money policy influences C, I, NX, has a indirect effect to AD. Fiscal policy will have a direct effect on G and AD.  

                Interest rate change($\uparrow$, as example) will affect exchange market, stock market($\downarrow$, as it's more difficult to raise money), national bond market, commercial market($\downarrow$, as it will lower the need for consumption).
    
    \section{Financial Market}
        \subsection{Function of Financial Markets} 
            Perform the essential function of channeling funds from lenders to borrowers. It can also promote \emph{economic efficiency} by producing a efficient allocation of capital. It will also directly improve the well-being of consumers by allowing them to time purchases better (allow consumers to use the money in the future to purchase by loans or something else).
            \subsubsection{Channel of Financing}
                \textbf{Direct Finance}: Borrowers borrow funds directly from lenders in financial markets by selling them securities. More specifically, the relationship between lender and borrower is direct and clear. 

                \textbf{Indirect Finance}: Involves a financial intermediary that stands between the lender-savers. The relationship between intial lender and final  borrower is indirect. (Banks)

                The graph of flow of funds can be referred to ppt.  
        \subsection{Structure of Financial Market}
            \subsubsection{Debt and Equity Markets}
                \textbf{Debt instrument} is a \emph{contractual agreement} by the borrower to pay the holder of the instrument fixed dollar amount(interest and principal payments) at regular intervals until a specified date(the \emph{Maturity Date}, when a final payment is made, includes: \emph{bonds}, \emph{mortgage}.

                It has different terms includes: 
                
                \quad short-term($M < 1yr$): T-Bill

                \quad intermediate term($1yr < M < 10yr$): T-Note
   
                \quad long-term ($M > 10yr$): T-Bond \newline

                \textbf{Equity instrument} is claim to share in the net income(income after expenses and taxes) and the asset of a business, includes: \emph{common stock}, \emph{preferred stock}. 

                The difference between common stock and preferred stock: Preferred stock has priority in dividends receiving and liquidation during the corruption. Common stock owner has \emph{voting right} based on the share owned while limited in priority stock owner. And the common stock has priority to buy new shares to avoid dilution. From the crisis perspective, common stock holder has bigger crisis than preferred stock. 

                Priority stock has two type: accumulative and non-accumulative. Accumulative has a right to claim the unpaid dividends in the later year while non-accumulative can just claim current dividends. Accumulative stock is more analogous to long-term bond. 

                The main disadvantage of owning a corporation's equities rather than its debt is that an equity holder is a \emph{residual claimant}.
            \subsubsection{Primary and Secondary Markets}
                \textbf{Primary markets} are those where new security issues sold to initial buyers. Investment Banks will \emph{underwrite} securities in primary markets.  

                \textbf{Second markets} are markets where securities previously issued are bought and sold between investors. Broker(agency) and dealers  work in secondary markets. 
            \subsubsection{Exchanges and Over-the-Counter(OTC) Markets} 
                \textbf{Exchanges markets}: Trades conducted in central locations(eg. New York Stock Exchange, NASDAQ)      
                \textbf{OTC markets}: Dealers at different locations buy and sell(eg. OTCBB).  
            \subsubsection{Money and Capital Markets}
                \textbf{Money Markets} deal with short-term debt instruments. 
                
                \textbf{Capital markets} deal with longer-term debt and equity instruments. 
                
            \subsubsection{Money Markets Instruments}
                Treasury Bills[TB]: Short-term; A solution to the government deficit; IT has the lowest rate in the markets, so called gilt-edged / risk-free  bond, as it's guaranteed by the taxation of the government. They are usually sold at discount by auction to generate interest.
                
                Negotiable Bank Certificates of Deposit (large dominations) [CD]: Large domination; Transferable in second market which can receive the interest before maturity; Of low risk as it's published by guaranteed banks and they absorbs funds quickly. For bank it's of large dominations and can't be taken out before maturity. 
                
                Commercial paper: a unsecured promissory note with a fixed maturity less tha 270 days and is published by well-known cooperation; During selling on credits, or in the credit term, there is no interest and will form accounts receivable/payable; When exceeding the credit term, they will change to note receivable/payable with interest. Notes include Promissory note and draft.
                
                Draft is an order, issued by the creditor for the debtor to pay fo a payee. Promissory note is a promise issued by debtor to pay back to the creditor. \emph{Acceptance} is needed in the draft to promise to pay the debt;\emph{Trader's acceptance} will be in the promissory from the debtor. Bank can also be the issuer by charging the debtor to stamp this trader's acceptance and it will be called \emph{bank acceptance} which has more liquidity as bank has better reputation. \emph{Endorsement} is transferring notes to other to retrieve money before maturity with guaranteeing the debt will be paid. \emph{Discount} is transferring notes to a bank before maturity subtracting the interest. In summary, they are different on their characteristic, issuer, acceptance; If the debt is paid out, it's called \emph{Honor}. 

                Federal Funds: a form of inter-bank offering as are borrowed between financial institution; short-term concentrating on overnight borrowings; large amount for immediate spending; interest rate is liberalized for inter-bank offering expect for Federal Funds;
                
                Repurchase agreement[RP]: After A sells a low-risk bond to B, RP is the scene that A buys back the bond at a higher price later. In fact, it's a loan for A with bond. For A this is a \emph{positive repo} and B is \emph{negative(reverse)} repo. For a, the bond still belongs to A when A is in need of funds. For B is safer to give the loan in this way with bonds as pledge. For normal \emph{pledged repo}, the pledged bonds can't be pledged again. While \emph{outright repo} allow for a shorter term of transition by another repurchase agreement with shorter term;

            \subsubsection{Capital Market Instruments}
                Capital market instruments are for long-term. 

                Bonds: T-Note \& T-Bond with large amount and low risk; 

                Government agency securities: by the agency of government or sponsored by government

                State and local government bonds 

                Cooperate bonds: with a relative hight risk, which introduces credit rating; It includes converted bonds, which can convert the bonds to stock at a price. 

                Corporate stocks:
                
                commercial loans, consumer loans, commercial and farm mortgages, residential mortgage.
    \subsection{Internationalization of}
        \subsubsection{foreign exchange}
        
            Euro-currencies: foreign currencies deposited in banks outside the home country.

            Euro-dollar: dollars deposited in foreign banks outside the US. 

            Euro is actually refers to offshore. 
        \subsubsection{World Bonds Market}
            Foreign bonds: sold in a foreign country and denominated in that country's currency. 

            Euro-bond: bond denominated in a currency other than that of the country in which it is sold. 
        \subsubsection{World Stock Markets}
            Stock price indices: composite indices and component indices.
    \subsection{Financial Intermediaries}
        \subsubsection{Types of Financial Intermediaries}
            Depository institutions: commercial banks, saving banks, credit union, as the only institutions that the main source of liabilities is deposits;  

            Contractual savings institutions: life insurance companies(Policy), fire and casualty insurance companies, pension funds(Contribution); As there is a contract between the institution and consumers;
            
            Investment intermediaries: Finance companies, mutual funds, money market mutual funds; They are related to the capital market;
    
    \subsection{Regulations of financial system}
        Security and Exchange Commission(SEC): Bond and other exchanges are supervised by SEC;
        
        Commodities Futures Trading Commission(CFTC): Futures market exchange; 

        Office of the Controller of the Currency(OCC): Belongs to treasury and is responsible for bank registration 

        Federal Deposit Insurance Corporation (FDIC): To guarantee the deposit  deposit institutions under some limitations.

        Fed reserve system: all the deposit institution; 

        The content of supervision: To increase information for investors to avoid insider trading and reduce adverse selection and moral hazard problems; To ensure the soundness of financial intermediaries, e.g. restrictions on entry, disclosure, limits on competition, restrictions on interest rate; To improve monetary control by monetary policy.
\newpage


\section{Money}
    \emph{Money}:anything that is generally accepted in payment for goods or services or in the repayment of debts
    
   \emph{Currency}: cash; consisting of dollar bills and coins and is one of type of money.
    
    \emph{Wealth}: the total collection of pieces of property that serve to store value. Wealth includes non-monetary part and monetary part which includes money.
    
    \emph{Income}: flow of earnings per unit of time; money belongs to the concept of stocks

    \subsection{Functions of money}
        Medium of Exchange: pays for goods and service with transaction; Without medium, barter economy will appear and bring high transaction costs(double coincidence of wants); So the money is a lubricant; 
        
        Unit of Account: the price;  
        
        Store of Value: used to save purchasing power to divide the process of buying and selling with high liquidity;
        
        Liquidity: the relative ease and speed with which an assets can be converted into a medium of exchange.

        Criteria of money: Standardized, Accepted, Divisible, Easy to carry, Not Deteriorate quickly

    \subsection{Evolution of Payments System}
        \subsubsection{Commodity Money}
            An object that clearly has value to everyone is a likely candidate to serve aas money, and a natural choice is a precious metal such as gold or silver. 

            Precious metals' Advantage: quality uniform; easy to shape; easy to divide; durable 

            Representative: usually bank note and it's based on precious metals

        \subsubsection{Credit Money}
            Fiat Money: Paper currency decreed by governments as legal tender.  
            
            Check: An \emph{instruction} from you to your bank to transfer money from your account to someone else account when she deposits the check; Who receives the check can deposit it in his bank account. This bank will collects money by contacting the bank where the check's original account resides. 

            E-money: Debit card(no overdrawing); Credit card(allow overdrawing with overdraw line); stored-value card/z-purse (allow offline as data is in the chip);e-cash; 
    \subsection{Measuring Money}
        In America: 

        \quad $\text{M}_0$: cash / currency; 
        
        \quad $\text{M}_1$: $\text{M}_0$ + Traveler's check + demand deposits + other checkable deposits (USA); They are real-purchasing power that they can directly pay for goods and services; Narrow money; 
        
        \quad $\text{M}_2$: $\text{M}_1$ + quasi-money (small-denomination time deposits + savings deposits and money market deposit accounts + money market mutual fund shares); They can't be used directly to pay for goods;

        In China: the deposits in China is especially for individuals

        \quad $\text{M}_0$: cash in circulation;
        
        \quad $\text{M}_1$: $\text{M}_0$ + demand deposits of \emph{enterprises}; Individual demand deposits are excluded as China doesn't allow check for individuals.

        \quad $\text{M}_2$: $\text{M}_1$ + time deposits of enterprises + saving deposits + other deposits

\section{Interest Rate}
    \subsection{Measuring Interest Rates}: 
        The proportion of a sum of money that is paid over a specified period of time: simple interest and compound interest

        $$I_S = P \times i \times n,\quad  S_S = P \times (1 + ni)$$
        $$I_C = S_C - P,\quad S_C = P(1 + i)^n$$

        Discounting the future: 
        $$PV = \frac{FV}{(1 + r)^n}$$

        Annuity: ordinary annuity, annuity due, differed annuity and perpetual annuity 

        \subsubsection{Four types of Credit Market Instrument} 

        \quad Simple Loan:  Lender will repay $P + I$ in the maturity date; Example on money market short-term instruments

        \quad Fixed Payment Loan(fully amortized loan): Lender will repay same amount in periods, which is actually the form of annuity. The start of repayments contain mainly interest and the end of repayments contain mainly principal. e.g. mortgage

        \quad Coupon Bond: : Lender will repay same amount of interest in periods, and will finally pay out interest and principal the tha last period; This is used in capital market instruments;

        \quad Discount Bond(Zero-Coupon Bound) : Borrower will lend at $(P - I)$ and get paid of $P$  

        \subsubsection{How to calculate interest rate}

            Theses are abstracted annual interests.

            Yield to  Maturity: the interest rate that equates the present value of cash flow payments received from a debt instrument with its value today. This is the same concept with IRR; When coupon bond is sold at par, the real interest rate is equal to coupon bond no matter what the term is as every year the lender pay out all the interest without any principal. The interest rate is negatively related to current price of the bound; The lower of actual price you buy the coupon bond, the more yield to maturity.  

            $$P = \sum_{t = 1}^{n}\frac{CF_t}{(1 + i)^t}$$

            For discount bond: 

            $$i = \frac{F - P}{P}$$

            current Yield (for coupon bound):

            $$i = \frac{C}{P}$$

            For coupon bound is not sold on face price:

            $$YTM =  \frac{c + \frac{P_s - P_b}{\text{year}}}{P_b} = i_c + \frac{P_s - P_b}{\text{year}\times P_b}$$

            Discount yield (for discount bound):
            $$i = \frac{F - P}{F} \times \frac{360}{\text{days to maturity}}$$

            $\frac{1}{F}$ will understate the interest rate; $\frac{1}{\text{days}}$ is to evaluate the per day interest; So for annual interest rate as 360 < 365 this will also understate the interest rate. 

            Consol or perpetuity: it's a perpetual bond with no maturity date and no repayment of principal that makes fixed coupon payments of C forever.
        
        \subsection{Rate of return} 
        The payments to the owner plus the change in value expressed as a fraction of the purchase price. 

        $$\text{Ret} = \frac{C}{P_t}(\text{current yield}) + \frac{P_{t + 1} - P_t}{P_t \cdot \text{year}}(\text{rate of capital gain})$$

        $P_{t+1}$ is the present value of future cash flow bought by this bond.
        
        The return on a bond is equal to the yield to maturity in the circumstances of one-year coupon bond. Bonds whose term to maturity is longer than the holding period are subject to interest-rate risk, as market interest-rate increases will lead to loss in return; The more distant a bond's maturity, the lower the rate of return that occurs as a result of an increase in the interest rate. Even if a bond has a substantial initial interest rate, its return can bee negative if market interest rates rise. There is no interest-rate risk for any bond whose time to maturity matches the holding period;

\end{document}