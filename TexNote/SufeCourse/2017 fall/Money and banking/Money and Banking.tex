\documentclass[10pt, a4paper]{article}
    \author{ianaesthetic}
    \title{Unity3D note}
\usepackage{indentfirst, amsmath, fontspec, listings, xcolor}

\newfontfamily\consolas{Consolas}
\lstset{numberstyle = \small\consolas, basicstyle=\small\consolas}

\XeTeXlinebreaklocale "zh"
\XeTeXlinebreakskip = 0pt plus 1pt

\begin{document}
    \section{Introduction} 
        Study: Financial Market, Financial Institution, Financial Management.
        \subsection{Financial Market} 
        \textbf{Financial Market}: Markets where the funds flow from lenders to the borrowers. It's the \emph{channel} funds from servers to investors and can \emph{promote} economic efficiency.
        \subsubsection{The Bond Market and Interest Rate}
            A \textbf{security (financial instrument)} is a claim on the issuer's future income or assets.  it includes \textbf{Bond} and \textbf{Stock}.

            \textbf{Bond} is a debt security that promises to make payments periodically for a specified period of time. 
            
            \textbf{Stock}: Common stock represents a share of ownership.

            \textbf{Interest Rate}: Cost of borrowing or the price paid for rental of funds. When it increases, It can affect consumption, saving and investment. 
        \subsubsection{The Foreign Exchange Market}
            \textbf{Definition}: where funds are converted from one currency into another. 

            \textbf{Foreign Exchange Rate}: the price of one currency in terms of another currency. It mainly influence the imports and exports. There three methods of quotations : 
            
            \textbf{Direct}: $100\ FOREX = e_x\ Domestic\ Currency$ 

            \textbf{Indirect}: $100\ Domestic\ Currency = e_y\ FOREX$

            \textbf{USD} $1\ USD = e_z\ Domestic\ Currency$
            \newline 

            (de)appreciate: the influence of market; 
            
            (de)valuate: the influence of government

            US Dollar Index(USDX): To evaluate the value of Dollar by exchange rate with other countries. 

            \subsubsection{Factors of Foreign Exchange Rate}
                The Exchange rate is something rated to the Demand \& Supply of foreign currency. For e - Q(foreign currency) graph, it satisfies the normal demand and supply curve. All the exchange rate is discussed in direct quotations. 

                Balance of Payment(BOP):\emph{surplus} (FOREX $\uparrow$, e $\downarrow$),\emph{deficit}(FOREX $\downarrow$, e $\downarrow$).
                
                Economic Performance can be apparent in some time point(end of the year). Good performance will lure foreign cash(e $\downarrow$).
                
                Interest rate will also affect(i $\uparrow$, e $\downarrow$, for it can lure foreign investment). 

                Price will affect as well(p $\uparrow$, e $\uparrow$).
    \subsection{Banking and Financial Institutions}
        \textbf{Functions}: 

            $a$. They make financial markets work 

            $b$. Financial intermediary for funds flowing from savers to investors

            $c$. Important effects on the performance of the economy as a whole
        
        \textbf{Examples}: 
            
            \quad Insurance Company; 
            
            \quad Banks; 
            
            \quad Securities Firm; 
            
            \quad Trust Company; 
            
            \quad Credit Union; 
            
            \quad Financial Company;
            
            \quad Financial Leasing Company;
            
            \quad Credit Rating Agency; 

            \quad Exchanger; 
            
            \quad Funds Management; 

        All examples can be divided to two types: \emph{banks-institution} and \emph{non-banks-institution}. Banks are the largest financial intermediaries in our economy, including Central Bank; Commercial Bank; Policy Bank(non-profitable); Specialized Bank. There is a trend of \emph{disintermediation}.
        
        \subsubsection{Financial Innovation}
            Anything new in Finance. 

            New financial product, financial institution, financial services and more appear, such as e-finance and financial derivatives. 
        \subsection{Money and Money policy}
            \textbf{Definition}: Money is defined as anything that is \emph{generally} accepted in payment for goods or services or in the repayment of debts. It is linked to changes in economic variables that affect all of us and are important to the health of the economy. 

            Money affect \emph{business cycle} (including four stages: \emph{recession},\emph{depression} \emph{recovery} and \emph{boom}. 

            Money growth rate will have a severe decrease and rapid growth. The recession will cause the money decreasing, for example, people are don't intend to consume. After a big recession, government will get in and add the amount of money in the market to boost economy. This phenomena can be used to  predict the performance.
            
            \subsubsection{Relationship between Money and Inflation} 
                \textbf{The aggregate price level}: the average price of goods and services in an economy. 
                
                \textbf{Inflation}: A continue rise in the price level which affects all economic players. 

                The rise of money supply will lead to the rise of inflation. 
            \subsubsection{Money and Interest Rates}
                Interest Rates are the price of money. The increase of money supply will lead to the decrease of interest rate, something like demand and supply.
            \subsubsection{Monetary and Fiscal Policy}
                \textbf{Monetary policy} is the management of the money supply and interest rates. (Central bank) 

                \textbf{Fiscal Policy} is government spending and taxation, which is set by department of Treasure. (fiscal revenue and expense, the department of treasury) 

                Both policies make government be able to manage and manipulate markets. Government raise money from taxation, profit of state-owned company, fee, etc. Government spends on procurement, investment, transfer payment,interest payment, etc. 

                The effect on aggregate demanding($C + I + G + NX$): Money policy influences C, I, NX, has a indirect effect to AD. Fiscal policy will have a direct effect on G and AD.  

                Interest rate change($\uparrow$, as example) will affect exchange market, stock market($\downarrow$, as it's more difficult to raise money), national bond market, commercial market($\downarrow$, as it will lower the need for consumption).
    
    \section{Financial Market}
        \subsection{Function of Financial Markets} 
            Perform the essential function of channeling funds from lenders to borrowers. It can also promote \emph{economic efficiency} by producing a efficient allocation of capital. It will also directly improve the well-being of consumers by allowing them to time purchases better (allow consumers to use the money in the future to purchase by loans or something else).
            \subsubsection{Channel of Financing}
                \textbf{Direct Finance}: Borrowers borrow funds directly from lenders in financial markets by selling them securities. More specifically, the relationship between lender and borrower is direct and clear. 

                \textbf{Indirect Finance}: Involves a financial intermediary that stands between the lender-savers. The relationship between intial lender and final  borrower is indirect. (Banks)

                The graph of flow of funds can be referred to ppt.  
        \subsection{Structure of Financial Market}
            \subsubsection{Debt and Equity Markets}
                \textbf{Debt instrument} is a \emph{contractual agreement} by the borrower to pay the holder of the instrument fixed dollar amount(interest and principal payments) at regular intervals until a specified date(the \emph{Maturity Date}, when a final payment is made, includes: \emph{bonds}, \emph{mortgage}.

                It has different terms includes: 
                
                \quad short-term($M < 1yr$): T-Bill

                \quad intermediate term($1yr < M < 10yr$): T-Note
   
                \quad long-term ($M > 10yr$): T-Bond \newline

                \textbf{Equity instrument} is claim to share in the net income(income after expenses and taxes) and the asset of a business, includes: \emph{common stock}, \emph{preferred stock}. 

                The difference between common stock and preferred stock: Preferred stock has priority in dividends receiving and liquidation during the corruption. Common stock owner has \emph{voting right} based on the share owned while limited in priority stock owner. And the common stock has priority to buy new shares to avoid dilution. From the crisis perspective, common stock holder has bigger crisis than preferred stock. 

                Priority stock has two type: accumulative and non-accumulative. Accumulative has a right to claim the unpaid dividends in the later year while non-accumulative can just claim current dividends. Accumulative stock is more analogous to long-term bond. 

                The main disadvantage of owning a corporation's equities rather than its debt is that an equity holder is a \emph{residual claimant}.
            \subsubsection{Primary and Secondary Markets}
                \textbf{Primary markets} are those where new security issues sold to initial buyers. Investment Banks will \emph{underwrite} securities in primary markets.  

                \textbf{Second markets} are markets where securities previously issued are bought and sold between investors. Broker(agency) and dealers  work in secondary markets. 
            \subsubsection{Exchanges and Over-the-Counter(OTC) Markets} 
                \textbf{Exchanges markets}: Trades conducted in central locations(eg. New York Stock Exchange, NASDAQ)      
                \textbf{OTC markets}: Dealers at different locations buy and sell(eg. OTCBB).  
            \subsubsection{Money and Capital Markets}
                \textbf{Money Markets} deal with short-term debt instruments. 
                
                \textbf{Capital markets} deal with longer-term debt and equity instruments. 
                
            \subsubsection{Money Markets Instruments}
                Treasury Bills[TB]: Short-term; A solution to the government deficit; IT has the lowest rate in the markets, so called gilt-edged / risk-free  bond, as it's guaranteed by the taxation of the government. They are usually sold at discount by auction to generate interest.
                
                Negotiable Bank Certificates of Deposit (large dominations) [CD]: Large domination; Transferable in second market which can receive the interest before maturity; Of low risk as it's published by guaranteed banks and they absorbs funds quickly.
                
                Commercial paper: a unsecured promissory note with a fixed maturity less tha 270 days and is published by well-known cooperation; In selling on credits, in the credit term, there is no interest and will form accounts receivable/payable; When exceeding the credit term, they will change to note receivable/payable with interest. Notes include Promissory note and draft.
                
                Draft is an order, by the creditor for the debtor to pay fo a payee. Promissory note is a promise by debtor to pay back to the creditor. \emph{Acceptance} is needed in the draft by the payee to agree with it; A similar \emph{trader's acceptance} will be in the promissory from the debtor. Bank can charge the debtor to stamp this trader's acceptance and it will be called \emph{bank acceptance} which has more liquidity as bank has better reputation. \emph{Endorsement} is transferring notes to other to retrieve money before maturity. \emph{Discount} is transferring notes to a bank before maturity subtracting the interest.   


                Federal Funds: a form of inter-bank offering as are borrowed between financial institution; short-term concentrating on overnight borrowings; large amount for immediate spending; interest rate is liberalized for inter-bank offering expect for Federal Funds;
                
                Repurchase agreement[RP]: After A sells a low-risk bond to B, RP is the scene that A buys back the bond at a higher price later. In fact, it's a loan for A with bond. A is positive repo and B is negative repo. For a, the bond still belongs to A when A is in need of funds. For B is safer to give the loan in this way with bonds as pledge. For normal pledged repo, the pledged bonds can't be pledged again. While outright repo allow for a shorter term of transition.    

            \subsubsection{Capital Market Instruments}
                Capital market instruments are for long-term. 

                Bonds: T-Note \& T-Bond with large amount and low risk; 

                Government agency securities

                State and local government bonds 

                Cooperate bonds: with a relative hight risk, which introduces credit rating; It includes converted bonds, which can convert the bonds to stock at a price. 

                corporate stocks
                
                commercial loans, consumer loans, commercial and farm mortgages, residential mortgage.
    \subsection{Internationalization of}
        \subsubsection{foreign exchange}
            Euro-currencies: foreign currencies deposited in banks outside the home country.

            Euro-dollar: dollars deposited in foreign banks outside the US. 

            Euro is actually refers to offshore. 
        \subsubsection{World Bonds Market}
            Foreign bonds: sold in a foreign country and denominated in that country's currency. 

            Euro-bond: bond denominated in a currency other than that of the country in which it is sold. 
        \subsubsection{World Stock Markets}
            Stock price indices: composite indices and component indices.
    \subsection{Financial Intermediaries}
        \subsubsection{Types of Financial Intermediaries}
            Depository institutions: commercial banks, saving banks, credit union;  

            Contractual savings institutions: life insurance companies, fire and casualty insurance companies, pension funds;
            
            Investment intermediaries: Finance companies, mutual funds, money market mutual funds; 
\end{document}